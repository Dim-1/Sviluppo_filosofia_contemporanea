\chapter{Il passaggio dalla filosofia dell'essere alla filosofia del soggetto}

\bigskip
\bigskip
\bigskip
\bigskip
\bigskip


Da Cartesio in avanti il soggetto è l'elemento indubitabile e inaggirabile, che non può essere evitato, in quanto la sua messa in discussione lo richiama sempre in causa. L'argomento cartesiano "Cogito, ergo sum" è detto confutativo, ovvero la messa in discussione del soggetto lo presuppone, quindi la confutazione stessa decade; meccanismo simile fu applicato da Aristotele nel IV libro della metafisica riguardo il principio di non contraddizione. Questo argomento è detto anche trascendentale, ovvero intrascindibile, non aggirabile, che ritroviamo sempre alle spalle.

In epoca moderna quindi, il paradigma della filosofia passa dalla centralità dell'essere e della sostanza al paradigma del soggetto (coscienza, spirito). Il paradigma ontologico degli antichi poneva come base l'essere delle cose, cioè l'indubitabilità dell'esistenza delle cose: la base del mondo è la sostanza (ousìa), e la sostanzialità è la proprietà che hanno le cose di esistere. Che un ente sia sostanza, da Platone in avanti, significa che ha in se stesso le ragioni della sua esistenza, che è fondato su se stesso: ousìa struttura fondamentale delle cose.

Per Platone, la ragion d'essere degli enti non sta in questo mondo, ma nell'iperuranio, nel mondo metafisico: ousìa sono le idee, sono oltre il sensibile e sono la ragion d'essere del mondo. Per Aristotele le sostanze sono nel mondo fisico, ovvero gli enti fisici hanno in loro stessi la ragione della loro esistenza. Aristotele però è un critico del naturalismo: ciò che consente ad un certo ente di essere, di avere sostanza, è la "forma" (eidos); la forma è la vera sostanza, che nel mondo fisico organizza la grezza materia per conferirgli sostanzialità. Ad allontanare ulteriormente Aristotele dal naturalismo è la giustificazione del divenire e del mutamento delle sostanze con una causa metafisica, il motore immobile, il pensiero di pensiero. La posizione aristotelica vede la ragione di esistenza delle cose nelle cose stesse, mentre nell'aldilà metafisico si trova la causa del movimento e del mutamento.

Per gli antichi l'essere è indubitabile, non c'è ragione di dubitare che le cose siano sostanza (cioè che hanno la proprietà di esistere), e la tesi della metafisica antica è che il fondamento del mondo fisico non sta in questo mondo: questo fondamento intelligibile è assoluto, cioè indubitabile, necessario e incontrovertibile. \textbf{La verità dell'aldiquà sta nell'aldilà.}

Per Cartesio l'indubitabilità passa dalla sostanza al soggetto, da Dio alla coscienza, dall'essere al pensiero. La metafisica non è abbandonata del tutto, però il fondamento primo trasla dalla sostanza al pensiero.

Conseguenza del paradigma cartesiano è la dubitabilità di tutto ciò che non è pensiero o soggetto: l'esistenza del mondo viene per la prima volta messa in dubbio. Dal punto di vista del soggetto, qualsiasi cosa esso si rappresenta è dubitabile e quello che per gli antichi era l'essere, ovvero il mondo, diventa nostra rappresentazione del mondo: le uniche cose vere sono il soggetto e le sue rappresentazioni, mentre è dubitabile che dietro la rappresentazione vi sia un mondo reale. La rappresentazione soggettiva è un'idea, ovvero le idee platoniche che davano sostanzialità al mondo, sono diventate proprie rappresentazioni soggettive.

Il compito della metafisica passa dall'essere il fondamento del mondo fisico o del suo divenire, a servire come dimostrazione dell'esistenza del mondo; precisamente Dio garantisce solo l'esistenza di un mondo esteso, che occupa spazio, un mondo geometrico con l'unica caratteristica di avere estensione ("res extensa"). Colori, suoni, durezza e ogni altro dato sensibile sono solo rappresentazioni soggettive; questo mondo postulato da Cartesio è solo esteso, senza altre caratteristiche, non è riproducibile o visualizzabile nella realtà (pure una sola linea si distingue per il contrasto o il cambiamento di colore), è puramente nero o ancor più astratto, è un mondo che può essere solo pensato come geometria, in termini geometrici di assiomi, teoremi, \dots.

La filosofia moderna successiva a Cartesio mette in discussione proprio la possibilità di dimostrare un mondo esterno, punto debole del pensiero cartesiano sin dalle sue fondamenta, basate sulla dimostrazione a priori dell'esistenza di un Dio garante del mondo. I successivi pensatori lasciano il soggetto e via via tolgono importanza alla metafisica.

\begin{itemize}
	\item \textbf{Leibnitz} sostenne che non siamo in grado di uscire dalle nostre rappresentazioni; la "monade" è un universo totalmente chiuso ("ne porte ne finestre"), che non comunica con le altre monadi.
	\item \textbf{Berkeley} sostenne che l'essere delle cose si risolve completamente nella percezione delle cose ("Esse est percipi").
	\item \textbf{Hume} sostenne che l'uomo non dispone di nessun argomento che possa dimostrare l'esistenza del mondo ("è vano domandare se i corpi esistano oppure no"): non sapendo niente riguardo la possibilità dell'esistenza del mondo, l'unica indagine deve riguardare il nostro intelletto, per comprendere perché riteniamo il mondo esistere. L'indagine di Hume è psicologica, mira a capire come si forma l'idea dell'esistenza del mondo, di cui si può dare solo rappresentazione.
	\item \textbf{Kant}, coerentemente con Cartesio e con lo spostamento del paradigma sempre più verso il soggetto, elimina la metafisica: la res extensa decade, nel sistema kantiano spazio e tempo diventano organizzazioni soggettive dei dati sensibili, attraverso le forme a priori dell'intuizione; inoltre il mondo esterno all'uomo acquisisce nuove caratteristiche (unità, causalità, possibilità, \dots), che hanno origine nelle forme a priori dell'intelletto, concetti puri attraverso cui il mondo viene catalogato. Da ciò ne deriva che l'esperienza del mondo ha una componente sensibile e una intellettuale, è il prodotto di forme a priori del soggetto, e che tutti gli oggetti che esperiamo sono una produzione delle categorie. Al di là delle forme a priori c'è solo l' "in sé" delle cose, che dobbiamo presupporre esistenti, altrimenti i nostri sensi non riceverebbero i dati sensibili; delle cose in sé non possiamo conoscere niente, non possono essere predicate.
\end{itemize}

