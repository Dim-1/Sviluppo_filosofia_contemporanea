\chapter{Nietzsche e le conseguenze radicali della dissoluzione dell'idealismo}
\bigskip
\bigskip
\bigskip
\bigskip
\bigskip
\bigskip


La filosofia di Friederich Nietzsche (1844-1900), con la dottrina dell'eterno ritorno e la teoria della volontà di potenza, attesta l'esito inevitabile cui doveva condurre la dissoluzione del sapere assoluto hegeliano. \textbf{L'eterno ritorno indica che la realtà non ha più un significato, in quanto gli viene tolto qualsiasi fondamento logico}: se tutto ritorna, qualsiasi sviluppo storico è messo in discussione, ed inoltre \textbf{vi è una sostanziale equivalenza del tutto, l'impossibilità di scale di valori, di gerarchie; se tutto ritorna,  l'idea di realtà è privata di razionalità e logicità.}

\textbf{La volontà di potenza mostra che la volontà è resa libera dalla verità, o meglio fa dipendere la verità dalla volontà stessa}, una volta tolti i vincoli di presunte verità logiche o oggettive. L'idea moderna di soggettività verrà ridimensionata, in quanto anche essa fondata sulla superiorità del pensare e della ragione, che, come vedremo, soggiacciano alla stessa volontà di potenza.

\section{Nichilismo ed eterno ritorno}

Il nucleo centrale del "pensiero più abissale" di Nietzsche sta nella messa in discussione della struttura lineare del tempo e perciò la sua subordinazione a logiche che pretendano di mostrarne lo sviluppo e il progresso: nessun momento è privilegiato all'interno del tempo; se tutto ritorna, tutto è uguale, tutto è equivalente. \textbf{Ogni attimo esaurisce la totalità dell'essenza, perché tutto è attimo}.
E' un annuncio con connotati nichilistici, venendo meno ogni fondamento, ogni valore e ogni verità. L'esistenza è senza senso e senza scopo, ma ritornante: \textbf{il nichilismo è  mancanza di significato del mondo e della vita, svalorizzazione dei valori supremi. Senza "cosa in sé" (costituzione assoluta delle cose) si è giunti al nichilismo più estremo, in cui la totalità è priva di senso.}

Questo pensiero è "terribile" solo per chi è rimasto legato alla vecchia idea metafisica di una verità in sé delle cose. \textbf{Per chi si è liberato di questa visione, l'annuncio dell'eterno ritorno è il guadagno di una nuova dimensione dell'esistenza, quella dell'innocenza del divenire: il divenire delle cose non porta da nessuna parte, non ci sono mete, ma tutto l'essere è pieno di significato, quello che gli attribuiamo.}

Il vero nichilismo è la convinzione che il mondo non sia nulla, e questo è l'esito di un processo che affonda le radici nelle origini platonico-cristiane della nostra civiltà, dove la volontà di trascendenza ha mascherato la volontà del nulla. Il mondo è stato ridotto a nulla, per fondare su questo nulla l'esistenza di Dio; successivamente la stessa civiltà creatrice di Dio ha riconosciuto la nullità di questo essere supremo, portando a compimento il nichilismo.

Il nichilismo estremo, la consapevolezza che non c'è più alcuna verità, ne al di là ne al di qua, è la necessaria conseguenza degli ideali finora coltivati, ed è comunque \textbf{il passaggio necessario verso il superuomo}, colui che porrà nuovi valori, una volta compreso che non esistano valori assoluti.

La fede in Dio è legata alla fede nella ragione, alla fiducia nella struttura razionale del mondo; \textbf{la morte di Dio porta con sé la morte del platonismo, cioè dell'intera configurazione morale, logica e razionale della nostra civiltà, venendo meno qualsiasi capacità di orientamento dell'uomo.}

\section{La volontà di potenza}

Eterno ritorno, innocenza del divenire, nichilismo, non sono eventi oggettivi e indipendenti dalla nostra volontà. Non vi è nel pensiero nietzscheano l'idea di una storia oggettiva da accettare passivamente, altrimenti il platonismo sarebbe superato solo in apparenza (la verità trascendente verrebbe sostituita da una immanente e circolare), perché rimarrebbe la pretesa di dire l'essenza ultima delle cose e \textbf{la metafisica} riconfermerebbe la sua inaggirabilità. Quest'ultima \textbf{è veramente superata quando la verità viene ricondotta al di sotto della volontà e da essa fatta dipendere.}

\subsection{La critica della conoscenza}

Quello che produce \textbf{il nichilismo (dovuto alla perdita di logica e razionalità del divenire che l'eterno ritorno comporta)} è la rivelazione che, dietro le eterne strutture dell'essere (e scientifiche) non c'è nessuna verità ma solo un apparato di produzione, di interpretazione. \textbf{La verità è qualcosa da creare e che da il nome al processo, è volontà di soggiogamento}, che di per sé non ha mai fine: introdurre la verità è un processo infinito, un attivo determinare (o passivo se siamo animati da forze negative, come ben ha fatto notare Gilles Deleuze).

\textbf{La dipendenza della verità dalla volontà di potenza fa del vero un processo in divenire: un divenire la cui logica è fatta dipendere dalla volontà stessa}, e quindi che non è fissa come la logica dialettica hegeliana. \textbf{Se esiste una volontà di imporre la propria prospettiva perché ci è più utile, più conforme ai propri fini, significa che il mondo non si presta  a essere compreso totalmente. La volontà di rendere l'essere conoscibile, di dominarlo attraverso il pensiero, di ridurre il mondo a pensabile significa formulare dei valori, fissare il divenire in alcune forme a partire dalla valutazione.}

"La verità sono illusioni di cui si è dimenticato la natura illusoria", è un artificiale apparato di falsificazione, funzionale al nostro istinto di sicurezza, a rendere possibile e sopportabile la nostra esistenza. La "verità" ci fornisce l'immagine di un mondo stabile, prevedibile, allontanando la paura di un mondo sconosciuto e imprevedibile, quindi pericoloso; conoscere e il "cercare la regola (della scienza e non solo)" serve a rendere l'estraneo a noi conoscibile.

Se il vero è determinato dalla volontà, ha senso mostrare la sua genesi biologico-antropologica ("genealogia"), ma non porre il problema della sua validità. Il metodo genealogico non fonda niente (come invece pretende di fare la metafisica), non giustifica, bensì vuole solo mostrare. In ultima analisi, \textbf{la verità non consiste in una costruzione in sé delle cose, ma nel processo della sua produzione}.

\subsection{Congedo della metafisica}

La critica di Nietzsche della verità è la critica di ogni pretesa metafisica, di ogni pretesa di conoscere l'essenza del mondo, di svelare il senso del tutto, di sovrapporre la propria significazione, le proprie spiegazioni  e ragioni al divenire del mondo. \textbf{La metafisica consiste, in definitiva, nella mancata accettazione dell'insensatezza delle cose e dell' "innocenza del divenire"}, nel voler spiegare il mondo e il cambiamento ad ogni costo.

\textbf{Volendo trovare una spiegazione, una ragione, un logos, la metafisica finisce per creare una seconda dimensione rispetto a quella reale, la dimensione del senso, al di là ed oltre a quella dell'evento} (a quella del \textit{pràgmata} sosterrebbe Heidegger). Questa pretesa di un "significato in sé" è assurda altrettanto quanto la "cosa in sé". Per Nietzsche l'uomo inventa la dimensione dell'essere come spiegazione alla dimensione del divenire, riconduce il divenire all'essere come al suo significato ultimo, proietta il suo "impulso di verità, il suo fine", come mondo dell'essere (metafisico), come "cosa in sé", come mondo già esistente. Perciò la metafisica è sempre dualistica, anche quando non teorizza un mondo trascendente.

Il congedo della pretesa del senso è il definitivo abbandono di  ogni dualismo tra essere e apparire: "il mondo vero è solo un'aggiunta mendace". Il superamento della metafisica è perciò il superamento dei bisogni umani di far fronte ad un mondo privo di senso, l'oltrepassamento della sua decadenza.

Insieme alla metafisica è congedato Schopenhauer e la sua metafisica della volontà, che, pur sbarazzandosi di Dio e mostrando un mondo privo di fini, pretende di svelare la cosa in sé, la sostanza ultima del mondo (da svelare e poi contrastare asceticamente negando la propria individualità); \textbf{per Nietzsche la volontà è invece la nostra capacità di affermazione individuale, la nostra imposizione di senso a un mondo che ne è privo.}

\subsection{La conoscenza come produzione linguistica e interpretazione}

Nietzsche anticipa il concetto ermeneutico di verità, cioè che il vero si presenta all'interno della prassi di vita e del processo dell'interpretare. In ciò il linguaggio ha un ruolo decisivo, in quanto da forma stabile e consolidata al mondo: crea un'immagine che poi noi riteniamo proprietà intrinseca. \textbf{Conosciamo il mondo attraverso il linguaggio, esito di una produzione metaforica, di creazione di relazioni delle cose con gli uomini.}

\textbf{Il concetto stesso è il "residuo della metafora" e la verità una somma di relazioni umane consolidate, tanto da dimenticarne l'origine antropica, e la verità, da relazioni umane, è poi considerata relazioni in sé delle cose.}

\textbf{Non esiste un mondo di fatti, di dati, di cose, ma solo un universo di simboli e di interpretazioni; inoltre, essendo all'interno di una certa prospettiva di mondo, non possiamo appunto conoscere che la nostra conoscenza è prospettica, che la nostra verità sia interpretazione. Scopo del meccanismo genealogico è proprio svelare questo gioco di prospettive una dentro l'altra}. \textbf{La conoscenza si risolve in un processo ermeneutico, infinito}\footnote{in quanto se la totalità della conoscenza è interpretazione, ogni interpretazione è interpretazione di un'altra interpretazione, senza mai arrivare ad un dato ultimo che fermi il processo.}: la conclusività hegeliana di un sapere assoluto è definitivamente spezzata.

Nietzsche non arriva a teorizzare un'esperienza della verità nell'interpretazione: a questa conclusione perverrà l'ermeneutica successiva ad Heidegger (soprattutto con Gadamer); anzi la critica nietzscheana si accompagna anche ad una critica radicale della ragione e della logica, mettendo fuori ogni possibilità di introdurre un concetto di validità, necessario per fare un'esperienza di verità all'interno di un orizzonte interpretativo. \textbf{Per Nietzsche infatti ha importanza l'atto creativo e affermativo, con cui creare sempre nuovi valori e interpretazioni, nuove prospettive di vita che gioiscano  dell'esistenza}, anche nei momenti dolorosi (di cui bisogna ridere e che non vanno giustificati metafisicamente come fatto negli ultimi due millenni): \textbf{il dionisiaco non conosce la verità (cui nessuno arriva), la vive.}

\subsection{La riduzione della razionalità a volontà}

Nietzsche mette in discussione anche gli strumenti logici con cui possiamo stabilire la validità del mondo: la razionalità è vista come un prodotto della decadenza, cioè di quella insufficiente forza che caratterizza l'uomo moderno per cui non è in grado di sopportare il mondo illogico.
Con Socrate, insieme alla ragione, nasce la morale: entrambe frutto del desiderio di vendetta della "plebaglia", che vuole imporsi là dove non ha la potenza di imporsi.

\textbf{Dietro la logica e la razionalità c'è dunque la volontà: la volontà di imprimere un senso alle cose, di sopravanzare, di sicurezza, in ultima analisi di sapere a tutti i costi}, tanto da arrivare a creare verità fittizie (come la matematica e la geometria) per rendere calcolabile uno schema d'essere da noi posto come reale.

La logica non esprime alcuna verità oggettiva, bensì è solo un imperativo che stabilisce cosa debba valere come vero, ordina un mondo che deve essere vero per noi. \textbf{La verità o la falsità di un giudizio non hanno senso; invece hanno significato la fede in esso e quanto sia importante per la vita e la sopravvivenza.}

\subsection{Oltre il nichilismo}

\textbf{Non basta ricondurre la verità alla volontà, ma bisogna superare la stessa volontà di verità (nata per sopportare l'illogicità del mondo): il passaggio dalla volontà di verità alla volontà di potenza è il vero salto oltre il nichilismo. Per il nichilista la verità è un'invenzione, ma continua ad averne bisogno e dunque si dispera proprio perché sa che non esiste un mondo vero: ammette la realtà del divenire (vedi la figura dell'indovino nello Zarathustra), senza sopportarla e senza sostenere un mondo privo di senso.}

Da ciò si prepara il superamento del nichilismo (che è una fase intermedia e di passaggio per l'umanità), l'avvento di una volontà di creare, della "transvalutazione di tutti i valori" e il passaggio ad un atteggiamento positivo di creazione di nuovi valori.

Poiché non esiste un "in sé" delle cose, il passaggio dalla volontà di verità a quella di potenza è il passaggio dal bisogno che le cose stiano in un certo modo al volere una certa configurazione del mondo: vero è ciò che viene posto dalla volontà, la configurazione organizzata di quella certa parte di mondo su cui la volontà esercita la sua forza.

Una verità così intesa è chiaramente illusione, che rimane l'unica verità possibile: \textbf{i veri filosofi non sono coloro che contemplano la verità, ma "coloro che comandano e legiferano", che non conoscono bensì creano}.

\subsection{La critica del soggetto e la trasformazione del trascendentale}

Questo cammino di demistificazione dell'oggettività del vero non conduce Nietzsche ad una prospettiva soggettivista: il soggetto stesso è un valore, dunque un prodotto della volontà e necessario alla vita, come il bisogno di sicurezza e di verità.\textbf{ Se tutto è interpretazione, non è la coscienza umana il soggetto dell'interpretare.}

\textbf{Il soggetto non è interprete ma interpretato, esso stesso è una maschera, costruito a partire dal linguaggio e dalla logica, i quali dunque non dipendono da lui. La soggettività non è più vista in grado di costituirsi come punto di vista a partire dal quale interrogare la totalità.}

Dunque il prospettivismo ha un significato ancora più ampio, ovvero il mondo ha dietro di sé innumerevoli sensi; tuttavia questo prospettivismo non è applicabile a se medesimo, cioè non può dirci cosa vale per vero.

Quale è quel punto di vista superiore al soggetto che vale per Nietzsche come vero? Qui abbiamo un'oscillazione da parte del filosofo: da un lato esso viene inteso come punto di vista empirico-biologico-naturale ("sono i nostri bisogni, che interpretano il mondo"; "Chi interpreta? I nostri affetti"). Quindi la biologia sostituisce la metafisica dell'anima: verità, ragione, soggettività, morale, logica sono ricondotte ad una base costitutiva fisiologica ("esigenze fisiologiche di una determinata specie di vita").

D'altra parte resta \textbf{da spiegare come Nietzsche possa mettere in discussione la costituzione oggettiva del mondo, se poi il punto di vista da cui avviene la sua dissoluzione è collocato daccapo all'interno del mondo medesimo. Non si può dire che non esistano i "fatti in sé" e poi prendere per validi i "fatti" biologici come base genetica del processo dell'interpretare}. Da ciò la necessità per Nietzsche di sollevare  ulteriormente il punto di vista interpretativo al di sopra del ambito empirico, e di pensarlo come un quadro trascendentale.

\textbf{Il punto di vista superiore alla vita biologica (all'istinto di conservazione) è la volontà di potenza, il trascendentale al di là del soggetto: "un'entità vivente vuole soprattutto scatenare la sua forza -- la vita stessa è volontà di potenza -- l'autoconservazione è soltanto una delle più indirette e più frequenti conseguenze di ciò".}

A partire dalla volontà di potenza viene posta e giustificata la tesi dell'eterno ritorno e la proposta positiva di una transvalutazione di tutti i valori. \textbf{E' la volontà ad interpretare ciò che eternamente ritorna, dandogli ogni volta un nuovo senso e affermando in modo differente, sempre diversamente, per cui il significato del divenire dipende dalla volontà: ciò che ritorna non è una ciclicità di eventi oggettivi e ripetitivi, ogni volta infatti cambia l'interpretazione}. Quando la volontà di potenza riesce ad imporsi sulla debole volontà di verità, non si da più un mondo vero ne apparente, ma solo il mondo che vogliamo.

Il superuomo, Zarathustra, può solo volere, rinunciando ad ogni presunzione di  sapere. \textbf{Volere è il trascendentale nietzscheano, la volontà è l'orizzonte indiscutibile.}


























