La filosofia dopo Hegel è stata caratterizzata
da una messa in discussione della vecchia concezio­ne del trascendentale, più che da un esplicita
tematizzazione dei nuovi orizzonti che si erano
formati. Il tema esplicito della filosofia
fino a Weber era stato la presa d'atto dell'
impossibilità di stabilire l'orizzonte originario
all'interno del quale fondare il sapere: non si
accetta più un piano trascendentale ultimo.

Con poche eccezioni, la filosofia del '900
mantiene il proprio piano filosofico al di sotto
del piano trascendentale, senza perciò poterlo
fondare e tematizzare.

Nel secolo passato abbiamo assistito all'unione
delle molteplici istanze nate dalla dissoluzione
del paradigma soggettivistico, ad un nuovo
territorio comune: il linguaggio, la dimensione
in cui il pensiero si esprime, per cui consente
un rapporto con il sapere filosofico meno pregiudicato in partenza rispetto alle strutture tematizzate dalla filosofia post-hegeliana.

\chapter{Il pensiero fenomenologico-ermeneutico}
\bigskip
\bigskip
\section{Gli inizi dell'ermeneutica}

Nel periodo romantico con Schleiermacher,
l'ermeneutica da tecnica dell'interpretazione diventa  una disciplina filosofica: l'interprete deve
trasferirsi interiormente nella soggettività dell'autore e riprodurlo in se stesso, soprattutto
attraverso il sentimento.

Con Dilthey l'ermeneutica finisce per coincidere
con l'essenza stessa della filosofia, e attraverso
il sapere ermeneutico apprendiamo la totalità delle
manifestazioni spirituali (cultura, scienze sociali, arte, \dots).
In Dilthey si assiste ad un'evoluzione da un'ermeneutica introspettiva (basata sull'esperienza
vissuta, Erlebnis) ad una focalizzata sull'espressione oggettiva e storica della vita.
Allargandosi alla totalità delle manifestazioni
dello spirito, l'oggetto specifico dell'ermeneutica
diventa il mondo storico.
Così si qualifica non solo l'oggetto dell'interpretazione,
ma anche il soggetto interpretante, che è determinato dalla vita storica (storicità): la trascendentalità non è più indipendente dalla coscienza
storica.

\textbf{Nell'interpretazione vi è "comunanza", perché "colui che
indaga la storia è il medesimo che fa la storia".}
Diverso è lo studio della natura, data l'assenza
di comunanza, che avviene secondo rapporti causali.
Il comprendere ermeneutico è un ritrovamento della soggettività che ha prodotto l'oggettività
esteriore; \textbf{"il soggetto del sapere è identico con
il suo oggetto".
Nella storia i soggetti manifestano loro stessi in
espressioni oggettive che vengono consegnate alla
comprensione degli altri. Tutte queste espressioni
sono il medesimo spirito che si manifesta, ciò
che accomuna noi tutti, e in esso rivive la nozione
hegeliana di spirito oggettivo, anche se per
Dilthey l'essenza di esso è la vita, cioè una
realtà che non ha la struttura trasparente e logica
dello spirito hegeliano} (che poteva risalire allo spirito
assoluto, cioè alla completa autotrasparenza
di sé e all'assoluta identità del soggetto con
l'oggetto).
Il circolo ermeneutico per Dilthey è incompleto,
dato il carattere vitale del processo stesso, cioè
lo spirito ha caratteristiche irrazionali, per cui ogni
interpretazione è solo relativa.
Il concetto di comunanza dell'ermeneutica di Dilthey
rivela che il vero soggetto ermeneutico è rappresentato
dalla connessione vitale che unisce soggetto e oggetto.
\textbf{Il soggetto della storia dello spirito è
la vita e le connessioni operanti tra gli individui
storici.} Queste relazioni intersoggettive non sono
relazioni esterne e dipendenti dagli individui, ma
hanno una loro dipendenza e oggettività,  risultando
come determinanti gli individui stessi. Sono perciò
queste relazioni il vero soggetto dell'interpretare
e della vita storica. Nell'ultimo Dilthey \textbf{si
passa dal soggetto inteso come orizzonte originario,
all'operare intersoggettivo degli individui
come soggetto autonom}o (non più prodotto degli
individui): in ciò non solo rivive lo spirito
oggettivo hegeliano e la tesi del primato
dei rapporti intersoggettivi rispetto all'autocoscienza, ma viene anticipata
l'originarietà dell'"altro" dal soggetto
e la posizione di una differenza irriducibile
ad esso, quale sarà tematizzata dai successivi
autori.

\section{La fenomenologia husserliana}

Edmund Husserl (Germania 1859-1938) definisce
l'impianto fondamentale della fenomenologia:
questa è la scienza dei fenomeni ed ha
caratterizzazione non empirica ma trascendentale.

\subsection{Dagli oggetti ai fenomeni}


Il carattere interpretativo introdotto da Nietzsche,
e tematizzato da Dilthey in relazione
alla conoscenza storica-spirituale, viene
ribadito nella fenomenologia husserliana e
esteso a tutta l'esperienza.
L'atteggiamento pratico-valutativo per Husserl
ha a che fare con l'apparato conoscitivo in
generale, così che l'esperienza stessa non è
una successione di dati di fatto, bensì una
costituzione di propri oggetti.
\textbf{Il mondo non è evidente come sembra: è invece
l'esito di moltissime integrazioni di significato
rispetto a ciò che effettivamente appare ai
nostri sensi.}

 Il compito della fenomenologia è
ricostruire, partendo da ciò che si presenta come
già costituito, questo terreno di evidenze
primarie, quell'esperienza primaria alla base
del mondo di cose e oggetti che noi abbiamo
costruito, un originario che non ha ne natura
corale ne oggettiva.
Secondo Husserl l'originario non ha le caratteristiche
della cosa in sé (materia extra soggettiva che
affetta la nostra sensibilità), ma è qualcosa
di immanente alla nostra percezione: \textbf{l'originario
è l'insieme delle percezioni elementari, ovvero quei
"vissuti" che non hanno ancora forma oggettiva e
trascendente} (trascendente per Husserl è tutto ciò che non riguarda il dato, ogni astrazione compiuta dalla coscienza) .
Come in Dilthey, ritorna il concetto di "Erlebnis",
la struttura semplice del percepire non ulteriormente scomponibile, a partire dal quale costruiamo
le nostre cose. La ricostruzione fenomenologica
parte dagli oggetti trascendenti e giunge
ai percepiti immanenti nella coscienza stessa:
dai "cogitata" al "cogitatio" (puro percepire).
Come per Kant, anche in Husserl c'è una
\textbf{"materia" del conoscere}, che però, a differenza di
Kant, non ha nulla di extrasoggettivo, ma \textbf{e' quell'elemento della percezione non più ulteriormente
scomponibile e non riducibile, e che è del
tutto immanente alla vita psichica. E'
chiamato il "dato iletico" (materiale), per
differenziarlo da quel dato formale, che poi diventa l'oggetto.
Husserl chiama il processo a ritroso, dall'oggetto alla materia percepita, riduzione,
in quanto si tratta di "ri-condurre" la
nostra esperienza e la complessità artificiale
degli oggetti alla sua struttura più intima
e vera , alla semplicità dell'"Erlebnis"
(sentimento vissuto).
Solo l'orizzonte dei vissuti è indubitabile e
certo: la riduzione fenomenologica è
perciò scienza rigorosa.}


L'esistenza della cosa è priva di reale fondamento:
è un essere, che la coscienza pone nelle sue
esperienze, che è visibile e determinabile
come ciò che permane identico nella molteplicità
delle apparizioni, ma all'infuori di questo è
un nulla.
\textbf{Ciò che noi effettivamente percepiamo sono
solo le molteplici apparizioni, quelle che
Husserl chiama "adombramenti", le varie
sfumature chiaroscure della cosa, ed ad essa
perveniamo collegando tutte queste sensazioni
in un che di unico, ritenendoli modificazioni di un sostrato identico che chiamano
"cosa".
L'identità dell'oggetto nella pluralità delle
percezioni è un prodotto della coscienza sintetica
"che riannoda la nuova percezione con il ricordo".}


\textbf{L'originario non è l'oggetto ma i suoi modi
di apparizione}, sempre diversi; anche il collegamento di più sensi è un momento della nostra vita
psichica.
\textbf{La costruzione della cosa è un processo di
progressiva formazione di senso, in cui
attribuiamo i pur differenti modi di
apparizione ad un unico oggetto, che poi
riteniamo causa dei nostri percepiti ed
indipendente da essi.}
L'operazione con cui dal modo di apparizione
perveniamo alla cosa non è estrinseca o arbitraria
rispetto al singolo modo di apparizione, ma ha il
suo fondamento nella struttura in cui il singolo
modo di apparizione è inserito; lo "sfondo" da cui astraiamo la cosa
non è l'oggetto ma qualcosa di più indefinito e aperto, ed in esso è radicata quell'operazione di "arricchimento di senso" che sta
alla base della nostra esperienza di oggetti.
\textbf{L'idea della cosa non avviene quindi secondo una
ricostruzione categoriale come in Kant,
ma a partire da una tendenza immanente
alle stesse rappresentazioni, che deriva dal
fatto che ognuna di esse non è mai
conclusa, ma rinvia necessariamente al
di là di sé. La costruzione della cosa è
in definitiva un passaggio dalla potenza
all'atto, un'operazione in cui attualizziamo
le potenzialità immanenti a ogni singola
rappresentazione.
Ogni singola manifestazione attualmente presente
rinvia potenzialmente a tutte le altre rappresentazioni della cosa ed ad uno "sfondo" più
complesso in cui il singolo modo di apparizione è
un momento astratto.} Al singolo modo di
apparire perveniamo solo "estraendolo" dallo sfondo
complesso in cui originariamente si trova,
sfondo che non appare mai in atto, ma sempre
presente in potenza.

In conclusione, \textbf{apodittica (indubitabile) non è la cosa ma solo
il suo modo di apparizione, non l'oggetto
ma il fenomeno, ciò che rimane immanente
alla coscienza}. L'esistenza del mondo e
di noi stessi è messa in discussione dalla
fenomenologia, che si limita a descrivere ciò
che realmente appare.

Alla base dell'atteggiamento fenomenologico
sta \textbf{una radicale "epoché"}, una sospensione del
giudizio sulla realtà e natura del mondo.
Tuttavia non viene negato ciò che vediamo, ma
tutta la nostra ricostruzioni spazio-temporale:
\textbf{il mondo e la natura sono "messi fra parentesi"},
cioè permane come mondo di fenomeni, ovvero
di apparizioni immanenti alla coscienza.
Per Kant dietro al fenomeno c'è il "noumeno", la cosa
in sé, di cui il fenomeno è la manifestazione
all'interno del soggetto; per Husserl il fenomeno è la natura ultima delle cose, oltre il
quale non si da nessuna essenza noumenica.
\textbf{Il fenomeno husserliano è il puro darsi originario
di cui la cosa è una costruzione successiva.}

\textbf{In Kant il processo di costruzione della cosa
nella nostra mente è non solo necessario, ma
guidato da una logica trascendentale, per cui
è garantito nelle sue pretese di verità. Per
Husserl questo processo è logicamente
arbitrario e dunque, anche se psicologicamente
giustificabile e spiegabile, falso nella misura
in cui pretende di valere come vera forma
del mondo.}

La fenomenologia non è la descrizione dell'esperienza in quanto costituita dal soggetto
ma \textbf{la descrizione dell'esperienza originaria;
è la scienza del "modo" in cui ci appaiono
oggetti e corpi. E' una scienza non del "che"
(che cosa?) ma del "come" (come appaiono
quelle che chiamano cose?)}, contrapponendosi al
positivismo che pretende di descrivere i fatti.
Il fenomeno non è un fatto, ma un
modo di darsi e presentarsi del reale, e
per coglierlo occorre un atteggiamento
"riflessivo" e indiretto, contrapposto a
quello ingenuo-naturale diretto.

\subsection{La fenomenologia come scienza eidetica}

Rispetto a Hume, per Husserl \textbf{alla scepsi deve
seguire la scienza}: in Hume la ragione viene
presupposta, e così la psiche e un corpo che la
genera, contraddicendo quanto aveva sostenuto
riguardo la dubitabilità del mondo materiale.
Il fenomenismo humeano si riduce essenzialmente
a uno psicologismo, ovvero un'analisi delle
rappresentazioni soggettive (impressioni e idee),
con la descrizione degli stati di coscienza,
ovvero dei modi nei quali il nostro apparato
conoscitivo raccoglie i dati della percezione.
Ma la fenomenologia non è scienza di dati
di fatto (ancorché immanenti alla coscienza),
bensì scienza di essenze, o scienza "eidetica".
Per essa \textbf{è impossibile descrivere in maniera obiettiva
un fenomeno}: la psicologia vi riesce solo in quanto
lo concettualizza, trasformando un modo di
apparire in un dato di fatto della psiche.

"Ogni Erlebnis (vissuto) rinvia ad altri "Erlebnisse"
e così via in maniera infinita (infatti lo "sfondo" è
sempre in potenza); perciò \textbf{ogni opera di
astrazione è una falsificazione di ciò che
realmente si mostra.}

Nonostante ciò secondo Husserl è possibile
una descrizione rigorosa delle essenze cui
i fenomeni si riconducono: infatti \textbf{i fenomeni
si danno all'interno di "modalità tipiche"
al di fuori delle quali non è possibile alcuna
esperienza}. Ad esempio non si può dare un
colore senza una superficie, o una luminosità senza colore, eccetera. \textbf{C'è una "struttura
essenziale dell'esperienza" che si presenta allo
sguardo fenomenologico nella sua innegabile
evidenza: a questo si rivolge la fenomenologia, evitando così la deriva scettica e la
rincorsa senza fine al fondo dei fenomeni.}

\textbf{Il fondo ultimo dei fenomeni si raggiunge
grazie ad una seconda riduzione, successiva a
quella fenomenica, la "riduzione eidetica",
la quale consente di ricondurre i fenomeni
ad un terreno ancora più originario,
costituito dai loro modi universali di
manifestarsi, cioè al loro "eidos".}
L'eidos fenomenologico ha natura pre-concettuale,
quindi non può essere astratto attraverso un processo
logico e di generalizzazione. In quanto struttura
essenziale comune ai vari modi di apparizione dei
fenomeni, \textbf{vi si accede grazie all'intuizione, cioè
all'indubitabile immediatezza del vedere
fenomenologico} ("Wesensschan", intuizione o
visione dell'essenza).

In quanto strutture essenziali del modo di darsi
dell'esperienza, le essenze possono riferirsi
sia a fenomeni psichici (gli stati o atti dell'io)
sia a quelli fisici, e così accanto ai modi di
presentarsi dei corpi o dei suoni, si daranno quelle
delle percezioni, dei ricordi e dei giudizi.
Ma il punto fondamentale è che in queste
essenze sono depositate le leggi fondamentali
della nostra esperienza, leggi che non
provengono kantianamente dal soggetto ma sono
immanenti all'esperienza stessa. \textbf{Le eidos
svolgono perciò la funzione di "apriori", di
condizioni di possibilità dell' esperienza;
Husserl lo definisce" apriori materiale"}, perché
immanente alla "materia" del mondo, \textbf{differenziandolo dall'"apriori" formale Kantiano.}
\textbf{Questa legge essenziale che regola la nostra
esperienza è indipendente dal soggetto }o dall'esigenza di avere un'esperienza ordinata;
\textbf{è una "legalità oggettiva" appartenente alla
natura dell'esperienza}. Grazie a questa
legalità la fenomenologia realizza il suo
programma di una descrizione rigorosa
del terreno originario dell'esperienza.

