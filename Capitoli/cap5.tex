La filosofia dopo Hegel è stata caratterizzata
da una messa in discussione della vecchia concezio­ne del trascendentale, più che da un esplicita
tematizzazione dei nuovi orizzonti che si erano
formati. Il tema esplicito della filosofia
fino a Weber era stato la presa d'atto dell'
impossibilità di stabilire l'orizzonte originario
all'interno del quale fondare il sapere: non si
accetta più un piano trascendentale ultimo.

Con poche eccezioni, la filosofia del '900
mantiene il proprio piano filosofico al di sotto
del piano trascendentale, senza perciò poterlo
fondare e tematizzare.

Nel secolo passato abbiamo assistito all'unione
delle molteplici istanze nate dalla dissoluzione
del paradigma soggettivistico, ad un nuovo
territorio comune: il linguaggio, la dimensione
in cui il pensiero si esprime, per cui consente
un rapporto con il sapere filosofico meno pregiudicato in partenza rispetto alle strutture tematizzate dalla filosofia post-hegeliana.

\chapter{Il pensiero fenomenologico-ermeneutico}
\bigskip
\bigskip
\section{Gli inizi dell'ermeneutica}

Nel periodo romantico con Schleiermacher,
l'ermeneutica da tecnica dell'interpretazione diventa  una disciplina filosofica: l'interprete deve
trasferirsi interiormente nella soggettività dell'autore e riprodurlo in se stesso, soprattutto
attraverso il sentimento.

Con Dilthey l'ermeneutica finisce per coincidere
con l'essenza stessa della filosofia, e attraverso
il sapere ermeneutico apprendiamo la totalità delle
manifestazioni spirituali (cultura, scienze sociali, arte, \dots).
In Dilthey si assiste ad un'evoluzione da un'ermeneutica introspettiva (basata sull'esperienza
vissuta, Erlebnis) ad una focalizzata sull'espressione oggettiva e storica della vita.
Allargandosi alla totalità delle manifestazioni
dello spirito, l'oggetto specifico dell'ermeneutica
diventa il mondo storico.
Così si qualifica non solo l'oggetto dell'interpretazione,
ma anche il soggetto interpretante, che è determinato dalla vita storica (storicità): la trascendentalità non è più indipendente dalla coscienza
storica.

\textbf{Nell'interpretazione vi è "comunanza", perché "colui che
indaga la storia è il medesimo che fa la storia".}
Diverso è lo studio della natura, data l'assenza
di comunanza, che avviene secondo rapporti causali.
Il comprendere ermeneutico è un ritrovamento della soggettività che ha prodotto l'oggettività
esteriore; \textbf{"il soggetto del sapere è identico con
il suo oggetto".
Nella storia i soggetti manifestano loro stessi in
espressioni oggettive che vengono consegnate alla
comprensione degli altri. Tutte queste espressioni
sono il medesimo spirito che si manifesta, ciò
che accomuna noi tutti, e in esso rivive la nozione
hegeliana di spirito oggettivo, anche se per
Dilthey l'essenza di esso è la vita, cioè una
realtà che non ha la struttura trasparente e logica
dello spirito hegeliano} (che poteva risalire allo spirito
assoluto, cioè alla completa autotrasparenza
di sé e all'assoluta identità del soggetto con
l'oggetto).
Il circolo ermeneutico per Dilthey è incompleto,
dato il carattere vitale del processo stesso, cioè
lo spirito ha caratteristiche irrazionali, per cui ogni
interpretazione è solo relativa.
Il concetto di comunanza dell'ermeneutica di Dilthey
rivela che il vero soggetto ermeneutico è rappresentato
dalla connessione vitale che unisce soggetto e oggetto.
\textbf{Il soggetto della storia dello spirito è
la vita e le connessioni operanti tra gli individui
storici.} Queste relazioni intersoggettive non sono
relazioni esterne e dipendenti dagli individui, ma
hanno una loro dipendenza e oggettività,  risultando
come determinanti gli individui stessi. Sono perciò
queste relazioni il vero soggetto dell'interpretare
e della vita storica. Nell'ultimo Dilthey \textbf{si
passa dal soggetto inteso come orizzonte originario,
all'operare intersoggettivo degli individui
come soggetto autonomo} (non più prodotto degli
individui): in ciò non solo rivive lo spirito
oggettivo hegeliano e la tesi del primato
dei rapporti intersoggettivi rispetto all'autocoscienza, ma viene anticipata
l'originarietà dell'"altro" dal soggetto
e la posizione di una differenza irriducibile
ad esso, quale sarà tematizzata dai successivi
autori.

\section{La fenomenologia husserliana}

Edmund Husserl (Germania 1859-1938) definisce
l'impianto fondamentale della fenomenologia:
questa è la scienza dei fenomeni ed ha
caratterizzazione non empirica ma trascendentale.

\subsection{Dagli oggetti ai fenomeni}


Il carattere interpretativo introdotto da Nietzsche,
e tematizzato da Dilthey in relazione
alla conoscenza storico-spirituale, viene
ribadito nella fenomenologia husserliana e
esteso a tutta l'esperienza.
L'atteggiamento pratico-valutativo per Husserl
ha a che fare con l'apparato conoscitivo in
generale, così che l'esperienza stessa non è
una successione di dati di fatto, bensì una
costituzione di propri oggetti.
\textbf{Il mondo non è evidente come sembra: è invece
l'esito di moltissime integrazioni di significato
rispetto a ciò che effettivamente appare ai
nostri sensi.}

 Il compito della fenomenologia è
ricostruire, partendo da ciò che si presenta come
già costituito, questo terreno di evidenze
primarie, quell'esperienza primaria alla base
del mondo di cose e oggetti che noi abbiamo
costruito, un originario che non ha ne natura
corale ne oggettiva.
Secondo Husserl l'originario non ha le caratteristiche
della cosa in sé (materia extra soggettiva che
affetta la nostra sensibilità), ma è qualcosa
di immanente alla nostra percezione: \textbf{l'originario
è l'insieme delle percezioni elementari, ovvero quei
"vissuti" che non hanno ancora forma oggettiva e
trascendente} (trascendente per Husserl è tutto ciò che non riguarda il dato, ogni astrazione compiuta dalla coscienza).
Come in Dilthey, ritorna il concetto di "Erlebnis",
la struttura semplice del percepire non ulteriormente scomponibile, a partire dal quale costruiamo
le nostre cose. La ricostruzione fenomenologica
parte dagli oggetti trascendenti e giunge
ai percepiti immanenti nella coscienza stessa:
dai "cogitata" al "cogitatio" (puro percepire).
Come per Kant, anche in Husserl c'è una
\textbf{"materia" del conoscere}, che però, a differenza di
Kant, non ha nulla di extrasoggettivo, ma \textbf{e' quell'elemento della percezione non più ulteriormente
scomponibile e non riducibile, e che è del
tutto immanente alla vita psichica. E'
chiamato il "dato iletico" (materiale), per
differenziarlo da quel dato formale che poi diventa l'oggetto.
Husserl chiama il processo a ritroso, dall'oggetto alla materia percepita, riduzione,
in quanto si tratta di "ri-condurre" la
nostra esperienza e la complessità artificiale
degli oggetti alla sua struttura più intima
e vera , alla semplicità dell'"Erlebnis"
(sentimento vissuto).
Solo l'orizzonte dei vissuti è indubitabile e
certo: la riduzione fenomenologica è
perciò scienza rigorosa.}


L'esistenza della cosa è priva di reale fondamento:
è un essere, che la coscienza pone nelle sue
esperienze, che è visibile e determinabile
come ciò che permane identico nella molteplicità
delle apparizioni, ma all'infuori di questo è
un nulla.
\textbf{Ciò che noi effettivamente percepiamo sono
solo le molteplici apparizioni, quelle che
Husserl chiama "adombramenti", le varie
sfumature chiaroscure della cosa, ed ad essa
perveniamo collegando tutte queste sensazioni
in un che di unico, ritenendoli modificazioni di un sostrato identico che chiamano
"cosa".
L'identità dell'oggetto nella pluralità delle
percezioni è un prodotto della coscienza sintetica
"che riannoda la nuova percezione con il ricordo".}


\textbf{L'originario non è l'oggetto ma i suoi modi
di apparizione}, sempre diversi; anche il collegamento di più sensi è un momento della nostra vita
psichica.
\textbf{La costruzione della cosa è un processo di
progressiva formazione di senso, in cui
attribuiamo i pur differenti modi di
apparizione ad un unico oggetto, che poi
riteniamo causa dei nostri percepiti ed
indipendente da essi.}
L'operazione con cui dal modo di apparizione
perveniamo alla cosa non è estrinseca o arbitraria
rispetto al singolo modo di apparizione, ma ha il
suo fondamento nella struttura in cui il singolo
modo di apparizione è inserito; lo "sfondo" da cui astraiamo la cosa
non è l'oggetto ma qualcosa di più indefinito e aperto, ed in esso è radicata quell'operazione di "arricchimento di senso" che sta
alla base della nostra esperienza di oggetti.
\textbf{L'idea della cosa non avviene quindi secondo una
ricostruzione categoriale come in Kant,
ma a partire da una tendenza immanente
alle stesse rappresentazioni, che deriva dal
fatto che ognuna di esse non è mai
conclusa, ma rinvia necessariamente al
di là di sé. La costruzione della cosa è
in definitiva un passaggio dalla potenza
all'atto, un'operazione in cui attualizziamo
le potenzialità immanenti a ogni singola
rappresentazione.
Ogni singola manifestazione attualmente presente
rinvia potenzialmente a tutte le altre rappresentazioni della cosa ed ad uno "sfondo" più
complesso in cui il singolo modo di apparizione è
un momento astratto.} Al singolo modo di
apparire perveniamo solo "estraendolo" dallo sfondo
complesso in cui originariamente si trova,
sfondo che non appare mai in atto, ma sempre
presente in potenza.

In conclusione, \textbf{apodittica (indubitabile) non è la cosa ma solo
il suo modo di apparizione, non l'oggetto
ma il fenomeno, ciò che rimane immanente
alla coscienza}. L'esistenza del mondo e
di noi stessi è messa in discussione dalla
fenomenologia, che si limita a descrivere ciò
che realmente appare.

Alla base dell'atteggiamento fenomenologico
sta \textbf{una radicale "epoché"}, una sospensione del
giudizio sulla realtà e natura del mondo.
Tuttavia non viene negato ciò che vediamo, ma
tutta la nostra ricostruzione spazio-temporale:
\textbf{il mondo e la natura sono "messi fra parentesi"},
cioè permane come mondo di fenomeni, ovvero
di apparizioni immanenti alla coscienza.
Per Kant dietro al fenomeno c'è il "noumeno", la cosa
in sé, di cui il fenomeno è la manifestazione
all'interno del soggetto; per Husserl il fenomeno è la natura ultima delle cose, oltre il
quale non si da nessuna essenza noumenica.
\textbf{Il fenomeno husserliano è il puro darsi originario
di cui la cosa è una costruzione successiva.}

\textbf{In Kant il processo di costruzione della cosa
nella nostra mente è non solo necessario, ma
guidato da una logica trascendentale, per cui
è garantito nelle sue pretese di verità. Per
Husserl questo processo è logicamente
arbitrario e dunque, anche se psicologicamente
giustificabile e spiegabile, falso nella misura
in cui pretende di valere come vera forma
del mondo.}

La fenomenologia non è la descrizione dell'esperienza in quanto costituita dal soggetto
ma \textbf{la descrizione dell'esperienza originaria;
è la scienza del "modo" in cui ci appaiono
oggetti e corpi. E' una scienza non del "che"
(che cosa?) ma del "come" (come appaiono
quelle che chiamano cose?)}, contrapponendosi al
positivismo che pretende di descrivere i fatti.
Il fenomeno non è un fatto, ma un
modo di darsi e presentarsi del reale, e
per coglierlo occorre un atteggiamento
"riflessivo" e indiretto, contrapposto a
quello ingenuo-naturale diretto.

\subsection{La fenomenologia come scienza eidetica}

Rispetto a Hume, per Husserl \textbf{alla scepsi deve
seguire la scienza}: in Hume la ragione viene
presupposta, e così la psiche e un corpo che la
genera, contraddicendo quanto aveva sostenuto
riguardo la dubitabilità del mondo materiale.
Il fenomenismo humeano si riduce essenzialmente
a uno psicologismo, ovvero un'analisi delle
rappresentazioni soggettive (impressioni e idee),
con la descrizione degli stati di coscienza,
ovvero dei modi nei quali il nostro apparato
conoscitivo raccoglie i dati della percezione.
Ma la fenomenologia non è scienza di dati
di fatto (ancorché immanenti alla coscienza),
bensì scienza di essenze, o scienza "eidetica".
Per essa \textbf{è impossibile descrivere in maniera obiettiva
un fenomeno}: la psicologia vi riesce solo in quanto
lo concettualizza, trasformando un modo di
apparire in un dato di fatto della psiche.

"Ogni Erlebnis (vissuto) rinvia ad altri "Erlebnisse"
e così via in maniera infinita (infatti lo "sfondo" è
sempre in potenza); perciò \textbf{ogni opera di
astrazione è una falsificazione di ciò che
realmente si mostra.}

Nonostante ciò secondo Husserl è possibile
una descrizione rigorosa delle essenze cui
i fenomeni si riconducono: infatti \textbf{i fenomeni
si danno all'interno di "modalità tipiche"
al di fuori delle quali non è possibile alcuna
esperienza}. Ad esempio non si può dare un
colore senza una superficie, o una luminosità senza colore, eccetera. \textbf{C'è una "struttura
essenziale dell'esperienza" che si presenta allo
sguardo fenomenologico nella sua innegabile
evidenza: a questo si rivolge la fenomenologia, evitando così la deriva scettica e la
rincorsa senza fine al fondo dei fenomeni.}

\textbf{Il fondo ultimo dei fenomeni si raggiunge
grazie ad una seconda riduzione, successiva a
quella fenomenica, la "riduzione eidetica",
la quale consente di ricondurre i fenomeni
ad un terreno ancora più originario,
costituito dai loro modi universali di
manifestarsi, cioè al loro "eidos".}
L'eidos fenomenologico ha natura pre-concettuale,
quindi non può essere astratto attraverso un processo
logico e di generalizzazione. In quanto struttura
essenziale comune ai vari modi di apparizione dei
fenomeni, \textbf{vi si accede grazie all'intuizione, cioè
all'indubitabile immediatezza del vedere
fenomenologico} ("Wesensschan", intuizione o
visione dell'essenza).

In quanto strutture essenziali del modo di darsi
dell'esperienza, le essenze possono riferirsi
sia a fenomeni psichici (gli stati o atti dell'io)
sia a quelli fisici, e così accanto ai modi di
presentarsi dei corpi o dei suoni, si daranno quelle
delle percezioni, dei ricordi e dei giudizi.
Ma il punto fondamentale è che in queste
essenze sono depositate le leggi fondamentali
della nostra esperienza, leggi che non
provengono kantianamente dal soggetto ma sono
immanenti all'esperienza stessa. \textbf{Le eidos
svolgono perciò la funzione di "apriori", di
condizioni di possibilità dell' esperienza;
Husserl lo definisce" apriori materiale"}, perché
immanente alla "materia" del mondo, \textbf{differenziandolo dall'"apriori" formale Kantiano.}
\textbf{Questa legge essenziale che regola la nostra
esperienza è indipendente dal soggetto }o dall'esigenza di avere un'esperienza ordinata;
\textbf{è una "legalità oggettiva" appartenente alla
natura dell'esperienza}. Grazie a questa
legalità la fenomenologia realizza il suo
programma di una descrizione rigorosa
del terreno originario dell'esperienza.

Individuata questa legalità fenomenologica, che
riesce a descrivere il terreno originario dell'esperienza,
rimane da osservare se un tale accesso è possibile:
un accesso immediato all'esperienza, privo di mediazioni concettuali; per dirla con Kant, capire se è possibile un'intuizione
senza concetti.

\subsection{Il soggetto trascendentale}

\textbf{La riduzione eidetica non può fermarsi finché non abbia
saputo riconoscere il vero universale di tutti i modi di
darsi dei fenomeni: l'"Io trascendentale".
La "riduzione trascendentale" riconduce le essenze
all'Io: "l'intero mondo oggettivo" è per me; quindi l'Io inteso da Husserl altro non è che il puro apparire dei fenomeni, il puro apparire del mondo.}
Questa trasformazione del mondo, ridotto al
soggetto, comporta alla fine anche una
trasformazione del soggetto stesso.

Definendo l'Io trascendentale, Husserl intende
differenziarlo da quello che egli chiama
"Io psicologico", ovvero la nostra identità
individuale: essa è esito di una ricostruzione, e non
può costituire quell'orizzonte originario cui tutto
viene ricondotto. Husserl critica la concezione psicologica
humeana del soggetto: l'identità è una finzione
psicologica, esito di una somma arbitraria di una
serie di rappresentazioni.
"I fenomeni della fenomenologia trascendentale vengono
caratterizzati come irreali", non hanno realtà in
loro stessi ma dipendono dall'Io, cioè dall'unica
vera realtà.
In sostanza, per non ricadere nello psicologismo guardando alle rappresentazioni psichiche come a elementi della
natura umana, bisogna ricondurre i fenomeni a
quell'unità non psicologica, non umana,
non mondana che è l'Io trascendentale.
\textbf{L'Io  è posto come condizione trascendentale
dell'esperienza: Kant colloca questa scoperta
all'interno del mondo (delle cose in sé); la
riduzione trascendentale dissolve ogni datità
delle cose e svela la fenomenologia come una
forma di idealismo}.

Husserl però prende le distanze dall'idealismo classico tedesco:

\begin{itemize}


\item \textbf{la fenomenologia non deduce nessuna totalità,}
bensì procede in modo inverso, partendo dell'esperienza e riconducendo i fenomeni nella loro
infinita varietà ai modi tipici di manifestarsi
e quindi da ultimo all'Io. \textbf{Il pensare è
visto come forma dei fenomeni, non separabile
da essi e di cui è la manifestazione}, e non
può perciò generare nessun automovimento
a priori da cui dedurre il mondo.

\item  Rispetto a Hegel, \textbf{il mondo dei dati empirici
è "messo tra parentesi": non viene messa in
discussione la natura sensibile ma solo la tesi
che lo interpreta come esistente all'infuori
di noi}. Al contrario in Hegel il fenomeno
era ricondotto a strutture logiche concettuali
in cui esso perde i suoi tratti empirici.

\item  \textbf{Resta centrale nella fenomenologia la
natura "intenzionale" della coscienza, ovvero
il suo costitutivo riferirsi sempre a qualcosa,
il suo intendere sempre qualcosa. I "vissuti",
"in quanto sono coscienza di qualcosa, si
dicono intenzionalmente riferiti a questo
qualche cosa". Non si da coscienza senza
che essa sia al tempo stesso coscienza di
qualcosa}: ciò non significa che quel qual cosa
è una realtà sussistente e indipendente dalla
coscienza, ma che \textbf{quel qualcosa è condizione del
darsi della coscienza in quel momento. Poiché la
coscienza non potrebbe esistere senza ciò che
essa di volta in volta intenziona, il rapporto
tra soggetto e mondo non è pensato nei termini
idealistici dell'identità, ma in quelli della
correlazione: non si può pensare la coscienza
senza il mondo fenomenico intenzionato da
essa.}

\end{itemize}

\subsection{L'intersoggettività}

Nelle "Meditazioni cartesiane" Husserl descrive come
\textbf{il metodo fenomenologico è in grado di accettare
altri soggetti}: non in senso mondano (spaziale e
naturale), ma la semplice esistenza di un'altra
esperienza, oltre a quella vissuta dal proprio io.
Husserl la definisce \textbf{"esperienza dell'estraneità",
vale a dire la possibilità di poter distinguere all'interno
della mia esperienza fra ciò che è mio proprio e ciò
che mi è estraneo.}
Egli parte dalla possibilità di poter circoscrivere
all'interno dei fenomeni di cui si ha esperienza quelli
che appartengano alla sfera del mio proprio, cioè
quelli che rinviano alla mia soggettività e
caratterizzanti la mia esperienza interna:
di quest'ultima è parte essenziale quella che
Husserl chiama \textbf{"esperienza del proprio corpo"}:

\begin{itemize}
	\item \textbf{Korper, corpo fisico-spaziale, di cui ho
	esperienza esterna.}
	\item \textbf{Leib, corpo organico, con cui si ha sensazione
	interna del proprio corpo, come veicolo di
	conoscenze percettive e perciò più "soggetto"
	che "oggetto" dell'esperienza.}
\end{itemize}

In altre parole \textbf{il mio corpo può essere sia oggetto
di osservazione (Korper) che soggetto percepiente
(Leib), cioè le due parti sono reciprocamente connesse.
Ciò rende possibile distinguere un Korper estraneo
come non connesso al proprio Leib, ma ad
un altro, sulla base della somiglianza del
corpo estraneo con il proprio. Ciò rende possibile
l'esperienza di un altro Leib, cioè l'immedesimazione
da parte mia nell'interiorità altrui.}

\textbf{L'altro non è percepito in modo evidente, come
fenomeno originario (apodittico), ma attraverso
una percezione indiretta ("appercezione") che
avviene sulla base di una prima percezione, che
rende presente ciò che non lo è (Leib dell'altro).
L'altro soggetto è "costituito"}, non è semplice
presenza attuale, bensì avviene mediante
trasferimento di senso;\textbf{ in altre parole l'altro soggetto viene interpretato.
Da qui si dischiude l'intera dimensione dell'intersoggettività:} non vi è alcuna differenza
ontologica fra la mia natura e quella altrui;
"essi sono per sé come io sono per me".

\textbf{Se tutti condividiamo la stessa natura, assieme
siamo "soggetti trascendentali"; non sono
un soggetto solitario ma parte di una
comunità, detta "l'intersoggettività trascendentale",
a partire dalla quale si costituisce "un unico
identico mondo".}

\textbf{Il mondo oggettivo è perciò il prodotto della
convergenza di differenti intenzionalità
intersoggettive (di intersoggettività trascendentale),
e il singolo soggetto può avere esperienza in
quanto si trova in comunità con sui simili.
Husserl, a partire dall'assoluta immanenza degli atti
intenzionali del soggetto, costituisce un mondo
intersoggettivo trascendente rispetto a quegli atti.}
Un ruolo centrale è svolto dal processo di
immedesimazione, per cui pensiamo che un
altro Korper possa avere la nostra stessa
esperienza interna. Con questa operazione, la
fenomenologia travalica ciò che è evidente
e diventa capace di aprirsi al mondo dei
soggetti reali, trascendenti rispetto ai
fenomeni, rendendo accessibile la complessa
realtà del mondo umano e delle sue relazioni.

\textbf{L'attestazione fenomenologica della relazione di
somiglianza non è coerente con il rigore
apodittico a cui la fenomenologia dovrebbe
attenersi}: si perde in rigore attribuendo realtà
ad un soggetto perché un corpo assomiglia
al proprio. Inoltre una relazione di somiglianza
non autorizza quel trasferimento di senso che
finiva per attribuire a un corpo estraneo
la soggettività che si accompagna al mio corpo
proprio. \textbf{Attenendosi al puro vedere non potrò mai
vedere un Leib nell'altro, ma sempre e solo un Korper.}

\subsection{La crisi delle scienze europee (1935-37)}

L'ultima grande opera di Husserl ci giunge incompleta.
\textbf{La crisi delle scienze consiste nella perdita del loro
significato per la vita, perché per esse contano solo
i fatti; perciò il prezzo di questa attenzione esclusiva
all'obiettivabile è la dimenticanza di quel soggetto
che è l'origine prima del processo di oggettivazione.
La critica dell'"obiettivismo scientifico" è la riduzione
del mondo a universo di oggetti ritenuti indipendenti dal soggetto osservante, e considerati sotto
l'aspetto meramente quantitativo} (come già osservato da Weber con la sua teoria di razionalizzazione del mondo, intesa come quantizzazione misurabile e calcolabile, che porta al disincanto da parte dell'umanità,  ad una mancanza di significato nello svolgersi della vita e ad una diminuita libertà personale, essendo gli individui guidati da questa razionalità, intesa come burocrazia ed economia).

Per Husserl \textbf{dobbiamo compiere un'operazione fenomenologica,
ricostruendo l'esperienza qualitativa alla base del
processo di scientificizzazione e che da esso è stata
rimossa.} Questa presa di distanza è detta da Husserl:
\textbf{"l'epoché della scienza obiettiva".
Questo scetticismo non conduce però dagli oggetti
ai fenomeni ma ad una sorta di terreno intermedio,
detto "Lebenswelt", il mondo della vita o mondo
vitale: un vero e proprio terreno vitale di costituzione,
capace di costruire quel mondo oggettivo che caratterizza
l'immagine scientifica del mondo.
Il mondo vitale è quello osservato con occhi
prescientifici, fatto di qualità e non di quantità,
dove tutte le cose sono "soggetto-relativo",
ovvero ancora in relazione con la nostra interiorità.}

La Lebenswelt ha dunque alcune caratteristiche che
la prima fenomenologia husserliana individua nelle percezioni
vitali primarie (i vissuti, gli Erlebnisse), e in primis
la correlazione con la nostra soggettività. Tuttavia
\textbf{il mondo vitale è un mondo di cose collocate
nello spazio-tempo, dove queste cose stanno in
rapporto con noi, osservate a partire dalla nostra
prospettiva e non in maniera asettica (come nelle
scienze).
Per la nostra esperienza quotidiana di vita è
"sicuro" ciò che per il fenomenologo non lo è, vale
a dire l'esistenza di oggetti e di un mondo
al di fuori di noi. L'obiettivismo scientifico
si sviluppa a partire dalla convinzione di fondo
che attraversa il mondo vitale, la fede (radicalizzata) nello
spazio-tempo} (sempre come affermato da Weber, crediamo nella razionalità,  perché ci serve da un punto di vista pratico, senza tuttavia dare nessuna dimostrazione teoretica della razionalità del mondo).

 \textbf{La Lebenswelt non è solo un mondo
di cose, bensì una prospettiva originaria a partire
dalla quale il soggetto guarda il mondo: essa
ha dunque la posizione di un orizzonte trascendentale,
sulla base di cui si costituisce la nostra esperienza.
Questo terreno primario ha natura pratico-vitale,
è descrizione di un mondo pre-categoriale al cui
centro è la dimensione primaria della vita umana}
(recupero di Dilthey e della sua visione della vita
come orizzonte intrascindibile, condizionante il
modo in cui i soggetti si relazionano al mondo).
\textbf{Questo orizzonte trascendentale è a sua volta
costituito}, carico di integrazioni di senso, e,
per Husserl e il filosofo, \textbf{bisognoso di essere sottoposto ad
una riduzione trascendentale: il mondo vitale}
non va indagato con l'atteggiamento della vita
naturale mondana, ma \textbf{richiede "un'epoché universale
assolutamente peculiare", ponendoci al di sopra di
quella coscienza nella quale il mondo è "qui",}
l'universo di tutto ciò che è alla mano.

\textbf{L'indagine sull'orizzonte trascendentale della
vita riconduce alla soggettività trascendentale,
senza tuttavia richiudersi dentro il
paradigma soggettivistico della modernità, in
quanto anche il soggetto si è rivelato costituito
(messo fra parentesi dall'epoché fenomenologica).
Ciò che rimane dopo le riduzioni fenomenologiche è
solo il puro apparire dei fenomeni, e il trascendentale
ultimo è ciò che rende possibile il manifestarsi
fenomenico, ciò che Husserl chiama "Io
trascendentale". Quindi il soggetto che ha ridotto
tutto a sé (compreso se medesimo) è diventato
identico con l'apparire del mondo.}

\textbf{L'intersoggettività viene qui spiegata da Husserl con il
processo di "auto-estraniazione"}, un'operazione grazie
alla quale il soggetto trasferisce la propria soggettività
all'altro, all'estraneo. \textbf{L'attribuzione della
soggettività è analoga alla costituzione del sé
come soggetto}: percepiamo di noi stessi la nostra
attualità e la presenza, nella nostra attualità, del
ricordo di noi stessi nel passato, e ciò ci consente di collegare
la percezione del sé attuale con il ricordo del sé
passato, costituendo così un io permanente nel
tempo. \textbf{Il "tu" viene costituito in modo analogo
al "sé" temporale: invece che un trasferimento
dell'io nel tempo, si compie un trasferimento
dell'io nell'estraneo.}
Anche qui \textbf{il processo di "auto-estranazione" mostra
la debolezza fenomenologica della relazione di
"somiglianza"}. Inoltre, conferire permanenza all'io
non ha fondamento logico ma solo psicologico, dato che
un soggetto permanente nel tempo non potrà mai
apparire come attualmente evidente, così per il
processo di auto-estraniazione, dove \textbf{non c'è nessuna
evidenza che consenta al soggetto trascendentale
di attribuire la propria soggettività a un corpo
estraneo.}

Inoltre \textbf{emerge la tensione fra soggettività  che si è posta come trascendentale}, cioè identica all'intero orizzonte dell'apparire, \textbf{e una pluralità di soggetti ai quali è attribuita la medesima trascendentalità: se alla soggettività altrui va attribuito il medesimo statuto della soggettività originaria, allora il soggetto trascendentale cessa di essere l'unico orizzonte di senso, vedendo messa in discussione la sua pretesa di trascendentalità}. Abbassato a semplice punto di vista accanto ad altri, diventa qualcosa di molto simile al soggetto empirico-psicologico. \textbf{E' sempre la nozione di intersoggettività trascendentale a mostrarsi intrinsecamente contraddittoria.}

Le procedure fenomenologiche stabilite da Husserl rendano indimostrabile una comunità di soggetti, come abbiamo già avuto modo di osservare. Tuttavia \textbf{Husserl ha bisogno di altri soggetti, in quanto l'esperienza non è desoggettivizzata, ma ha natura egologica, cioè, per poter definire in termini di "io" l'orizzonte dell'esperienza, necessita di un "tu". L'intersoggettività, punto debole della filosofia di Husserl, è condizione necessaria per la stessa comprensione di un "io" trascendentale. Il vero punto debole della fenomenologia husserliana è l'insistenza sulla radicalità di un puro vedere intuitivo, privato di qualsiasi mediazione concettuale, linguistica o pratica, a produrre la "messa tra parentesi" del mondo, che rende la fenomenologia irriducibile a qualsiasi esito di "mondanizzazione".}

\section{Heidegger: Essere e tempo (1927)}

Martin Heidegger (1889-1976) pone la sua opera in
diretta continuità con quella di Husserl. Suo scopo
è manifestare le cose nella loro essenza, al di là di
tutte le indebite integrazioni di senso che si sovrappongono a ciò che esse sono in verità: in questo quadro
si può collocare il programma heideggeriano di intendere
la fenomenologia come ontologia. Infatti il fenomeno
nella sua originarietà è proprio l'essere, ovvero quella dimensione
che di solito rimane nascosta, e a cui è perciò necessario
lo sguardo fenomenologico per mostrarla. "Dietro i
fenomeni della fenomenologia non c'è essenzialmente
nient'altro", \textbf{l'essere che si propone di manifestare non è
qualcosa al di là dei fenomeni ma è il loro pieno manifestarsi.}

\subsection{Daisen come trascendentale}

Quanto detto non conduce Heidegger a una
riabilitazione della concezione realistico-sostanzialistica dell'essere. Ciò che si manifesta allo sguardo
del fenomenologo rimane nella correlazione
essenziale col soggetto, non è indipendente.
\textbf{In "Essere e tempo" (Sein und Zeit, 1927) il mondo
si costituisce all'interno di quell'ente particolare che
è l'essere umano, che tuttavia Heidegger non
definisce più soggetto, ma lo qualifica come
"\textit{Dasein}", ovvero l'Esserci , l'esser-qui, l'essere in
quanto si determina in un "dove".}

Qui sono tratte le
conseguenze dell'esito fenomenologico che aveva fatto
coincidere l'io con l'apparire del mondo.
\textbf{L'essere del mondo non è indipendente dal \textit{Dasein}: la
"mondità" (Weltlichkeit), cioè l'appartenenza al mondo
delle cose che ci circondano, non è un attributo che
differenzia le cose da noi, ma le fa nostre, in quanto
essa sono mondane per virtù nostra. La mondità è}
un nostro attributo, o, secondo la terminologia
heideggeriana, \textbf{un "esistenziale", cioè un carattere
costitutivo e inaggirabile del \textit{Dasein}.}


\textbf{L'analitica del mondo passa attraverso l'analisi del
\textit{Dasein}, così come in Husserl passa dall'Io
trascendentale}. Altra analogia con il maestro, il
\textit{Dasein} non può essere inteso come soggetto psicologico,
cioè non si contrappone al mondo come al suo oggetto,
ma è l'essere dell'oggetto, è l'apertura al mondo,
quell'orizzonte che non può essere una cosa ma
rende possibili le cose.

\subsection{La critica ad Husserl e all'antologia tradizionale}

\textbf{Heidegger imputa ad Husserl una concezione
ristretta della riduzione fenomenologica, la quale,
invece di manifestare l'essere nella sua pienezza,
si limita a manifestare solo una dimensione,
quella della presenza allo sguardo, dell'apparizione hic et nunc.}
Come annuncia il titolo dell'opera, Heidegger
si propone di dimostrare come \textbf{essere e tempo
non siano in rapporto estrinseco ma si richiamino
reciprocamente. L'essere è strutturalmente
temporale }e non aver compreso questa essenziale
implicazione ha costituito il limite fondamentale
della tradizione ontologica, che ha sempre
privilegiato il presente rispetto alle altre dimensioni
\footnote{infatti "ente" è participio presente del verbo essere, così
	come "essere" è l'infinito presente.}.

\textbf{La metafisica ha inteso l'essere come presenza
stabile}: solo ciò che è stabilmente presente può
essere definito ente; \textbf{è uno stare nel tempo, che
attraversa il tempo come presenza stabile.}
Da ciò la preminenza che fin dai greci ha avuto
l'atteggiamento teoretico-osservativo (contemplativo) rispetto alle
altre dimensioni dell'umano. Prima di Heidegger è sempre stata la teoria ciò che
da accesso alla dimensione dell'esser presente: vero è
ciò che appare ora al suo sguardo.


\textbf{La riduzione fenomenologica husserliana approda
all'eidos eliminando le integrazioni di senso
sovrapposte alla pura fenomenicità, ma contemporaneamente eliminando tutte quelle dimensioni che non
sono riducibili all'orizzonte di ciò che è presente
allo sguardo (che è presente "sottomano").}

Heidegger si propone di ripercorrere nuovamente
il cammino della riduzione fenomenologica,
per giungere alla vera dimensione primaria in cui
l'essere del mondo viene dato al \textit{Dasein}.
\textbf{Questo manifestarsi del mondo all'esserci ha
la caratteristica che
il suo costituire il mondo non è allontanarsi
da esso, ma l'istituire una relazione necessaria
con il mondo. Così come il mondo non può essere
senza \textit{Dasein}, ugualmente il \textit{Dasein} non può essere
senza il mondo.} Questo rapporto di appartenenza
non può essere espresso dalle categorie della metafisica
della presenza, piuttosto come \textbf{un rapporto di familiarità
reciproca. Il "qui" del \textit{Dasein} è uno "stare-presso",
l'avere familiarità con il mondo che esso stesso
costituisce.
Da ciò le altre determinazioni che Heidegger
attribuisce al \textit{Dasein} , ovvero "l'in-essere" e
"l'essere-nel-mondo".}

\textbf{L'esperienza originaria del mondo} non \textbf{è} l'esperienza
teoretico-osservativa, bensì \textbf{l'esperienza pratica del
"prendersi cura" del mondo (avere familiarità con il mondo).
Le cose del mondo si presentano sotto l'aspetto
pratico della loro utilizzabilità, e il mondo è
una totalità di cose utilizzabili.}
L'in sé dell'ente non è l'ente ridotto alla mera
essenzialità ontologica-presentificante, ma è il
contrario di ogni in sé, ovvero \textbf{l'essere-per, la
disponibilità pratica, la maneggevolezza, la relazione.}

In definitiva,\textbf{ la fenomenologia heideggeriana non deve
condurre alle essenze ideali ma all'universo dei
"pragmata"} ("ciò con cui si ha a che fare nel commercio
prendente cura"), dunque si tratta di ridurre
quegli stessi fenomeni (husserliani) ad un
orizzonte ancora più originario, ovvero al
\textbf{rapporto pratico-poietico verso le cose.}

\subsection{Trasformazione del trascendentale e \textit{Dasein} come "cura"}

Il \textit{Dasein}, ovvero quell'essere a cui il mondo stesso si
riconduce, non è un mero orizzonte trascendentale
inteso come il puro apparire degli eventi (come sostenuto da
Husserl), quindi ristretto alla dimensione
della presenza in un certo istante.
\textbf{Heidegger ritiene invece che la natura del \textit{Dasein} sia
espressa nel concetto di esistenza}: introdotto per caratterizzare il \textit{Dasein}, questo concetto
di esistenza non
indica l'attualità della cosa, la sua dimensione ontica
presente ("l'esser presente sottomano"); \textbf{essa significa
al contrario l'esser al di là della mera datità, poter
essere qualcosa di diverso da ciò che è presentemente.
\textit{Dasein} è un non ente, ma possibilità di essere e
progetto: ciò che esso è gli deriva dalla sua stessa
possibilità, ovvero la sua realtà gli appartiene,
in quanto non gli viene data dall'esterno ma è
il realizzarsi della sua possibilità.
Se l'esistenza va intesa come possibilità e apertura,
ciò determina la natura del \textit{Dasein} come pratica,
proprio perché non è una cosa ma poter essere,
esso è essenzialmente progetto e prassi.}
Il mondo non è dischiuso da un orizzonte costituito
dall'io puro teoretico ma da un orizzonte pratico-progettuale.

Il carattere di progetto della comprensione non
significa che esso coglie in modo tematico ciò
rispetto a cui progetta, cioè la possibilità, ovvero
\textbf{l'esserci non edifica il suo essere in base a un
piano. L'origine del suo progettare sfugge al \textit{Dasein},
perché esso è un "progetto gettato", è in balia
del mondo, gettato in un "da" (qui).
Il \textit{Dasein} è la condizione trascendentale del mondo,
ma al tempo stesso è consegnato a questo mondo
che lui stesso ha aperto: l'orizzonte che fa essere
il mondo non è dominato dal suo "esser qui", infatti
non conosce il donde e il suo dove. La sua
trascendentalità convive con la sua fatticità} (che non
è fattualità), cioè Heidegger apre a un nuovo
significato del trascendentale: \textbf{la condizione di
possibilità è a sua volta condizionata.}

\textbf{Il \textit{Dasein} è contemporaneamente progetto e affezione,
infinite possibilità  e finitezza dovuta al
trovarsi già in una certo orizzonte che genera una
"situazione emotiva", cioè la fatticità consiste
nel suo esser determinato da altro e l'esser
consegnato alla sua propria situazione.}
Questo intreccio di trascendentalità e fatticità,
\textbf{di apertura e dipendenza dal mondo}, viene inquadrato
nella celebre definizione heideggeriana dell'\textbf{"essere
dell'esserci come cura"}. Cura non è una
pratica mondana, ma \textbf{un'"apriorità esistenziale"
che dispone il \textit{Dasein} in un atteggiamento di
coinvolgimento emotivo con il suo ambiente.
Definire l'esistenza come cura o se preferiamo, come l'avere familiarità con il mondo, significa} che
non contempliamo distaccati il mondo, ma \textbf{che
abbiamo un rapporto coinvolgente con le cose, una
"pratica" con esse, una partecipazione interessata.
Qualsiasi sguardo sul mondo risulta intrecciato con
il "commercio" che intratteniamo con esso, ovvero abbiamo
una "visione ambientale" che non può prescindere
dal carattere di "roba utile" che ha tutto ciò
che noi incontriamo. La cura può essere intesa
come "prassi originaria" nel quale il \textit{Dasein} si
trova coinvolto e che costituisce il suo essere
specifico.}

\textbf{Sono tre i caratteri fondamentali della cura:}
\begin{enumerate}
	\item \textbf{la progettualità esistenziale}, cioè l'esistenza
	come poter essere, come essere" avanti a sé";
	\item \textbf{la gettatezza,} l'esser consegnato in un mondo,
	che ci assegna in un "già";
	\item \textbf{il prendervi cura delle cose del mondo}, cioè
	l'esser-presso gli utilizzabili.
\end{enumerate}

Queste tre modalità svelano la temporalità del
\textit{Dasein}, rispettivamente futuro, passato e presente.

\subsection{La tematizzazione dell'intersoggettività}

Nel mondo incontriamo sia l'insieme dei mezzi
sia gli altri \textit{Dasein}; perciò \textbf{il \textit{Dasein} esiste per
la rete di coinvolgimenti pratici con le cose e
per le relazioni con gli altri. Heidegger non fornisce una dimostrazione dell'esistenza degli altri soggetti}, perché il \textit{Dasein} isolato (da presupporre per poi dimostrare l'esistenza di altri individui), senza il mondo e senza gli altri, è impossibile, una pura astrazione basata sulla falsa presupposizione che l'essere sia esclusivamente presenza.

\textbf{L'intersoggettività è Con-esserci, un ente comunitario che vive in uno spazio esistenziale di pratiche e di cure e non in uno spazio esteso dove trovano collocazione altri altri soggetti; essi non sono oggetti osservabili ma esserci legati fra loro in una visione regolata dalle forme della cura}. Come ci si prende cura del mondo, così si ha cura degli altri: ciò non va inteso in senso etico, bensì la cura regola il rapporto con gli altri anche quando addirittura lottiamo contro essi.
In "Essere e tempo" scompare quell'intersoggettività teoretica, la cui conclusione terminava con inevitabili aporie, e si fa avanti \textbf{un'intersoggettività pratica, in cui noi siamo già da sempre insieme agli altri. Questa è la precondizione perché gli altri possano essere compresi da noi: se l'altro non fosse già in me rimarrebbe per sempre un'alterità irraggiungibile.}



