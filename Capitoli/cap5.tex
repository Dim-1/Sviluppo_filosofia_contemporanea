La filosofia dopo Hegel è stata caratterizzata
da una messa in discussione della vecchia concezio­ne del trascendentale, più che da un esplicita
tematizzazione dei nuovi orizzonti che si erano
formati. Il tema esplicito della filosofia
fino a Weber era stato la presa d'atto dell'
impossibilità di stabilire l'orizzonte originario
all'interno del quale fondare il sapere: non si
accetta più un piano trascendentale ultimo.

Con poche eccezioni, la filosofia del '900
mantiene il proprio piano filosofico al di sotto
del piano trascendentale, senza perciò poterlo
fondare e tematizzare.

Nel secolo passato abbiamo assistito all'unione
delle molteplici istanze nate dalla dissoluzione
del paradigma soggettivistico, ad un nuovo
territorio comune: il linguaggio, la dimensione
in cui il pensiero si esprime, per cui consente
un rapporto con il sapere filosofico meno pregiudicato in partenza rispetto alle strutture tematizzate dalla filosofia post-hegeliana.

\chapter{Il pensiero fenomenologico-ermeneutico}
\bigskip
\bigskip
\section{Gli inizi dell'ermeneutica}

Nel periodo romantico con Schleiermacher,
l'ermeneutica da tecnica dell'interpretazione diventa  una disciplina filosofica: l'interprete deve
trasferirsi interiormente nella soggettività dell'autore e riprodurlo in se stesso, soprattutto
attraverso il sentimento.

Con Dilthey l'ermeneutica finisce per coincidere
con l'essenza stessa della filosofia, e attraverso
il sapere ermeneutico apprendiamo la totalità delle
manifestazioni spirituali (cultura, scienze sociali, arte, \dots).
In Dilthey si assiste ad un'evoluzione da un'ermeneutica introspettiva (basata sull'esperienza
vissuta, Erlebris) ad una focalizzata sull'espressione oggettiva e storica della vita.
Allargandosi alla totalità delle manifestazioni
dello spirito, l'oggetto specifico dell'ermeneutica
diventa il mondo storico.
Così si qualifica non solo l'oggetto dell'interpretazione,
ma anche il soggetto interpretante, che è determinato dalla vita storica (storicità): la trascendentalità non è più indipendente dalla coscienza
storica.

Nell'interpretazione vi è "comunanza", perché "colui che
indaga la storia è il medesimo che fa la storia".
Diverso è lo studio della natura, data l'assenza
di comunanza, che avviene secondo rapporti causali.
Il comprendere ermeneutico è un ritrovamento della
della soggettività che ha prodotto l'oggettività
esteriore; "il soggetto del sapere è identico con
il suo oggetto".
Nella storia i soggetti manifestano loro stessi in
espressioni oggettive che vengono consegnate alla
comprensione degli altri. Tutte queste espressioni
sono il medesimo spirito che si manifesta, ciò
che accomuna noi tutti, e in esso rivive la nozione
hegeliana di spirito oggettivo, anche se per
Dilthey l'essenza di esso è la vita, cioè una
realtà che non ha la struttura trasparente e logica
dello spirito hegeliano (che poteva risalire allo spirito
assoluto, cioè alla completa autotrasparenza
di sé e all'assoluta identità del soggetto con
l'oggetto).
Il circolo ermeneutico per Dilthey è incompleto,
dato il carattere vitale del processo stesso, cioè
lo spirito ha caratteristiche irrazionali, per cui ogni
interpretazione è solo relativa.
Il concetto di comunanza dell'ermeneutica di Dilthey
rivela che il vero soggetto ermeneutico è rappresentato
dalla connessione vitale che unisce soggetto e oggetto.
Il soggetto della storia dello spirito è
la vita e le connessioni operanti tra gli individui
storici. Queste relazioni intersoggettive non sono
relazioni esterne e dipendenti dagli individui, ma
hanno una loro dipendenza e oggettività,  risultando
come determinanti gli individui stessi. Sono perciò
queste relazioni il vero soggetto dell'interpretare
e della vita storica. Nell'ultimo Dilthey si
passa dal soggetto inteso come orizzonte originario,
all'operare intersoggettivo degli individui
come soggetto autonomo (non più prodotto degli
individui): in ciò non solo rivive lo spirito
oggettivo hegeliano e la tesi del primato
dei rapporti intersoggettivi rispetto all'autocoscienza, ma viene anticipata
l'originarietà dell'"altro" dal soggetto
e la posizione di una differenza irriducibile
ad esso, quale sarà tematizzata dai successivi
autori.

\section{La fenomenologia husserliana}