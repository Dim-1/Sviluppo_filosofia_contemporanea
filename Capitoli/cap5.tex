La filosofia dopo Hegel è stata caratterizzata
da una messa in discussione della vecchia concezio­ne del trascendentale, più che da un esplicita
tematizzazione dei nuovi orizzonti che si erano
formati. Il tema esplicito della filosofia
fino a Weber era stato la presa d'atto dell'
impossibilità di stabilire l'orizzonte originario
all'interno del quale fondare il sapere: non si
accetta più un piano trascendentale ultimo.

Con poche eccezioni, la filosofia del '900
mantiene il proprio piano filosofico al di sotto
del piano trascendentale, senza perciò poterlo
fondare e tematizzare.

Nel secolo passato abbiamo assistito all'unione
delle molteplici istanze nate dalla dissoluzione
del paradigma soggettivistico, ad un nuovo
territorio comune: il linguaggio, la dimensione
in cui il pensiero si esprime, per cui consente
un rapporto con il sapere filosofico meno pregiudicato in partenza rispetto alle strutture tematizzate dalla filosofia post-hegeliana.

\chapter{Il pensiero fenomenologico-ermeneutico}
\bigskip
\bigskip
\section{Gli inizi dell'ermeneutica}

Nel periodo romantico con Schleiermacher,
l'ermeneutica da tecnica dell'interpretazione diventa  una disciplina filosofica: l'interprete deve
trasferirsi interiormente nella soggettività dell'autore e riprodurlo in se stesso, soprattutto
attraverso il sentimento.

Con Dilthey l'ermeneutica finisce per coincidere
con l'essenza stessa della filosofia, e attraverso
il sapere ermeneutico apprendiamo la totalità delle
manifestazioni spirituali (cultura, scienze sociali, arte, \dots).
In Dilthey si assiste ad un'evoluzione da un'ermeneutica introspettiva (basata sull'esperienza
vissuta, \textit{Erlebnis}) ad una focalizzata sull'espressione oggettiva e storica della vita.
Allargandosi alla totalità delle manifestazioni
dello spirito, l'oggetto specifico dell'ermeneutica
diventa il mondo storico.
Così si qualifica non solo l'oggetto dell'interpretazione,
ma anche il soggetto interpretante, che è determinato dalla vita storica (storicità): la trascendentalità non è più indipendente dalla coscienza
storica.

\textbf{Nell'interpretazione vi è "comunanza", perché "colui che
indaga la storia è il medesimo che fa la storia".}
Diverso è lo studio della natura, data l'assenza
di comunanza, che avviene secondo rapporti causali.
Il comprendere ermeneutico è un ritrovamento della soggettività che ha prodotto l'oggettività
esteriore; \textbf{"il soggetto del sapere è identico con
il suo oggetto".
Nella storia i soggetti manifestano loro stessi in
espressioni oggettive che vengono consegnate alla
comprensione degli altri. Tutte queste espressioni
sono il medesimo spirito che si manifesta, ciò
che accomuna noi tutti, e in esso rivive la nozione
hegeliana di spirito oggettivo, anche se per
Dilthey l'essenza di esso è la vita, cioè una
realtà che non ha la struttura trasparente e logica
dello spirito hegeliano} (che poteva risalire allo spirito
assoluto, cioè alla completa autotrasparenza
di sé e all'assoluta identità del soggetto con
l'oggetto).
Il circolo ermeneutico per Dilthey è incompleto,
dato il carattere vitale del processo stesso, cioè
lo spirito ha caratteristiche irrazionali, per cui ogni
interpretazione è solo relativa.
Il concetto di comunanza dell'ermeneutica di Dilthey
rivela che il vero soggetto ermeneutico è rappresentato
dalla connessione vitale che unisce soggetto e oggetto.
\textbf{Il soggetto della storia dello spirito è
la vita e le connessioni operanti tra gli individui
storici.} Queste relazioni intersoggettive non sono
relazioni esterne e dipendenti dagli individui, ma
hanno una loro dipendenza e oggettività,  risultando
come determinanti gli individui stessi. Sono perciò
queste relazioni il vero soggetto dell'interpretare
e della vita storica. Nell'ultimo Dilthey \textbf{si
passa dal soggetto inteso come orizzonte originario,
all'operare intersoggettivo degli individui
come soggetto autonomo} (non più prodotto degli
individui): in ciò non solo rivive lo spirito
oggettivo hegeliano e la tesi del primato
dei rapporti intersoggettivi rispetto all'autocoscienza, ma viene anticipata
l'originarietà dell'"altro" dal soggetto
e la posizione di una differenza irriducibile
ad esso, quale sarà tematizzata dai successivi
autori.

\section{La fenomenologia husserliana}

Edmund Husserl (Germania 1859-1938) definisce
l'impianto fondamentale della fenomenologia:
questa è la scienza dei fenomeni ed ha
caratterizzazione non empirica ma trascendentale.

\subsection{Dagli oggetti ai fenomeni}


Il carattere interpretativo introdotto da Nietzsche,
e tematizzato da Dilthey in relazione
alla conoscenza storico-spirituale, viene
ribadito nella fenomenologia husserliana e
esteso a tutta l'esperienza.
L'atteggiamento pratico-valutativo per Husserl
ha a che fare con l'apparato conoscitivo in
generale, così che l'esperienza stessa non è
una successione di dati di fatto, bensì una
costituzione di propri oggetti.
\textbf{Il mondo non è evidente come sembra: è invece
l'esito di moltissime integrazioni di significato
rispetto a ciò che effettivamente appare ai
nostri sensi.}

 Il compito della fenomenologia è
ricostruire, partendo da ciò che si presenta come
già costituito, questo terreno di evidenze
primarie, quell'esperienza primaria alla base
del mondo di cose e oggetti che noi abbiamo
costruito, un originario che non ha ne natura
corale ne oggettiva.
Secondo Husserl l'originario non ha le caratteristiche
della cosa in sé (materia extra soggettiva che
affetta la nostra sensibilità), ma è qualcosa
di immanente alla nostra percezione: \textbf{l'originario
è l'insieme delle percezioni elementari, ovvero quei
"vissuti" che non hanno ancora forma oggettiva e
trascendente} (trascendente per Husserl è tutto ciò che non riguarda il dato, ogni astrazione compiuta dalla coscienza).
Come in Dilthey, ritorna il concetto di \textit{Erlebnis},
la struttura semplice del percepire non ulteriormente scomponibile, a partire dal quale costruiamo
le nostre cose. La ricostruzione fenomenologica
parte dagli oggetti trascendenti e giunge
ai percepiti immanenti nella coscienza stessa:
dai "\textit{cogitata}" al "\textit{cogitatio}" (puro percepire).
Come per Kant, anche in Husserl c'è una
\textbf{"materia" del conoscere}, che però, a differenza di
Kant, non ha nulla di extrasoggettivo, ma \textbf{e' quell'elemento della percezione non più ulteriormente
scomponibile e non riducibile, e che è del
tutto immanente alla vita psichica. E'
chiamato il "dato iletico" (materiale), per
differenziarlo da quel dato formale che poi diventa l'oggetto.
Husserl chiama il processo a ritroso, dall'oggetto alla materia percepita, riduzione,
in quanto si tratta di "ri-condurre" la
nostra esperienza e la complessità artificiale
degli oggetti alla sua struttura più intima
e vera , alla semplicità dell'\textit{Erlebnis}
(sentimento vissuto).
Solo l'orizzonte dei vissuti è indubitabile e
certo: la riduzione fenomenologica è
perciò scienza rigorosa.}


L'esistenza della cosa è priva di reale fondamento:
è un essere, che la coscienza pone nelle sue
esperienze, che è visibile e determinabile
come ciò che permane identico nella molteplicità
delle apparizioni, ma all'infuori di questo è
un nulla.
\textbf{Ciò che noi effettivamente percepiamo sono
solo le molteplici apparizioni, quelle che
Husserl chiama "adombramenti", le varie
sfumature chiaroscure della cosa, ed ad essa
perveniamo collegando tutte queste sensazioni
in un che di unico, ritenendoli modificazioni di un sostrato identico che chiamano
"cosa".
L'identità dell'oggetto nella pluralità delle
percezioni è un prodotto della coscienza sintetica
"che riannoda la nuova percezione con il ricordo".}


\textbf{L'originario non è l'oggetto ma i suoi modi
di apparizione}, sempre diversi; anche il collegamento di più sensi è un momento della nostra vita
psichica.
\textbf{La costruzione della cosa è un processo di
progressiva formazione di senso, in cui
attribuiamo i pur differenti modi di
apparizione ad un unico oggetto, che poi
riteniamo causa dei nostri percepiti ed
indipendente da essi.}
L'operazione con cui dal modo di apparizione
perveniamo alla cosa non è estrinseca o arbitraria
rispetto al singolo modo di apparizione, ma ha il
suo fondamento nella struttura in cui il singolo
modo di apparizione è inserito; lo "sfondo" da cui astraiamo la cosa
non è l'oggetto ma qualcosa di più indefinito e aperto, ed in esso è radicata quell'operazione di "arricchimento di senso" che sta
alla base della nostra esperienza di oggetti.
\textbf{L'idea della cosa non avviene quindi secondo una
ricostruzione categoriale come in Kant,
ma a partire da una tendenza immanente
alle stesse rappresentazioni, che deriva dal
fatto che ognuna di esse non è mai
conclusa, ma rinvia necessariamente al
di là di sé. La costruzione della cosa è
in definitiva un passaggio dalla potenza
all'atto, un'operazione in cui attualizziamo
le potenzialità immanenti a ogni singola
rappresentazione.
Ogni singola manifestazione attualmente presente
rinvia potenzialmente a tutte le altre rappresentazioni della cosa ed ad uno "sfondo" più
complesso in cui il singolo modo di apparizione è
un momento astratto.} Al singolo modo di
apparire perveniamo solo "estraendolo" dallo sfondo
complesso in cui originariamente si trova,
sfondo che non appare mai in atto, ma sempre
presente in potenza.

In conclusione, \textbf{apodittica (indubitabile) non è la cosa ma solo
il suo modo di apparizione, non l'oggetto
ma il fenomeno, ciò che rimane immanente
alla coscienza}. L'esistenza del mondo e
di noi stessi è messa in discussione dalla
fenomenologia, che si limita a descrivere ciò
che realmente appare.

Alla base dell'atteggiamento fenomenologico
sta \textbf{una radicale "\textit{epoché}"}, una sospensione del
giudizio sulla realtà e natura del mondo.
Tuttavia non viene negato ciò che vediamo, ma
tutta la nostra ricostruzione spazio-temporale:
\textbf{il mondo e la natura sono "messi fra parentesi"},
cioè permane come mondo di fenomeni, ovvero
di apparizioni immanenti alla coscienza.
Per Kant dietro al fenomeno c'è il "noumeno", la cosa
in sé, di cui il fenomeno è la manifestazione
all'interno del soggetto; per Husserl il fenomeno è la natura ultima delle cose, oltre il
quale non si da nessuna essenza noumenica.
\textbf{Il fenomeno husserliano è il puro darsi originario
di cui la cosa è una costruzione successiva.}

\textbf{In Kant il processo di costruzione della cosa
nella nostra mente è non solo necessario, ma
guidato da una logica trascendentale, per cui
è garantito nelle sue pretese di verità. Per
Husserl questo processo è logicamente
arbitrario e dunque, anche se psicologicamente
giustificabile e spiegabile, falso nella misura
in cui pretende di valere come vera forma
del mondo.}

La fenomenologia non è la descrizione dell'esperienza in quanto costituita dal soggetto
ma \textbf{la descrizione dell'esperienza originaria;
è la scienza del "modo" in cui ci appaiono
oggetti e corpi. E' una scienza non del "che"
(che cosa?) ma del "come" (come appaiono
quelle che chiamano cose?)}, contrapponendosi al
positivismo che pretende di descrivere i fatti.
Il fenomeno non è un fatto, ma un
modo di darsi e presentarsi del reale, e
per coglierlo occorre un atteggiamento
"riflessivo" e indiretto, contrapposto a
quello ingenuo-naturale diretto.

\subsection{La fenomenologia come scienza eidetica}

Rispetto a Hume, per Husserl \textbf{alla scepsi deve
seguire la scienza}: in Hume la ragione viene
presupposta, e così la psiche e un corpo che la
genera, contraddicendo quanto aveva sostenuto
riguardo la dubitabilità del mondo materiale.
Il fenomenismo humeano si riduce essenzialmente
a uno psicologismo, ovvero un'analisi delle
rappresentazioni soggettive (impressioni e idee),
con la descrizione degli stati di coscienza,
ovvero dei modi nei quali il nostro apparato
conoscitivo raccoglie i dati della percezione.
Ma la fenomenologia non è scienza di dati
di fatto (ancorché immanenti alla coscienza),
bensì scienza di essenze, o scienza "eidetica".
Per essa \textbf{è impossibile descrivere in maniera obiettiva
un fenomeno}: la psicologia vi riesce solo in quanto
lo concettualizza, trasformando un modo di
apparire in un dato di fatto della psiche.

"Ogni \textit{Erlebnis} (vissuto) rinvia ad altri \textit{Erlebnisse}"
e così via in maniera infinita (infatti lo "sfondo" è
sempre in potenza); perciò \textbf{ogni opera di
astrazione è una falsificazione di ciò che
realmente si mostra.}

Nonostante ciò secondo Husserl è possibile
una descrizione rigorosa delle essenze cui
i fenomeni si riconducono: infatti \textbf{i fenomeni
si danno all'interno di "modalità tipiche"
al di fuori delle quali non è possibile alcuna
esperienza}. Ad esempio non si può dare un
colore senza una superficie, o una luminosità senza colore, eccetera. \textbf{C'è una "struttura
essenziale dell'esperienza" che si presenta allo
sguardo fenomenologico nella sua innegabile
evidenza: a questo si rivolge la fenomenologia, evitando così la deriva scettica e la
rincorsa senza fine al fondo dei fenomeni.}

\textbf{Il fondo ultimo dei fenomeni si raggiunge
grazie ad una seconda riduzione, successiva a
quella fenomenica, la "riduzione eidetica",
la quale consente di ricondurre i fenomeni
ad un terreno ancora più originario,
costituito dai loro modi universali di
manifestarsi, cioè al loro "\textit{eidos}".}
L'\textit{eidos} fenomenologico ha natura pre-concettuale,
quindi non può essere astratto attraverso un processo
logico e di generalizzazione. In quanto struttura
essenziale comune ai vari modi di apparizione dei
fenomeni, \textbf{vi si accede grazie all'intuizione, cioè
all'indubitabile immediatezza del vedere
fenomenologico} ("\textit{Wesensschan}", intuizione o
visione dell'essenza).

In quanto strutture essenziali del modo di darsi
dell'esperienza, le essenze possono riferirsi
sia a fenomeni psichici (gli stati o atti dell'io)
sia a quelli fisici, e così accanto ai modi di
presentarsi dei corpi o dei suoni, si daranno quelle
delle percezioni, dei ricordi e dei giudizi.
Ma il punto fondamentale è che in queste
essenze sono depositate le leggi fondamentali
della nostra esperienza, leggi che non
provengono kantianamente dal soggetto ma sono
immanenti all'esperienza stessa. \textbf{Le \textit{eidos} (il modo di darsi dei fenomeni)
svolgono perciò la funzione di "apriori", di
condizioni di possibilità dell'esperienza;
Husserl lo definisce" apriori materiale"}, perché
immanente alla "materia" del mondo, \textbf{differenziandolo dall'"apriori" formale kantiano.}
\textbf{Questa legge essenziale che regola la nostra
esperienza è indipendente dal soggetto }o dall'esigenza di avere un'esperienza ordinata;
\textbf{è una "legalità oggettiva" appartenente alla
natura dell'esperienza}. Grazie a questa
legalità la fenomenologia realizza il suo
programma di una descrizione rigorosa
del terreno originario dell'esperienza.

Individuata questa legalità fenomenologica, che
riesce a descrivere il terreno originario dell'esperienza,
rimane da osservare se un tale accesso è possibile:
un accesso immediato all'esperienza, privo di mediazioni concettuali; per dirla con Kant, capire se è possibile un'intuizione
senza concetti.

\subsection{Il soggetto trascendentale}

\textbf{La riduzione eidetica non può fermarsi finché non abbia
saputo riconoscere il vero universale di tutti i modi di
darsi dei fenomeni: l'"Io trascendentale".
La "riduzione trascendentale" riconduce le essenze
all'Io: "l'intero mondo oggettivo" è per me; quindi l'Io inteso da Husserl altro non è che il puro darsi dei fenomeni, il puro apparire del mondo.}
Questa trasformazione del mondo, ridotto al
soggetto, comporta alla fine anche una
trasformazione del soggetto stesso.

Definendo l'Io trascendentale, Husserl intende
differenziarlo da quello che egli chiama
"Io psicologico", ovvero la nostra identità
individuale: essa è esito di una ricostruzione, e non
può costituire quell'orizzonte originario cui tutto
viene ricondotto. Husserl critica la concezione psicologica
humeana del soggetto: l'identità è una finzione
psicologica, esito di una somma arbitraria di una
serie di rappresentazioni.
"I fenomeni della fenomenologia trascendentale vengono
caratterizzati come irreali", non hanno realtà in
loro stessi ma dipendono dall'Io, cioè dall'unica
vera realtà.
In sostanza, per non ricadere nello psicologismo guardando alle rappresentazioni psichiche come a elementi della
natura umana, bisogna ricondurre i fenomeni a
quell'\textbf{unità non psicologica, non umana,
non mondana che è l'Io trascendentale.}
\textbf{L'Io  è posto come condizione trascendentale
dell'esperienza: Kant colloca questa scoperta
all'interno del mondo (delle cose in sé); la
riduzione trascendentale dissolve ogni datità
delle cose e svela la fenomenologia come una
forma di idealismo}.

Husserl però prende le distanze dall'idealismo classico tedesco:

\begin{itemize}


\item \textbf{la fenomenologia non deduce nessuna totalità,}
bensì procede in modo inverso, partendo dell'esperienza e riconducendo i fenomeni nella loro
infinita varietà ai modi tipici di manifestarsi
e quindi da ultimo all'Io. \textbf{Il pensare è
visto come forma dei fenomeni, non separabile
da essi e di cui è la manifestazione}, e non
può perciò generare nessun automovimento
a priori da cui dedurre il mondo.

\item  Rispetto a Hegel, \textbf{il mondo dei dati empirici
è "messo tra parentesi": non viene messa in
discussione la natura sensibile ma solo la tesi
che lo interpreta come esistente all'infuori
di noi}. Al contrario in Hegel il fenomeno
era ricondotto a strutture logiche concettuali
in cui esso perde i suoi tratti empirici.

\item  \textbf{Resta centrale nella fenomenologia la
natura "intenzionale" della coscienza, ovvero
il suo costitutivo riferirsi sempre a qualcosa,
il suo intendere sempre qualcosa. I "vissuti",
"in quanto sono coscienza di qualcosa, si
dicono intenzionalmente riferiti a questo
qualche cosa". Non si da coscienza senza
che essa sia al tempo stesso coscienza di
qualcosa}: ciò non significa che quel qualcosa
è una realtà sussistente e indipendente dalla
coscienza (come vedremo essere in Sartre), ma che \textbf{quel qualcosa è condizione del
darsi della coscienza in quel momento. Poiché la
coscienza non potrebbe esistere senza ciò che
essa di volta in volta intenziona, il rapporto
tra soggetto e mondo non è pensato nei termini
idealistici dell'identità, ma in quelli della
correlazione: non si può pensare la coscienza
senza il mondo fenomenico intenzionato da
essa.}

\end{itemize}

\subsection{L'intersoggettività}

Nelle "Meditazioni cartesiane" Husserl descrive come
\textbf{il metodo fenomenologico è in grado di accettare
altri soggetti}: non in senso mondano (spaziale e
naturale), ma la semplice esistenza di un'altra
esperienza, oltre a quella vissuta dal proprio io.
Husserl la definisce \textbf{"esperienza dell'estraneità",
vale a dire la possibilità di poter distinguere all'interno
della mia esperienza fra ciò che è mio proprio e ciò
che mi è estraneo.}
Egli parte dalla possibilità di poter circoscrivere
all'interno dei fenomeni di cui si ha esperienza quelli
che appartengano alla sfera del mio proprio, cioè
quelli che rinviano alla mia soggettività e
caratterizzanti la mia esperienza interna:
di quest'ultima è parte essenziale quella che
Husserl chiama \textbf{"esperienza del proprio corpo"}:

\begin{itemize}
	\item \textbf{\textit{Korper}, corpo fisico-spaziale, di cui ho
	esperienza esterna.}
	\item \textbf{\textit{Leib}, corpo organico, con cui si ha sensazione
	interna del proprio corpo, come veicolo di
	conoscenze percettive e perciò più "soggetto"
	che "oggetto" dell'esperienza.}
\end{itemize}

In altre parole \textbf{il mio corpo può essere sia oggetto
di osservazione (\textit{Korper}) che soggetto percepiente
(\textit{Leib}), cioè le due parti sono reciprocamente connesse.
Ciò rende possibile distinguere un \textit{Korper} estraneo
come non connesso al proprio \textit{Leib}, ma ad
un altro, sulla base della somiglianza del
corpo estraneo con il proprio. Ciò rende possibile
l'esperienza di un altro \textit{Leib}, cioè l'immedesimazione
da parte mia nell'interiorità altrui.}

\textbf{L'altro non è percepito in modo evidente, come
fenomeno originario (apodittico), ma attraverso
una percezione indiretta ("appercezione") che
avviene sulla base di una prima percezione, che
rende presente ciò che non lo è (\textit{Leib} dell'altro).
L'altro soggetto è "costituito"}, non è semplice
presenza attuale, bensì avviene mediante
trasferimento di senso;\textbf{ in altre parole l'altro soggetto viene interpretato.
Da qui si dischiude l'intera dimensione dell'intersoggettività:} non vi è alcuna differenza
ontologica fra la mia natura e quella altrui;
"essi sono per sé come io sono per me".

\textbf{Se tutti condividiamo la stessa natura, assieme
siamo "soggetti trascendentali"; non sono
un soggetto solitario ma parte di una
comunità, detta "l'intersoggettività trascendentale",
a partire dalla quale si costituisce "un unico
identico mondo".}

\textbf{Il mondo oggettivo è perciò il prodotto della
convergenza di differenti intenzionalità
intersoggettive (di intersoggettività trascendentale),
e il singolo soggetto può avere esperienza in
quanto si trova in comunità con sui simili.
Husserl, a partire dall'assoluta immanenza degli atti
intenzionali del soggetto, costituisce un mondo
intersoggettivo trascendente rispetto a quegli atti.}
Un ruolo centrale è svolto dal processo di
immedesimazione, per cui pensiamo che un
altro \textit{Korper} possa avere la nostra stessa
esperienza interna. Con questa operazione, la
fenomenologia travalica ciò che è evidente
e diventa capace di aprirsi al mondo dei
soggetti reali, trascendenti rispetto ai
fenomeni, rendendo accessibile la complessa
realtà del mondo umano e delle sue relazioni.

\textbf{L'attestazione fenomenologica della relazione di
somiglianza non è coerente con il rigore
apodittico a cui la fenomenologia dovrebbe
attenersi}: si perde in rigore attribuendo realtà
ad un soggetto perché un corpo assomiglia
al proprio. Inoltre una relazione di somiglianza
non autorizza quel trasferimento di senso che
finiva per attribuire a un corpo estraneo
la soggettività che si accompagna al mio corpo
proprio. \textbf{Attenendosi al puro vedere non potrò mai
vedere un \textit{Leib} nell'altro, ma sempre e solo un \textit{Korper}.}

\subsection{La crisi delle scienze europee (1935-37)}

L'ultima grande opera di Husserl ci giunge incompleta.
\textbf{La crisi delle scienze consiste nella perdita del loro
significato per la vita, perché per esse contano solo
i fatti; perciò il prezzo di questa attenzione esclusiva
all'obiettivabile è la dimenticanza di quel soggetto
che è l'origine prima del processo di oggettivazione.
La critica dell'"obiettivismo scientifico" è la riduzione
del mondo a universo di oggetti ritenuti indipendenti dal soggetto osservante, e considerati sotto
l'aspetto meramente quantitativo} (come già osservato da Weber con la sua teoria di razionalizzazione del mondo, intesa come quantizzazione misurabile e calcolabile, che porta al disincanto da parte dell'umanità,  ad una mancanza di significato nello svolgersi della vita e ad una diminuita libertà personale, essendo gli individui guidati da questa razionalità, intesa come burocrazia ed economia).

Per Husserl \textbf{dobbiamo compiere un'operazione fenomenologica,
ricostruendo l'esperienza qualitativa alla base del
processo di scientificizzazione e che da esso è stata
rimossa.} Questa presa di distanza è detta da Husserl:
\textbf{"l'\textit{epoché} della scienza obiettiva".
Questo scetticismo non conduce però dagli oggetti
ai fenomeni ma ad una sorta di terreno intermedio,
detto "\textit{Lebenswelt}", il mondo della vita o mondo
vitale: un vero e proprio terreno vitale di costituzione,
capace di costruire quel mondo oggettivo che caratterizza
l'immagine scientifica del mondo.
Il mondo vitale è quello osservato con occhi
prescientifici, fatto di qualità e non di quantità,
dove tutte le cose sono "soggetto-relativo",
ovvero ancora in relazione con la nostra interiorità.}

La \textit{Lebenswelt} ha dunque alcune caratteristiche che
la prima fenomenologia husserliana individua nelle percezioni
vitali primarie (i vissuti, gli \textit{Erlebnisse}), e in primis
la correlazione con la nostra soggettività. Tuttavia
\textbf{il mondo vitale è un mondo di cose collocate
nello spazio-tempo, dove queste cose stanno in
rapporto con noi, osservate a partire dalla nostra
prospettiva e non in maniera asettica (come nelle
scienze).
Per la nostra esperienza quotidiana di vita è
"sicuro" ciò che per il fenomenologo non lo è, vale
a dire l'esistenza di oggetti e di un mondo
al di fuori di noi. L'obiettivismo scientifico
si sviluppa a partire dalla convinzione di fondo
che attraversa il mondo vitale, la fede (radicalizzata) nello
spazio-tempo} (sempre come affermato da Weber, crediamo nella razionalità,  perché ci serve da un punto di vista pratico, senza tuttavia dare nessuna dimostrazione teoretica della razionalità del mondo).

 \textbf{La \textit{Lebenswelt} non è solo un mondo
di cose, bensì una prospettiva originaria a partire
dalla quale il soggetto guarda il mondo: essa
ha dunque la posizione di un orizzonte trascendentale,
sulla base di cui si costituisce la nostra esperienza.
Questo terreno primario ha natura pratico-vitale,
è descrizione di un mondo pre-categoriale al cui
centro è la dimensione primaria della vita umana}
(recupero di Dilthey e della sua visione della vita
come orizzonte intrascindibile, condizionante il
modo in cui i soggetti si relazionano al mondo).
\textbf{Questo orizzonte trascendentale è a sua volta
costituito}, carico di integrazioni di senso, e,
per Husserl e il filosofo, \textbf{bisognoso di essere sottoposto ad
una riduzione trascendentale: il mondo vitale}
non va indagato con l'atteggiamento della vita
naturale mondana, ma \textbf{richiede "un'\textit{epoché} universale
assolutamente peculiare", ponendoci al di sopra di
quella coscienza nella quale il mondo è "qui",}
l'universo di tutto ciò che è alla mano.

\textbf{L'indagine sull'orizzonte trascendentale della
vita riconduce alla soggettività trascendentale,
senza tuttavia richiudersi dentro il
paradigma soggettivistico della modernità, in
quanto anche il soggetto si è rivelato costituito
(messo fra parentesi dall'epoché fenomenologica).
Ciò che rimane dopo le riduzioni fenomenologiche è
solo il puro apparire dei fenomeni, e\textit{} il trascendentale
ultimo è ciò che rende possibile il manifestarsi
fenomenico, ciò che Husserl chiama "Io
trascendentale". Quindi il soggetto che ha ridotto
tutto a sé (compreso se medesimo) è diventato
identico con l'apparire del mondo.}

\textbf{L'intersoggettività viene qui spiegata da Husserl con il
processo di "auto-estraniazione"}, un'operazione grazie
alla quale il soggetto trasferisce la propria soggettività
all'altro, all'estraneo. \textbf{L'attribuzione della
soggettività è analoga alla costituzione del sé
come soggetto}: percepiamo di noi stessi la nostra
attualità e la presenza, nella nostra attualità, del
ricordo di noi stessi nel passato, e ciò ci consente di collegare
la percezione del sé attuale con il ricordo del sé
passato, costituendo così un io permanente nel
tempo. \textbf{Il "tu" viene costituito in modo analogo
al "sé" temporale: invece che un trasferimento
dell'io nel tempo, si compie un trasferimento
dell'io nell'estraneo.}
Anche qui \textbf{il processo di "auto-estranazione" mostra
la debolezza fenomenologica della relazione di
"somiglianza"}. Inoltre, conferire permanenza all'io
non ha fondamento logico ma solo psicologico, dato che
un soggetto permanente nel tempo non potrà mai
apparire come attualmente evidente, così per il
processo di auto-estraniazione, dove \textbf{non c'è nessuna
evidenza che consenta al soggetto trascendentale
di attribuire la propria soggettività a un corpo
estraneo.}

Inoltre \textbf{emerge la tensione fra soggettività  che si è posta come trascendentale}, cioè identica all'intero orizzonte dell'apparire, \textbf{e una pluralità di soggetti ai quali è attribuita la medesima trascendentalità: se alla soggettività altrui va attribuito il medesimo statuto della soggettività originaria, allora il soggetto trascendentale cessa di essere l'unico orizzonte di senso, vedendo messa in discussione la sua pretesa di trascendentalità}. Abbassato a semplice punto di vista accanto ad altri, diventa qualcosa di molto simile al soggetto empirico-psicologico. \textbf{E' sempre la nozione di intersoggettività trascendentale a mostrarsi intrinsecamente contraddittoria.}

Le procedure fenomenologiche stabilite da Husserl rendano indimostrabile una comunità di soggetti, come abbiamo già avuto modo di osservare. Tuttavia \textbf{Husserl ha bisogno di altri soggetti, in quanto l'esperienza non è desoggettivizzata, ma ha natura egologica, cioè, per poter definire in termini di "io" l'orizzonte dell'esperienza, necessita di un "tu". L'intersoggettività, punto debole della filosofia di Husserl, è condizione necessaria per la stessa comprensione di un "io" trascendentale. Il vero punto debole della fenomenologia husserliana è l'insistenza sulla radicalità di un puro vedere intuitivo, privato di qualsiasi mediazione concettuale, linguistica o pratica, a produrre la "messa tra parentesi" del mondo, che rende la fenomenologia irriducibile a qualsiasi esito di "mondanizzazione".}

\section{Heidegger: Essere e tempo (1927)}

Martin Heidegger (1889-1976) pone la sua opera in
diretta continuità con quella di Husserl. Suo scopo
è manifestare le cose nella loro essenza, al di là di
tutte le indebite integrazioni di senso che si sovrappongono a ciò che esse sono in verità: in questo quadro
si può collocare il programma heideggeriano di intendere
la fenomenologia come ontologia. Infatti il fenomeno
nella sua originarietà è proprio l'essere, ovvero quella dimensione
che di solito rimane nascosta, e a cui è perciò necessario
lo sguardo fenomenologico per mostrarla. "Dietro i
fenomeni della fenomenologia non c'è essenzialmente
nient'altro", \textbf{l'essere che si propone di manifestare non è
qualcosa al di là dei fenomeni ma è il loro pieno manifestarsi.}

\subsection{\textit{Dasein} come trascendentale}

Quanto detto non conduce Heidegger a una
riabilitazione della concezione realistico-sostanzialistica dell'essere. Ciò che si manifesta allo sguardo
del fenomenologo rimane nella correlazione
essenziale col soggetto, non è indipendente.
\textbf{In "Essere e tempo" (\textit{Sein und Zeit}, 1927) il mondo
si costituisce all'interno di quell'ente particolare che
è l'essere umano, che tuttavia Heidegger non
definisce più soggetto, ma lo qualifica come
"\textit{Dasein}", ovvero l'Esserci , l'esser-qui, l'essere in
quanto si determina in un "dove".}

Qui sono tratte le
conseguenze dell'esito fenomenologico che aveva fatto
coincidere l'io con l'apparire del mondo.
\textbf{L'essere del mondo non è indipendente dal \textit{Dasein}: la
"mondità" (\textit{Weltlichkeit}), cioè l'appartenenza al mondo
delle cose che ci circondano, non è un attributo che
differenzia le cose da noi, ma le fa nostre, in quanto
essa sono mondane per virtù nostra. La mondità è}
un nostro attributo, o, secondo la terminologia
heideggeriana, \textbf{un "esistenziale", cioè un carattere
costitutivo e inaggirabile del \textit{Dasein}.}


\textbf{L'analitica del mondo passa attraverso l'analisi del
\textit{Dasein}, così come in Husserl passa dall'Io
trascendentale}. Altra analogia con il maestro, \textbf{il
\textit{Dasein}} non può essere inteso come soggetto psicologico,
cioè  \textbf{non si contrappone al mondo come al suo oggetto,
ma} \textbf{è l'essere dell'oggetto, è l'apertura al mondo,
quell'orizzonte che non può essere una cosa ma
rende possibili le cose}.

\subsection{La critica ad Husserl e all'antologia tradizionale}

\textbf{Heidegger imputa ad Husserl una concezione
ristretta della riduzione fenomenologica, la quale,
invece di manifestare l'essere nella sua pienezza,
si limita a manifestare solo una dimensione,
quella della presenza allo sguardo, dell'apparizione hic et nunc.}
Come annuncia il titolo dell'opera, Heidegger
si propone di dimostrare come \textbf{essere e tempo
non siano in rapporto estrinseco ma si richiamino
reciprocamente. L'essere è strutturalmente
temporale }e non aver compreso questa essenziale
implicazione ha costituito il limite fondamentale
della tradizione ontologica, che ha sempre
privilegiato il presente rispetto alle altre dimensioni
\footnote{infatti "ente" è participio presente del verbo essere, così
	come "essere" è l'infinito presente.}.

\textbf{La metafisica ha inteso l'essere come presenza
stabile}: solo ciò che è stabilmente presente può
essere definito ente; \textbf{è uno stare nel tempo, che
attraversa il tempo come presenza stabile.}
Da ciò la preminenza che fin dai greci ha avuto
l'atteggiamento teoretico-osservativo (contemplativo) rispetto alle
altre dimensioni dell'umano. Prima di Heidegger è sempre stata la teoria ciò che
da accesso alla dimensione dell'esser presente: vero è
ciò che appare ora al suo sguardo.


\textbf{La riduzione fenomenologica husserliana approda
all'eidos eliminando le integrazioni di senso
sovrapposte alla pura fenomenicità, ma contemporaneamente eliminando tutte quelle dimensioni che non
sono riducibili all'orizzonte di ciò che è presente
allo sguardo (che è presente "sottomano").}

Heidegger si propone di ripercorrere nuovamente
il cammino della riduzione fenomenologica,
per giungere alla vera dimensione primaria in cui
l'essere del mondo viene dato al \textit{Dasein}.
\textbf{Questo manifestarsi del mondo all'esserci ha
la caratteristica che
il suo costituire il mondo non è allontanarsi
da esso, ma l'istituire una relazione necessaria
con il mondo. Così come il mondo non può essere
senza \textit{Dasein}, ugualmente il \textit{Dasein} non può essere
senza il mondo.} Questo rapporto di appartenenza
non può essere espresso dalle categorie della metafisica
della presenza, piuttosto come \textbf{un rapporto di familiarità
reciproca. Il "qui" del \textit{Dasein} è uno "stare-presso",
l'avere familiarità con il mondo che esso stesso
costituisce.
Da ciò le altre determinazioni che Heidegger
attribuisce al \textit{Dasein} , ovvero "l'in-essere" e
"l'essere-nel-mondo".}

\textbf{L'esperienza originaria del mondo} non \textbf{è} l'esperienza
teoretico-osservativa, bensì \textbf{l'esperienza pratica del
"prendersi cura" del mondo (avere familiarità con il mondo).
Le cose del mondo si presentano sotto l'aspetto
pratico della loro utilizzabilità, e il mondo è
una totalità di cose utilizzabili.}
L'in sé dell'ente non è l'ente ridotto alla mera
essenzialità ontologica-presentificante, ma è il
contrario di ogni in sé, ovvero \textbf{l'essere-per, la
disponibilità pratica, la maneggevolezza, la relazione.}

In definitiva,\textbf{ la fenomenologia heideggeriana non deve
condurre alle essenze ideali ma all'universo dei
"\textit{pragmata}"} ("ciò con cui si ha a che fare nel commercio
prendente cura"), dunque si tratta di ridurre
quegli stessi fenomeni (husserliani) ad un
orizzonte ancora più originario, ovvero al
\textbf{rapporto pratico-poietico verso le cose.}

\subsection{Trasformazione del trascendentale e \textit{Dasein} come "cura"}

Il \textit{Dasein}, ovvero quell'essere a cui il mondo stesso si
riconduce, non è un mero orizzonte trascendentale
inteso come il puro apparire degli eventi (come sostenuto da
Husserl), quindi ristretto alla dimensione
della presenza in un certo istante.
\textbf{Heidegger ritiene invece che la natura del \textit{Dasein} sia
espressa nel concetto di esistenza}: introdotto per caratterizzare il \textit{Dasein}, questo concetto
di esistenza non
indica l'attualità della cosa, la sua dimensione ontica
presente ("l'esser presente sottomano"); \textbf{essa significa
al contrario l'esser al di là della mera datità, poter
essere qualcosa di diverso da ciò che è presentemente.
\textit{Dasein} è un non ente, ma possibilità di essere e
progetto: ciò che esso è gli deriva dalla sua stessa
possibilità, ovvero la sua realtà gli appartiene,
in quanto non gli viene data dall'esterno ma è
il realizzarsi della sua possibilità.
Se l'esistenza va intesa come possibilità e apertura (come essere qualcosa di diverso da ciò che si è presentemente),
ciò determina la natura del \textit{Dasein} come pratica,
proprio perché non è una cosa ma poter essere,
esso è essenzialmente progetto e prassi.}
Il mondo non è dischiuso da un orizzonte costituito
dall'io puro teoretico ma da un orizzonte pratico-progettuale.

Il carattere di progetto della comprensione non
significa che esso coglie in modo tematico ciò
rispetto a cui progetta, cioè la possibilità, ovvero
\textbf{l'esserci non edifica il suo essere in base a un
piano. L'origine del suo progettare sfugge al \textit{Dasein},
perché esso è un "progetto gettato", è in balia
del mondo, gettato in un "\textit{da}" (qui).
Il \textit{Dasein} è la condizione trascendentale del mondo,
ma al tempo stesso è consegnato a questo mondo
che lui stesso ha aperto: l'orizzonte che fa essere
il mondo non è dominato dal suo "esser qui", infatti
non conosce il donde e il suo dove. La sua
trascendentalità convive con la sua fatticità} (che non
è fattualità), cioè Heidegger apre a un nuovo
significato del trascendentale: \textbf{la condizione di
possibilità è a sua volta condizionata.}

\textbf{Il \textit{Dasein} è contemporaneamente progetto e affezione,
infinite possibilità  e finitezza dovuta al
trovarsi già in una certo orizzonte che genera una
"situazione emotiva", cioè la fatticità consiste
nel suo esser determinato da altro e l'esser
consegnato alla sua propria situazione.}
Questo intreccio di trascendentalità e fatticità,
\textbf{di apertura e dipendenza dal mondo}, viene inquadrato
nella celebre definizione heideggeriana dell'\textbf{"essere
dell'esserci come cura"}. Cura non è una
pratica mondana, ma \textbf{un'"apriorità esistenziale"
che dispone il \textit{Dasein} in un atteggiamento di
coinvolgimento emotivo con il suo ambiente.
Definire l'esistenza come cura o se preferiamo, come l'avere familiarità con il mondo, significa} che
non contempliamo distaccati il mondo, ma \textbf{che
abbiamo un rapporto coinvolgente con le cose, una
"pratica" con esse, una partecipazione interessata.
Qualsiasi sguardo sul mondo risulta intrecciato con
il "commercio" che intratteniamo con esso, ovvero abbiamo
una "visione ambientale" che non può prescindere
dal carattere di "roba utile" che ha tutto ciò
che noi incontriamo. La cura può essere intesa
come "prassi originaria" nel quale il \textit{Dasein} si
trova coinvolto e che costituisce il suo essere
specifico.}

\textbf{Sono tre i caratteri fondamentali della cura:}
\begin{enumerate}
	\item \textbf{la progettualità esistenziale}, cioè l'esistenza
	come poter essere, come essere "avanti a sé";
	\item \textbf{la gettatezza,} l'esser consegnato in un mondo,
	che ci assegna in un "già";
	\item \textbf{il prendersi cura delle cose del mondo}, cioè
	l'esser-presso gli utilizzabili.
\end{enumerate}

Queste tre modalità svelano la temporalità del
\textit{Dasein}, rispettivamente futuro, passato e presente.

\subsection{La tematizzazione dell'intersoggettività}

Nel mondo incontriamo sia l'insieme dei mezzi
sia gli altri \textit{Dasein}; perciò \textbf{il \textit{Dasein} esiste per
la rete di coinvolgimenti pratici con le cose e
per le relazioni con gli altri. Heidegger non fornisce una dimostrazione dell'esistenza degli altri soggetti}, perché il \textit{Dasein} isolato (da presupporre per poi dimostrare l'esistenza di altri individui), senza il mondo e senza gli altri, è impossibile, una pura astrazione basata sulla falsa presupposizione che l'essere sia esclusivamente presenza.

\textbf{L'intersoggettività è Con-esserci, un ente comunitario che vive in uno spazio esistenziale di pratiche e di cure e non in uno spazio esteso dove trovano collocazione altri soggetti; essi non sono oggetti osservabili ma esserci legati fra loro in una visione regolata dalle forme della cura}. Come ci si prende cura del mondo, così si ha cura degli altri: ciò non va inteso in senso etico, bensì la cura regola il rapporto con gli altri anche quando addirittura lottiamo contro essi.
In "Essere e tempo" scompare quell'intersoggettività teoretica, la cui conclusione terminava con inevitabili aporie, e si fa avanti \textbf{un'intersoggettività pratica, in cui noi siamo già da sempre insieme agli altri. Questa è la precondizione perché gli altri possano essere compresi da noi: se l'altro non fosse già in me rimarrebbe per sempre un'alterità irraggiungibile.}

\subsection{Verso l'ermeneutica e l'incontro con il linguaggio}

In Heidegger la descrizione fenomenologica non è
più orientata a un universo di pure evidenze
apodittiche, bensì a quell'orizzonte
pratico-vitale che si dischiude a partire da un
\textit{Dasein} concepito come "vita fattizia". In questo
quadro \textbf{la descrizione fenomenologica diventa
interpretazione e la fenomenologia trapassa
nell'ermeneutica}; questa poggia su fondamenti nuovi
e diversi rispetto a quelli introdotti da Dilthey.

\textbf{Il mondo non è una totalità di cose davanti a noi
ma di cose "per" noi, cioè di mezzi: la caratteristica
fondamentale del mezzo è quella del utilizzo per
fare qualcos'altro, o per dirla come Heidegger, del
rimando, del rinvio a qualcos'altro che non è
immediatamente presente, dell'essere qualcosa per.
Il mondo si rivela perciò una totalità di enti che
rimandano ad altro e significano altro: in questa
rete di rapporti consiste la significatività, che rappresenta la
struttura del mondo.}

\textbf{Il mondo è il senso, non unico, ma una totalità
infinita e mai conclusa di rimandi, in cui il \textit{Dasein}
è "gettato".
La comprensione comporta un rapporto interpretativo,
in cui il \textit{Dasein} riesce a dischiudere la
significatività delle cose e il loro strutturale
rinvio ad altro. Ogni ente è strutturalmente
temporale e dunque sempre rinviante al di là di
ciò che è presente allo sguardo.}

Il primo carattere dell'ermeneutica heideggeriana si fonda sulla strutturale
temporalità-storicità dell'ente.
La nostra natura interpretativa e il nostro muoverci in uno spazio fatto di segni e
di mezzi non si
basano solo sulla struttura rinviante degli enti:
\textbf{i significati non rimangono muti ma si
articolano in parole, trovano cioè espressione
nel linguaggio.
Il discorso è l'articolazione della comprensibilità,
quindi l'interpretazione del mondo è l'interpretazione delle parole nelle quali si esprimono i differenti e plurali significati del mondo.
Il discorso è il linguaggio incarnato nelle pratiche
quotidiane, "il linguaggio in senso esistenziale": Heidegger intende dire che
il rapporto linguistico con il mondo avviene solo con
modalità pratiche (discutendo, intercedendo, asserendo,
avvertendo, eccetera), e non fissando una struttura
linguistica indipendente dalla cura del \textit{Dasein}
(che equivarrebbe ad una caduta nell'ontologia
della presenza).
In questa modalità del nostro discorrere si realizza
non solo un incontro comunicativo con gli altri
esserci, ma soprattutto il nostro contatto con le
cose. La nostra comprensione del mondo non
è un "vedere" enti semplicemente presenti, ma avviene
parlando intorno a essi}, trovando le parole in
grado di esprimere il significato e la complessa
struttura dei rimandi, discutere con gli altri.

Rispetto a Dilthey, l'ermeneutica heideggeriana
ritiene che la storicità dell'ente non è solo l'elemento
comune tra soggetto e oggetto, ma  il fondamento
dell'apertura dell'ente e del suo caratteristico
rinviare; inoltre essa si fonda non tanto sulla
natura pratico-vitale quanto sull'esprimersi
linguistico del mondo. 
\textbf{Il discorso è un
esistenziale, elemento costitutivo ed
essenziale del rapporto fra \textit{Dasein} e mondo:
il mondo ci viene incontro con le parole e
il nostro compito è quello di interpretarle.}

\textbf{Con Heidegger l'ermeneutica passa dall'essere
l'essenza stessa della filosofia (Dilthey) a
connotare la nostra stessa esistenza, ad essere
il carattere proprio del nostro rapporto con il
mondo.}
La connessione vitale tra soggetto e oggetto è la
condizione irrinunciabile perché si instauri un
rapporto di comprensione: partendo da questa
intuizione diltheyana,\textbf{ Heidegger sostiene 
il rapporto di comunanza che lega
il soggetto interpretante all'oggetto come
un legame con la cosa che precede la
conoscenza della cosa stessa nella coscienza
dell'interprete. Nel processo di interpretazione
siamo guidati da una pre-comprensione,
un'anticipazione che funge da condizione e guida
nel comprendere.
Se ogni oggetto è già compreso prima che cominci il
processo dell'interpretare, il movimento ermeneutico
consisterà in un "circolo", in cui la comprensione
è guidata dalla pre-comprensione, che viene
a sua volta rimessa in discussione, rielaborata
a partire dalla comprensione.}

\textbf{La pre-comprensione è la condizione della comprensione,}
non possiamo liberarci di questa pre-struttura, neppure
attraverso una sorta di  "trasferimento interiore"
nell'interpretato per avere una comprensione
ideale, come sostenuto da Dilthey precedentemente:
ciò è irrealizzabile e controproducente, in
quanto elimina quella pre-comprensione che è
condizione del comprendere.

\subsection{La critica heideggeriana della società}

La collocazione mondana dell'esserci, l'esser collocato
in un "ci", in un luogo fatto di pratiche,
usi, condizionamenti, progetti, rapporti intersoggettivi,
discorsi e interpretazioni, può essere la sua rovina.
\textbf{Proprio perché aperto alle cose, coinvolto
emotivamente nel mondo, il \textit{Dasein} può
perdersi in esso, diventare una cosa fra le cose e
vanificare la sua trascendenza, la sua possibilità
progettuale, il suo essere sempre oltre.}

\textbf{Di fronte alla totale apertura del suo progettarsi e
alle infinite possibilità davanti a sé, l'esserci sperimenta
la condizione dell'angoscia. L'angoscia è ambivalente:
è esperienza di libertà ma al tempo stesso esperienza del
nulla, apre a infinite possibilità ma anche alla loro
indeterminatezza. Di fronte a questa incertezza,
l'esserci è spinto a cercare un posto dove "sentirsi
a casa propria": quel luogo è il mondo degli
utilizzabili e delle cose intramondane di cui è
spinto a occuparsi e prendersi cura.
Questa "fuga dalla libertà" è detta da Heidegger
"decadimento (spesso tradotto con il termine deiezione)":
la condizione in cui il \textit{Dasein} si trova quotidianamente, perché è da sempre presso il mondo.} 

Questo perdersi nelle cose mondane porta alla perdita di
autenticità, la possibilità di essere se stesso, libero,
in quanto si è determinati dalle cose del mondo.
Anche \textbf{la cura degli altri può portare a questa
condizione esistenziale di inautenticità: lo
smarrimento nelle relazioni sociali, nell'anonimità,
nella sottomissione al collettivo.} Qui si tratta
dell'influenza negativa della massa collettiva,
non di determinati altri. \textbf{La rinuncia alla
propria individualità è delegare ad altri ciò
che "si" deve dire o fare, dissolvendosi così nel
modo d'essere degli altri e generando
deresponsabilizzazione. In questa situazione,
il discorso (modalità fondamentale della nostra
comprensione del mondo) diventa chiacchiera.}

La sfera pubblica (tradotta anche come "pubblicità") è
medietà, livellamento, omologazione, e non confronto
di idee e apertura alle ragioni altrui.
\textbf{La soluzione heideggeriana è l'isolamento
esistenziale, la decisione individuale di
scegliere se stesso: non nel senso nietzscheano
della propria autoaffermazione, ma nella presa
d'atto della propria finitezza.} Questo è il
significato dell'\textbf{"essere-per-la-morte"}: l'assunzione
consapevole della mortalità e \textbf{la sua anticipazione
esistenziale nella pratica di vita, di fronte al
pericolo di perdersi nel mondo quel precorrimento
comporta infatti la de-assolutizzazione delle cose
mondane, così come la responsabilità della propria
esistenza, di contro alla deresponsabilizzazione
sociale. In ciò consiste la "decisione", l'assunzione di una condotta di vita che mantenga
la strutturale apertura del Dasein e gli impedisca
di perdersi nel mondo.} La valorizzazione dell'intersoggettività si capovolge così in
"solipsismo esistenziale". La mancanza di una
teoria positiva delle relazioni intersoggettive
fa si che la libertà dell'esserci si risolva in
una libertà dal mondo.

\section{Il secondo Heidegger}

Nell'ermeneutica precedente ad Heidegger, il linguaggio era
prevalentemente visto come strumento per
comunicare da parte di quei soggetti il cui incontro
determina appunto l'interpretazione del mondo. La
parola serviva a esprimere la propria soggettività, mentre
l'interpretazione era assunta come l'azione di un altro
individuo che risaliva dalla parola all'intenzione che
l'aveva prodotta. Perciò l'ermeneutica era vista soprattutto
come pratica soggettiva.

Con Heidegger la scoperta della pre-comprensione comporta
la messa in discussione dell'impianto soggettivistico
sopra descritto: \textbf{non è la coscienza la sede unica dell'interpretare}. La struttura di "pre" che condiziona
il \textit{Dasein} e il suo processo interpretativo fa si che
la comprensione non sia un processo nelle mani del
soggetto (come l'avevano inteso precedentemente Dilthey e
Schleiermacher), ma sia fondamentalmente sottratto a
esso. Ciò conferma \textbf{la struttura "gettata" del
\textit{Dasein}, l'impossibilità di progettarsi come soggetto
autonomo.}

\textbf{Nelle opere successive a "Essere e tempo",
Heidegger radicalizza questo carattere non
soggettivo dell'interpretare e del nostro incontro
con il linguaggio. Quest'ultimo cessa di essere
un semplice tramite dell'interpretazione o strumento
dell'espressione, per acquisire una dimensione di
sostanziale indipendenza rispetto ai soggetti: il linguaggio
non segue l'intenzione comunicativa, né viene
trasceso dal processo interpretativo, ma è esso stesso a
condurre il movimento del comprendere}.

\subsection{Ontologia e linguaggio}

La "svolta" heideggeriana che si attua a partire
dalla fine degli anni venti del secolo scorso, è \textbf{la
presa d'atto dell'impossibilità di continuare a
tematizzare il mondo, la verità, l'essere a partire
dal \textit{Dasein}. L'essere non può che darsi nel luogo della
sua manifestazione, ovvero nel \textit{Dasein}, che tuttavia,
essendo gettato, è messo in condizioni di essere da
qualcosa che è altro da lui, e questo altro è l'essere.}
Il \textit{Dasein} altro non è che il luogo in cui l'essere appare
e si manifesta, confermando la tesi di "Essere e tempo";
inoltre il \textit{Dasein} indica "l'essere-nel-mondo" da parte
dell'uomo, il suo prendersi cura delle cose (familiarità
col mondo), e la sua gettatezza, il suo esser consegnato
al mondo. \textbf{Il "ci" non è solo il luogo in cui
l'uomo è gettato ma anche il luogo in cui l'essere si apre.}

Lo sforzo compiuto già in "Essere e tempo" di svincolarsi
dall'ontologia della presenza doveva perciò essere accompagnato da una critica del fondamento soggettivistico per
realizzarsi compiutamente: l'intento della metafisica
occidentale è sempre stato quello di raccogliere la
totalità intorno a un principio in grado
di manifestarne il senso, così da svelare, grazie a
questo fondamento, la natura profonda delle cose del
mondo, la contemplazione della loro presenza; tale
fondamento è, soprattutto nella filosofia moderna, il
soggetto, che Heidegger va ad attaccare.
L'esserci non può essere inteso come fondamento: in
primo luogo non è un ente-presente ma è progetto
(ovvero possibilità di essere, rapporto pratico tra \textit{Dasein} e mondo), in secondo luogo
perché non può essere fondato su se stesso.

Lo sfondo, l'essere che sta alle spalle del \textit{Dasein},
non può essere pensato né come fondamento né
come ente, cioè come una presenza: in ciò sta
la \textbf{"differenza ontologica" tra essere e ente, tra
ciò che fa essere gli enti (e dunque non può essere
daccapo un ente) e gli enti stessi.
La "differenza ontologica" si mantiene aperta rinunciando
a ricondurre l'essere all'interno delle categorie del
nostro sapere che lo costringerebbero ad
essere-questa-cosa-qui, cioè ad essere presente.}

\subsection{Metafisica, nichilismo, tecnica}

\textbf{Se l'essere non è presenza, allora può porsi solo
come assenza, come nascondimento, come sottrazione
allo sguardo oggettivante}: questa sua natura è stata
tradita sin da Platone, che ha inteso l'essere
come idea, come ciò che è massimamente manifesto.
Per Heidegger, invece, l'"\textit{aletheia}" (non-nascondimento)
continua a mantenere un rapporto con l'oblio: la
svelatezza in cui essa consiste non è totale
manifestazione allo sguardo ma consiste nel
"mostrare il velo" che circonda l'essere, nel
mostrarlo nel suo nascondersi. Solo \textbf{l'ente è il
manifesto, mentre l'essere rimane nascosto e
obliato dietro l'ente, rimane nel "mistero"}.

\textbf{L'oblio appartiene costitutivamente all'essere:
da un lato è l'unico vero svelamento dell'essere,
dall'altro porta a quella sua radicale dimenticanza
in cui consiste la metafisica}. Nella storia della
filosofia non sono gli uomini ad aver
dimenticato l'essere, ma l'essere stesso ad
essersi sottratto allo sguardo: \textbf{ciò comporta che da una tale dimenticanza non si può uscire
con un atto di volontà e che la metafisica
è diventata il nostro destino.
Destino che ha i caratteri del nichilismo, inteso
come la considerazione dell'essere come nulla}
(e non tanto come insensatezza del mondo). La metafisica fa proprio questo: sostituisce l'essere
con l'ente, ritenendolo la sola verità, e tratta
l'essere come nulla ("il \textit{nihil} del nichilismo
significa che l'essere è tenuto per nulla").
\textbf{Nell'apparire dell'ente come tale, l'essere rimane
escluso e così la verità dell'essere è dimenticata.}

\textbf{Se il nichilismo è l'essenza della metafisica,
la tecnica è la realizzazione di quel destino.
Essa è la manipolazione dell'ente e la sua
trasformazione, realizzazione più profonda
dell'istanza della metafisica (da cui solo apparentemente sembra lontana),
quella della riduzione degli enti a oggetti presenti,
disponibili e manipolabili.}
L'uomo tecnologico è un uomo metafisico, perché al
pari di quello considera \textbf{l'essere ciò che può essere
obiettivato e manipolato.}
La tecnica è una modalità dell'essere di
mostrarsi: nel produrre tecnico, c'è infatti il
"condurre fuori il nascondimento nella disvelatezza",
c'è il portare qualcosa all'apparire, lo svelare.
\textbf{Il produttore tecnico è ambivalente: da un lato
ha occhi solo che per l'ente disponibile e oblia
quel" nascosto" che sta dietro, ma dall'altro
è la modalità specifica con cui l'essere
si mostra nel nostro tempo. La tecnica
è minaccia ad una verità più originaria
ma anche "provocazione", perché "là dove c'è
pericolo, cresce anche ciò che salva".}

\subsection{L'essere come evento}


\textbf{L'essere non è sostanza né un ente collocato al di
là dell'orizzonte mondano: perciò non vi potrà
mai essere una sua totale manifestazione, proprio perché
non vi è nulla da manifestare al di là dei modi
in cui esso storicamente si mostra.
L'essere è il suo consegnarsi all'esserci, è
la sua stessa storia. Noi comprendiamo l'essere nei
modi in cui esso si da, nelle epoche che scandiscono il
suo consegnarsi}: adesso è l'epoca della tecnica, la
sua modalità di darsi, di pro-dursi.

\textbf{L'essere è evento, storia, accadere}. L'accadere è
solo la modalità misteriosa in cui l'essere si da e si
consegna alla storia, e va bandita ogni pretesa
filosofica che vuole svelarne il senso o la direzione.
\textbf{Nel farsi evento, l'essere "si appropria" dell'uomo,
facendone il luogo della sua manifestazione,
ma al tempo stesso "si consegna" a lui: vi è
reciproca appropriazione di essere ad esserci,
familiarità.}

\textbf{L'evento fondamentale in cui consiste l'essere è la
storia del suo sottrarsi, del suo mostrarsi nascondendosi:
come se l'essere abbandonasse l'ente, come se
l'evento fosse lasciare l'ente in apparente autosufficienza,
a giustificare l'oblio metafisico come modalità
in cui l'essere si da nella storia.}

\textbf{Questo consegnarsi dell'essere all'uomo nascondendosi avviene in modo privilegiato
nel linguaggio.}
Le parole sembrano appartenere a noi e non a un
essere che sta al di là di noi; tuttavia proprio in
quelle parole l'essere si rivela.\textbf{ E' proprio il linguaggio
il modo privilegiato in cui l'essere può manifestarsi
ed è per questo che noi, i parlanti, ci
costituiamo come il luogo dell'apertura
dell'essere.
Il linguaggio è il luogo della reciproca appartenenza dell'uomo e dell'essere: ci è proprio ma
al contempo ci sfugge; è strumento nelle
nostre mani ma al tempo stesso irriducibile alle
intenzioni soggettive; esso indica la cosa, ma
al tempo stesso fa risuonare parole non dette,
significati non espressi, e dunque nascondendo
la cosa che sembrava aver svelato.}

\textbf{Il linguaggio costituisce quello sfondo da cui
veniamo gettati e a cui al tempo stesso
apparteniamo, quell'orizzonte sovrasoggettivo,
già individuato in "Essere e tempo" con
il tema della gettatezza e della 
pre-comprensione.}
Adesso però, nel secondo Heidegger, \textbf{il linguaggio
non sta solo davanti all'esserci , ovvero
l'essere del mondo, ma anche alle spalle del
\textit{Dasein}, lo sfondo che lo rende possibile e da cui
prende avvio la nostra interpretazione delle cose}.
In ciò consiste il compimento del tragitto
heideggeriano dal paradigma coscienzialistico a quello
linguistico: nell'indicazione che \textbf{il soggetto dell'interpretare non solo sta aldilà della
coscienza (acquisizione già raggiunta con la
tematizzazione della pre-comprensione), ma
è costituito dal linguaggio.}


\textbf{Il linguaggio,
in quanto condizione dell'aprirsi del mondo,
diventa orizzonte trascendentale.
Trascendentale non nel senso che spiega la cosa, ovvero il linguaggio
non è condizione di fondamento:
esso si limita a lasciar essere, a
rendere la cosa presente.}
\textbf{Nessuno di noi è mai all'origine delle parole,
dato che il loro uso presuppone un precedente
ascolto}. Il guadagno della dimensione
trascendentale del linguaggio, rispetto alla
prospettiva di "Essere e tempo", si accompagna
ora alla perdita della dimensione pragmatica
e intersoggettiva che in quel l'opera il linguaggio
acquisiva grazie alla sua tematizzazione come
discorso.

\textbf{Il superamento del soggetto ha prodotto così la
sua decapitazione, perché questa coappartenenza
del dire umano e del Dire originario è
tale che l'uomo può parlare solo in quanto,
appartenendo al Dire originario, tende
a questo in ascolto per essere in grado
di ripetere solamente, cancellando così
la responsabilità esistenziale teorizzata in
"Essere e tempo", cancellata dall'irruzione
di un "destino" cui si può solo obbedire.}

\section{Sartre: L'essere e il nulla (1943)}


L'opera di  Jean-Paul Sartre (1805-80), uscita sedici anni dopo
"Essere e tempo", risente di influssi fenomenologici husserliani,
hegeliani e heideggeriani.

\subsection{L'essere}

\textbf{Anche per Sartre l'essere è l'essere del fenomeno,
senza alcun altro essere dietro ai fenomeni stessi;
tuttavia questo essere non viene costituito dalla
coscienza, ma viene considerato trascendente rispetto
all'orizzonte coscienziale. In altri termini
l'essere per Sartre non è reso possibile dal
\textit{Dasein} (a cui solo si apre l'essere secondo Heidegger),
ma rimane indipendente alla coscienza.}

La fenomenologia assume qui un carattere
anti-idealistico, da cui lo sviluppo di una teoria
dell'essere indipendente dalla coscienza.
\textbf{La "prova ontologica" di Sartre} (analoga alla
celebre prova medievale)\textbf{ mostra, partendo dall'interno
della coscienza, la necessità di postulare un
essere esterno a essa: se ogni coscienza è sempre
necessariamente coscienza di qualcosa,
non si può dare coscienza senza un oggetto da essa
intenzionato e ad essa correlato}. Husserl finiva
per smentire se stesso, secondo Sartre, quando
aggiungeva che quell'oggetto non poteva essere considerato trascendente rispetto alla coscienza ma doveva
risolversi in essa.

Sartre chiama questo essere trascendente rispetto
alla coscienza "essere in sé": ciò non significa
che esso sia necessario (come la sostanza, \textit{ousia});
se fosse necessario, sarebbe indispensabile supporre
un fondamento, facendo dipendere l'essere da qualcos'altro, e in
definitiva non sarebbe un vero e proprio essere in sé.
\textbf{L'essere non ha una ragione per esistere, non
ha alcun senso né trascendente né immanente;
è pura gratuità e insensatezza, è al di là di
qualsiasi ragione postulabile. E' pura compattezza
e indifferenza, opacità e indistinzione , senza tempo
né differenze, né molteplicità né relazione al suo interno.}
Come già sostenuto da Kierkegaard, l'esistenza viene qui vista come immediata e non razionale, altrimenti sarebbe a sua volta deducibile da altro.

Sembra riemergere quella concezione tanto criticata da
Heidegger dell'essere come presenza: quasi una
riproposizione cartesiana della \textit{res extensa},
indipendente dal pensiero, e ancor più vaga, perché
l'essere di Sartre è privo di ogni determinazione
(in Descartes abbiamo estensione e movimento).

\subsection{La coscienza e il nulla}

Per spiegare la molteplicità e la differente varietà
dei fenomeni di cui facciamo esperienza e
che stanno in contraddizione con la natura
indifferenziata dell'essere in sé, è necessario
introdurre un'altra dimensione, quella della
\textbf{coscienza.
Essa non può appartenere che alla
sfera del nulla, essendo differente dall'essere,
altrimenti anche essa farebbe parte di quella
compattezza massiccia e indifferenziata.
Perciò l'opposizione di essere e coscienza è opposizione
di essere e nulla, e la trascendenza dell'essere rispetto
alla coscienza è lo sporgere dell'essere rispetto
al nulla.}

\textbf{La coscienza secondo Sartre non è essente, non ha
sostanza, né è contrapposta all'orizzonte dei fenomeni,
dato che essa è quell'orizzonte medesimo}. Al di
là dei \textbf{fenomeni} non c'è nessun essere, né
coscienziale né noumenico; essi trascendano la
coscienza non come un essere rispetto ad un altro,
bensì \textbf{come una differenza tra essere e nulla}.

Detto ciò, è evidente che il nulla che caratterizza la coscienza non è
\textit{nihil absolutam} (se così fosse non ci sarebbe
neppure la coscienza e tutto sarebbe compattezza
indeterminata), ma \textbf{è un "nulla attivo", capace di
intervenire sull'essere introducendovi delle
negazioni.}
\textbf{La coscienza è negatività,
capace di negare l'essere modificandone la
compattezza originaria; infatti, ogni volta che la coscienza attribuisce una funzione ad un oggetto, nega tutte le altre (se un libro è un soprammobile, non è più un'alzata per il tavolo, né una cosa da leggere, \dots), ovvero la coscienza è negazione}. Questa attività
negativa della coscienza si presenta in due
modalità:

\begin{itemize}
	\item \textbf{negazione frontale dell'essere dal quale
	nasce la differenza fra soggetto e oggetto},
	cioè "sporgenza" dell'essere rispetto alla
	coscienza. Ovvero \textbf{facciamo esperienza degli
	oggetti in quanto essi "non" sono la coscienza;
	qualcosa è presente in noi in quanto differente
	da chi la conosce}. E' una negazione di tipo interno,
	perché chi nega viene colpito da questa medesima
	negazione: se io non sono l'oggetto, significa
	che ne sono mancante, che ne vengo privato
	e dunque nel mentre nego di essere quell'oggetto
	vengo colpito retroattivamente da quella
	medesima negazione. \textbf{Questa negazione di essere è sentita solo dalla coscienza, che quindi sa della propria esistenza come negatività, sa di "essere-per-sé", a differenza degli oggetti, che sono in-sé.}
	\item \textbf{La seconda negazione è quella che entrando
	dentro l'essere non ne viene a sua
	volta colpita, ma istituisce delle relazioni
	negative tra gli enti al di fuori di essa.
	E' una negazione esterna che introduce nell'essere le differenze e la molteplicità: è da
	essa che nasce il "questo", quella
	determinatezza che viene introdotta
	dalla coscienza.}
\end{itemize}

Una tale concezione di relazioni identità-differenza
poste in essere dal negativo non basta a spiegare
la genesi di un mondo di qualità, la positiva
variabilità della natura (l'esser rosso dipende
dalla qualità rossa e non della semplice negazione
del non essere un altro colore), ma giustifica
solo la molteplicità quantitativa. E' una coscienza
che può produrre solo negazioni, senza
costruire positivamente un mondo, mentre
l'essere rimane talmente indeterminato
da confondersi paradossalmente proprio con
il nulla.

\subsection{La coscienza e la libertà}

In Sartre la coscienza non ha nell'autoriflessione
il suo tratto distintivo (in Hegel ad esempio il termine
"per sé" indica l'auto riflettersi del pensiero): \textbf{il
riferirsi a sé significa l'immediata presenza a se
stesso della coscienza, senza mediazione alcuna, in modo
pre-riflessivo} (se fosse riflessione mediata dipenderebbe da altro).

Non la riflessione su di sé ma il nulla è il
vero carattere della coscienza: \textbf{essa non è determinata
da nulla, da nessuna ragione, perciò è assolutamente libera.
"L'esistenza precede l'essenza", non c'è una natura
dell'uomo che preceda e determini i suoi atti, ma
sono i suoi atti liberi a determinare la natura.
"L'uomo non è altro che ciò che si fa. Questo è
il principio dell'esistenzialismo".}

\textbf{Il fondamento della libertà sta nella radicale
assenza di fondamenti, ed essa viene a coincidere
con l'assenza di senso; la libertà ha
carattere nichilistico, perché alla sua base
c'è solo il nulla.}

Libertà non coincide con pienezza e compimento,
al contrario indica una \textbf{carenza di essere: da ciò
la sua natura desiderante, in quanto il nulla vuole diventare
essere, ma questo desiderio è destinato all'insoddisfazione}, perché nel momento in cui diventasse
essere cesserebbe di essere libertà. "E' dunque
per natura coscienza infelice senza possibile
superamento dello stato di infelicità".
La strutturale irraggiungibilità dell'essere fa si che
la meta della nostra corsa (il significato della vita)
non sia mai raggiunto, e il senso della vita
è la corsa stessa. La coscienza non è altro
che un correre e la sua meta rimane solo se
stessa.
\textbf{In questa radicale insensatezza la nostra libertà
ci carica di una responsabilità totale, proprio
perché le nostre decisioni non hanno altro movente
che la nostra libertà. }

Per Sartre la libertà
esistenziale precede qualunque riflessività e qualunque
suo esercizio consapevole.
La mancanza di autotrasparenza implica anche
l'impossibilità da parte della libertà di essere
completamente presente a se stessa, cioè di
autodeterminarsi. \textbf{Come la coscienza, anche la
libertà è "gettata", si trova ad essere libera.
Siamo liberi ma al tempo stesso all'interno
di una condizione storica e culturale che non
siamo stati noi a scegliere. Quindi la libertà
è esercitata all'interno di una situazione
data: la libertà è libera in forza del suo
nulla ma sempre limitata dalla situazione in
cui si trova a operare (limitata dal suo stato "d'essere").
"Non siamo liberi di essere liberi", perché la
libertà, non essendo autofondata non ha potuto
scegliersi, noi ci troviamo di fatto in essa.}

\textbf{La libertà ha carattere fattuale}, e proprio questo
rende possibile la libertà esistenziale, perché se
la libertà non fosse in situazione sarebbe
auto fondata, cioè dipenderebbe dall'auto trasparenza
della ragione e quindi non sarebbe vera libertà (Kierkegaard): senza situazione non avrebbe di che scegliere, non sarebbe libera.

\textbf{L'esistenza è impegno morale, ma senza valori
o principi cui legarsi e da cui dipendere.
L'unico suo riferimento resta la libertà, una
libertà abissale e vuota, che ci schiaccia nella nostra
solitudine}. L'individuo è in Sartre
fondamentalmente se stesso; ha dei vincoli con gli
altri ma, come vedremo, questi non producono
alcun quadro etico comune, anzi sono fonte
di degradazione morale.

\subsection{L'intersoggettività e il conflitto}

\textbf{Sartre prosegue la tradizione fenomenologica che vede la singola esistenza aperta e legata alle esistenze altrui}: l'essere per sé può costituirsi indipendentemente dalla presenza di un altro, "solo questo per sé non sarebbe uomo".

In continuità con Heidegger, Sartre ritiene non necessaria una dimostrazione dell'esistenza altrui, tanto se ne ha certezza.
Compito della filosofia è trovare una giustificazione di tale certezza: prima occorre però una descrizione fenomenologica dell'esperienza di un altro, di quello che Sarte chiama il "per-altri".

\textbf{L'esperienza dell'altro è l'esperienza dell'esser visto}: quando sono guardato, è allora e solo allora che vedo un altro (un oggetto invece non guarda). \textbf{Questa esperienza, per Sartre, è negativa e alienante, perché chi la subisce si vede come un oggetto (osservato dal soggetto), fa passare tramite la negazione dell'altro (in quanto la coscienza nega l'essere) da essere-per-sé a essere-in-sé, una qualsiasi cosa del mondo. Questa esperienza dello sguardo altrui genera vulnerabilità, pudore e vergogna.}

\textbf{Descritta fenologicamente l'esperienza dell'altro, diventa possibile dimostrare l'insussistenza della posizione solipsista: il fatto innegabile dell'esperienza della propria oggettività ("l'oggettità") non può dipendere da me, risulta indeducibile dal per-sé (che è pura coscienza e esperienza a sé), quindi essa può derivare solo dall'esperienza di un altro e dal suo sguardo-negazione su di me}. Gli altri devono necessariamente esistere se colui che è osservato si sente oggetto: è l'altro che m fa essere, che mi mostra la mia esteriorità (essere simpatico, essere cattivo, \dots).

\textbf{Essere oggetto significa essere degradati}, privati della propria trascendenza e quindi anche della propria libertà, e fare esperienza del nulla altrui. \textbf{Per evitare questo processo di alienazione, l'uomo  rivolge all'altro la stessa oggettivazione che l'altro ha esercitato nei nostri confronti, opera una seconda negazione dopo quella subita. Posso tornare ad essere soggetto se riduco l'altro a oggetto, impedendogli di fare ciò che ogni soggetto sa e vuole fare: ridurmi a oggetto. Degradato a oggetto, l'altro si ritrova una soggettività vuota, e con ciò mi riconquisto, perché non posso essere oggetto per un oggetto.}

\textbf{In Sartre l'intersoggettività perde i tratti di rapporto fra soggetti per trasformarsi in una relazione fra oggetti, che reiterano i loro tentativi di sottrarsi all'oggettivazione ma finiscono per riprodurla incessantemente. Perciò l'unica relazione intersoggettiva che Sarte concepisce è quella del conflitto fra soggetti che lottano per ridursi a oggetti ("L’inferno sono gli altri").}

\textbf{Questo esito paradossale è radicato nell'assunto di fondo dell'ontologia sartriana: l'essere è solo positività e la coscienza solo negatività}. Da ciò un rapporto negativo fra il per-sé e l'in-sé, ma anche fra gli stessi soggetti.

Rispetto al paradigma heideggeriano, \textbf{in "Essere e nulla" assistiamo ad una retrocessione dentro la vecchia metafisica della presenza (lo sguardo è assunto in termini teoretici, come contemplazione), in cui ciò che conta è proprio la presenza allo sguardo (riduzione del rapporto intersoggettivo a rapporto soggetto-oggetto).}

\section{Gadamer: Verità e metodo (1960)}

Gadamer (1900-2002) assume come punto di partenza la
riflessione sull'ermeneutica svolta nell'opera
heideggeriana. In particolare continua la valorizzazione
di quella "pre-struttura" che anticipa ogni
nostro interpretare, vista come la vera condizione
per il realizzarsi di una comprensione.

Egli introduce il concetto di \textbf{"esperienza
extra metodica della verità": l'accesso alla verità
non avviene astraendo metodicamente dai nostri
punti di vista (come pretende di fare la scienza
moderna), ma valorizzando la pre-comprensione già
presente in noi.}

\subsection{Precomprensione e pregiudizio}

Della pre-comprensione non ci si può liberare, ma
neppure ci si deve liberare, in quanto è essa a
tenerci in rapporto con l'oggetto da comprendere.
\textbf{Ogni oggetto vive all'interno di quella pre-struttura
ed è proprio quella sua presenza anticipata che ci
consente di comprenderlo.}

\textbf{Ogni nuova conoscenza viene sempre predeterminata
dalle conoscenze precedenti, proprio perché è in base
a queste che noi possiamo aspettarci qualcosa, così da
avviare l'interpretazione. Questa pre-comprensione
potrà essere modificata dalla comprensione vera e propria,
accrescendo la struttura della nostra pre-comprensione
nei confronti di questo o altri oggetti da interpretare:
questo è il circolo ermeneutico.}

\textbf{Compito dell'ermeneutica è rendere consapevoli di
tutte queste presupposizioni} che guidano il processo
interpretativo, al fine di  "poterle controllare"
meglio.
\textbf{Le componenti del pregiudizio sono tutto ciò che
riguarda la tradizione storica che vive all'interno dell'interprete}: i suoi giudizi e le sue opinioni sono
il risultato della sua collocazione all'interno
del contesto della tradizione.
\textbf{La comune appartenenza alla tradizione dell'interprete
e dell'interpretato rende possibile la comprensione}; al contrario,
togliendo i pregiudizi, ci liberiamo del legame con
la tradizione e dunque con la cosa interpretata.
\textbf{I pregiudizi non sono ostacoli ma condizione della
comprensione.}

\subsection{La tradizione come soggetto dell'interpretare}

Da quanto detto, \textbf{la soggettività interpretante è
relegata in secondo piano dai suoi pregiudizi}, e dalla
sedimentazione storica delle opinioni che si accumulano
nell'interprete sotto forma di pre-comprensione.
\textbf{Il vero soggetto dell'interpretare è la tradizione
storica}, e noi apparteniamo alla storia.
\textbf{Le caratteristiche fondamentali della nostra esistenza
sono: storicità, cioè appartenenza alla tradizione;
finitezza, ovvero l'impossibilità di venire a capo
della propria origine.}

La comprensione va intesa come l'inserirsi nel vivo
di un processo di trasmissione storica, nel quale
passato e presente continuamente si sintetizzano.
La comprensione è un evento oggettivo che consiste
in una circolarità immanente alla tradizione
storica, e non consiste in un rapporto fra pre-comprensione soggettiva e comprensione dell'oggetto.
\textbf{L'anticipazione di senso che guida la nostra comprensione
di senso non è un atto della soggettività, ma si
determina in base alla comunanza che ci lega
alla tradizione. Comunanza che è in continuo
atto di farsi, e che perciò determina un'esperienza ermeneutica intesa come fusione di
orizzonti, quello del presente e quello del passato}
\footnote{In Dilthey avevamo invece il trasferimento del soggetto nell'orizzonte del passato, presumendo
	di poter leggere un autore con gli occhi del suo
	tempo.}.

\subsection{La struttura dialogica dell'esperienza ermeneutica}

Il rapporto tra tradizione e soggetto è di identità e differenza, familiarità ed estraneità. \textbf{La caratteristica dell'esperienza ermeneutica è l'impossibilità di ridurre quanto accade nel movimento della trasmissione storica ad un unico discorso.} La molteplicità delle voci di una tradizione ne impediscono la riduzione a un unico macrosoggetto.

Laddove Heidegger (dopo la "svolta") riduceva il linguaggio ad un monologo (perché è altro rispetto ai parlanti), Gadamer tiene ferma la strutturale pluralità dell'esperienza ermeneutico-linguistica. E se nel concetto gadameriano di tradizione rivive l'oggettività e l'indipendenza di ciò che Heidegger chiama linguaggio, con la sua doppia natura di irriducibilità e appartenenza ai parlanti, quel \textbf{linguaggio è tuttavia declinato e pensato in una struttura dialogica.}

\textbf{Il ruolo svolto dalla pre-comprensione è quello di porre domande  agli oggetti da interpretare, che, sollecitati, possano offrire le loro risposte. L'esperienza ermeneutica è una circolarità di domanda e risposta, un dialogo tra il piano dell'interprete e quello dell'interpretato: i vari orizzonti di tempo si fondano attraverso un dialogo ermeneutico.}

\textbf{Il domandare originario da cui prende il via il processo di interpretazione, è solo in apparenza la domanda del soggetto comprendente, perché in realtà essa proviene dall'oggetto interpretato, in quanto suscita interesse o aspettative.} Ma tutto ciò costituisce la domanda che quel testo, fatto, \dots, pone a noi interpretanti: \textbf{comprendere un oggetto significa comprendere questa domanda che l'oggetto ha suscitato.}

\textbf{La struttura di domanda e risposta è costituita da un continuo rinvio del domandare e del rispondere dal testo all'interprete, un rinvio in cui non c'è mai una risposta definitiva. Il compito dell'ermeneutica è tenere aperto questo orizzonte del domandare, fare in modo che risuoni sempre un'interrogazione e ottenere quindi l'apertura della coscienza, la sua disponibilità a comprendere sempre diversamente: in ciò consiste il primato ermeneutico della domanda.}

\subsection{Natura dialogico-speculativa e natura oggettiva del linguaggio}

La struttura dialogico-interpretativa dell'ermeneutica non dipende da noi, bensì è immanente al linguaggio. La parola ha infatti natura speculativa, cioè contiene in sé infiniti significati, in un rapporto "speculare", di scambio continuo, mai concluso, con significati non detti e non espressi intenzionalmente. L'interrogare  ridà espressione a questa infinità di non-detto, di non-esplicitato, che giace sempre al di sotto dell'esplicito.

Il dialogo non è la conseguenza di una forzatura soggettiva sul linguaggio ma è immanente a quella sua struttura speculativa, che apre alla pluralità delle voci che si confrontano e richiamano tra loro. Nel domandare il sottaciuto prende la parola e ciò che appariva unidimensionale diventa polisenso\footnote{Gadamer è contro la concezione greca del linguaggio, che ne vede solo il lato oggettivo, il rapporto parola-cosa.}. \textbf{Il linguaggio ha il suo vero essere solo nell'esercizio del dialogo, nel esercizio dell'interdirsi.}

\textbf{Tuttavia per Gadamer il linguaggio non è solo in mano ai parlanti, ma possiede anche aspetti oggettivi e trans-soggettivi}: infatti spesso la domanda si impone e il domandare diventa un patire piuttosto che un agire, quando l'avvio del discorso è suscitato dall'oggetto interpretato, come già fatto notare; oppure quanto emerge dal confronto dialogico non appartiene a nessuno dei parlanti, perché il linguaggio è qualcosa di oggettivo, ma anche perché \textbf{la stessa logica del dialogo, il suo \textit{logos}, ha una struttura che si impone ai dialoganti. La razionalità dialogica gadameriana alla fine rivela, sotto i suoi tratti intersoggettivi, un logos oggettivo e indipendente dai parlanti.}

Concludendo,\textbf{ il dialogo accade sulla base di un terreno comune e condiviso, il linguaggio, nel quale i dialoganti possono riconoscersi simili e al tempo stesso diversi, assumersi la loro individuale responsabilità di reagire con un "si" o con un "no" di fronte a tesi controverse. La rimozione di questa irriducibilità individuale del linguaggio da parte di Gadamer, rischia di trasformare il dialogo in finzione e parvenza, in un "monologo del pensiero"} (come lo stesso Gadamer imputava a Hegel): accanto all'inaggirabile presenza di una ragione oggettiva, si deve riconoscere l'ineliminabile costituzione intersoggettiva del dialogo.

\section{La radicalizzazione dell'ermeneutica: Deridda}

