La filosofia dopo Hegel è stata caratterizzata
da una messa in discussione della vecchia concezio­ne del trascendentale, più che da un esplicita
tematizzazione dei nuovi orizzonti che si erano
formati. Il tema esplicito della filosofia
fino a Weber era stato la presa d'atto dell'
impossibilità di stabilire l'orizzonte originario
all'interno del quale fondare il sapere: non si
accetta più un piano trascendentale ultimo.

Con poche eccezioni, la filosofia del '900
mantiene il proprio piano filosofico al di sotto
del piano trascendentale, senza perciò poterlo
fondare e tematizzare.

Nel secolo passato abbiamo assistito all'unione
delle molteplici istanze nate dalla dissoluzione
del paradigma soggettivistico, ad un nuovo
territorio comune: il linguaggio, la dimensione
in cui il pensiero si esprime, per cui consente
un rapporto con il sapere filosofico meno pregiudicato in partenza rispetto alle strutture tematizzate dalla filosofia post-hegeliana.

\chapter{Il pensiero fenomenologico-ermeneutico}