\chapter{Il pensiero neopositivistico-analitico}
\bigskip
\bigskip
\bigskip

La determinazione dei confini tra neopositivismo
e filosofia analitica non è semplice e comunque
è un esercizio controverso: dipende molto da
quanto ampia o ristretta consideriamo la definizione
di filosofia analitica.

Secondo la versione più ampia, analitica può esser
detta una filosofia che si basa sull'analisi, cioè sull'esplicazione dei concetti e dei
presupposti inespressi, sull'argomentazione, sull'individuazione
di parallogismi e fallacie: indica più uno stile
che una scuola filosofica. Il movimento analitico
assume dei contorni definiti (nonostante i numerosi
precursori del passato, dall'\textit{Organon} di Aristotele in
avanti) con Gottlob Frege (1848-1925), George
E. Moore (1873-1958) e Bertrand Russel (1872 -1970),
ovvero con l'applicazione di quello stile filosofico
al linguaggio. Il neopositivismo sarebbe perciò una
filosofia analitica (intesa in senso ampio) caratterizzata
verso l'attenzione specifica per il linguaggio scientifico.

Secondo la versione più ristretta la filosofia analitica
nascerebbe in Inghilterra negli anni trenta con il nuovo
corso che Wittgenstein stava imprimendo al suo
pensiero. Esso si caratterizza per l'attenzione al
linguaggio nella molteplicità delle sue forme, senza
perseguirne né una formalizzazione, né un programma
scientista, né una teoria empirica della conoscenza, e
neppure il formalismo logico di Frege, Russell e Carnap.
L'attenzione è alla molteplicità dei linguaggi: religioso,
etico, metafisico, giuridico, logico-matematico; non è insomma
una filosofia del linguaggio ordinario.

Il movimento neopositivista invece nasce nel 1924
con la costituzione del "Circolo di Vienna", cui
prendono parte filosofi e scienziati, fra i quali
Moritz Schlick (1882-1936, l'ispiratore), Neurath,
Carnap, Hahn e Frank. Il suo programma è
rigorosamente empirista e caratterizzato dal
metodo dell'analisi logica, nonché dall'idea che
solo le proposizioni verificabili empiricamente
abbiano senso, ovvero le proposizioni delle scienze
empiriche. Alle riunioni del circolo la figura
di Ludwig Wittgenstein costituì un costante
oggetto di ispirazione, nonostante non fece mai
parte del movimento neopositivista.

I neopositivisti
si videro costretti, con l'avvento del nazismo,
ad emigrare nei paesi anglosassoni, dove il loro
pensiero venne a liberalizzarsi soprattutto
a contatto con il pragmatismo americano,
assumendo quei tratti, pragmatici appunto, che
la filosofia analitica inglese era andata
assumendo sulla spinta del "nuovo corso"
wittgensteiniano.
Il linguaggio è dunque il vero tema delle due scuole e ciò
che le unisce; questo studio della filosofia
neopositivista-analitica è coevo all'ermeneutica
heidegeriana, e si apre anche ad un'esigenza
di de-assolutizzare il linguaggio, di ripensarlo
in connessione con quegli elementi pragmatici di
natura extralinguistica. Il razionalismo critico
popperiano rappresenta l'avvio di quella fase
pragmatica del neopositivismo che aprirà la
strada alla condotta epistemologia post-neopositivista.


Lo stesso processo avviene all'interno della filosofia
analitica, in cui all'analisi logica si sostituiva
l'indagine del linguaggio ordinario, nel quale
svolgono un ruolo centrale le pratiche d'uso e le
abitudini linguistiche. Di tale svolta pragmatica
le "Ricerche filosofiche" di Wittgenstein
rappresentano il testo esemplare. In esso la
centralità del linguaggio si riconferma proprio
a partire dal riconoscimento delle sue
componenti non linguistiche: prosecuzione
ideale è la filosofia post-analitica, tipica del
pensiero americano del secondo Novecento,
caratterizzata dall'intreccio di linguaggio,
naturalismo, realismo e pragmatismo.

\section{Wittgenstein: Tractatus logico-philosophicus (1921)}

\subsection{"Il mondo è tutto ciò che accade"}

La prima proposizione del Tractatus enuncia
fin dall'inizio il radicale approccio antisoggettivistico
di Ludwig Wittgenstein (1880-1951). Del
mondo non si danno né soggetti, né ragioni, né norme,
né tanto meno enti metafisici; esso è solo un accadere
di fatti e non di cose. Gli  enti del mondo ci appaiono
non come entità isolate, astratte dal loro accadere e
determinabili per sé sole, ma sempre e solo in connessione
tra loro, ovvero come fatti. Un fatto è un evento
complesso, che noi possiamo scomporre raggiungendo
la dimensione delle cose e degli enti (gli "oggetti"),
perdendo però la sua dimensione reale, il suo essere
un evento reale. Gli oggetti non appartengano
alla realtà: essi non accadano (come il fatto) e tuttavia
senza di essi non potrebbero sussistere nemmeno i
fatti. Dunque gli oggetti sono solo la possibilità
del mondo reale, ovvero il mondo prima della sua
configurazione fattuale, l'insieme di tutte le
possibilità. Un oggetto è reale solo
in quanto è accaduto, cioè solo in quanto è
"in" un fatto.

L'ontologia del Tractatus è un'ontologia dell'accaduto,
non fondata sull'ente; l'accadimento da realtà, ed è
per definizione puramente casuale, senza ordine, né
leggi né necessità. Tradotta alla lettera dal tedesco la prima
proposizione del Tractatus è: "il mondo è tutto ciò
che è il caso".
Ritorna qui l'assoluta insensatezza del mondo, tema
già annunciato da Nietzsche e ripreso anche
da Heidegger: "La credenza del nesso causale è la
superstizione". Le leggi scientifiche sono una nostra
forzatura dell'accadere attuale: il mondo è solo
la totalità dei fatti, mentre nessi logici, leggi
necessarie, rapporti di causa, non appartengono al
mondo.

Tutto ciò non può non coinvolgere il soggetto, ovvero
il punto di vista a partire dal quale i fatti si mostrano
come tali. Se il mondo è la totalità dei fatti,
o il soggetto appartiene ad essi (cioè è un accaduto
come gli altri) o non appartiene al mondo: per Wittgenstein
il soggetto non può far parte del mondo, perché
ogni elemento del mondo può essere osservato e
dunque ridotto a fatto. Il soggetto è solo il punto di
vista a partire dal quale il mondo si manifesta,
coincide cioè con il "campo visivo", che in quanto
tale non può essere visto all'interno del campo
visivo stesso. Il pensiero che osserva i fatti non
costituisce una realtà ontologica, è solo
"l'immagine logica dei fatti", è cioè il fatto
in quanto è stato rappresentato, in quanto è reso
visibile e ha perciò assunto forma logica. Ciò non
implica l'esistenza di una seconda dimensione
al di sopra di quella fattuale, bensì indica solo
la visibilità dei fatti, il loro manifestarsi.

Come in Husserl, l'io è l'apparire del mondo,
soggetto e mondo coincidono.
Il soggetto è "il limite del mondo", la totalità del
campo visivo, dei fatti mondani raffigurabili, né
dentro né fuori il mondo.

Wittgenstein individua, all'interno dei fatti, una
classe particolare di essi con capacità raffigurative,
cioè in grado di rappresentare il mondo: le immagini.
Esse sono fatti che raffigurano altri fatti, sono
simboli. Ogni immagine, per raffigurare il mondo,
deve avere in sé quella che Wittgenstein chiama
la "forma logica", cioè la capacità di raffigurare.
Il linguaggio fa parte di queste immagini ed è
costituito da segni, grazie ai quali il pensiero
(la raffigurazione del mondo) può trovare un
mezzo sensibile con cui esprimersi. Il linguaggio
diventa il punto di vista mondano in grado di
raffigurare il mondo, ciò che nella filosofia
moderna era la funzione attribuita al soggetto.

\subsection{Le proposizioni}

"Nelle proposizioni il pensiero si esprime
sensibilmente"; ovvero "la totalità delle
proposizioni" raffigura il mondo. Perciò qualsiasi
soggetto nel mondo lo intendiamo come linguaggio:
quest'ultimo partecipa sia al mondo del pensare,
cioè del raffigurare, sia al mondo dei fatti, cioè
dell'accadere sensibile. Il linguaggio è sia immagine
logica che espressione sensibile di quell'immagine
logica, e come il soggetto e il campo visivo, non fa
cioè parte del mondo. Fatto è solo il lato sensibile
della proposizione, non quello logico.

Secondo Wittgenstein non possono esserci proposizioni
in grado di rappresentare la relazione fra una
proposizione e un fatto: ovvero "la proposizione non
può rappresentare ciò che, con la realtà, deve avere
in comune per poterla rappresentare, la forma logica.
Per poter rappresentare la forma logica dovremmo
poter situare noi stessi con la proposizione fuori
dalla logica, vale a dire, fuori del mondo".
La forma logica è sempre presupposta, essendo ciò
che ci permette di raffigurare e descrivere il
mondo: ogni volta che tentiamo di rappresentare
tale forma, la presupponiamo. La forma logica
della realtà resta immanente alla proposizione
stessa e non può esser fatta oggetto di un'altra
proposizione. Questo non esprimibile Wittgenstein
lo definisce "il mistico", ciò su cui "si deve
tacere", ciò che si mostra nel mondo ma non
è afferrabile, che non può esser detto.
Il mistico non è al di là del linguaggio, ma
lo accompagna, essendone la condizione, senza
cui il linguaggio non potrebbe raffigurare il mondo
e i fatti non troverebbero il luogo del loro
apparire.

Da ciò consegue che la capacità raffigurativa
del linguaggio non può essere ritenuta il
prodotto di un soggetto; la forma logica della
realtà non è assolutamente riconducibile ad
altro che a se stessa, non si può risalire
alla condizione del raffigurare.
L'unica relazione che posso stabilire è quella
tra soggetto psicologico e proposizioni,
cioè tra due fatti, e così perdo il
lato raffigurativo del linguaggio.
La relazione stessa non è esprimibile, e i due
termini (fatti) della relazione sono interni
al nostro campo visivo.

\subsection{Il piano trascendentale}

Il superamento della trascendentalità
soggettiva non conduce ancora il Wittgenstein
del Tractatus a un trascendentale linguistico.
Il linguaggio è si inaggirabile perché non vi
è altra descrizione possibile dei fatti che
non sia quella proposizionale; tuttavia il
mondo non è risolvibile nel linguaggio che
lo descrive, mantiene la propria indipendenza
ontologica. Le proporzioni sono il nostro
ponte verso il mondo, e il loro senso
sta nella possibilità di riferirsi ad uno
"stato di cose". La realtà è dunque
costituita solo dai fatti, che vengono riprodotti
dal linguaggio, e non modificati.
Wittgenstein tiene ferma una teoria della
verità come corrispondenza, ovvero il
linguaggio è trasparente e il mondo dei
fatti può apparirci come incontaminato
dalla mediazione linguistica.

Il vero trascendentale per Wittgenstein è
la "forma logica", la natura raffigurativa
del linguaggio ( "La logica è trascendentale").
Da un lato le proposizioni della logica
descrivono l'armatura del mondo, ma dall'altro questa descrizione rimane formale,
indica cioè le condizioni di verità del
linguaggio, ed è perciò compatibile con
tutti gli stati di cose: le proporzioni
della logica non "trattano nulla".

La condizione logica del mondo, pur
coincidendo con l'apparire del mondo, e dunque avvicinandosi al trascendentale
husserliano, se ne differenzia in modo
netto perché essa resta un apriori formale,
senza funzione costitutiva nei confronti
del mondo, che viene lasciato esattamente
come è, mentre la condizione logica è
compatibile con qualsiasi accadere.

