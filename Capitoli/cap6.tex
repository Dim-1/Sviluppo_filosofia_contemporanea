\chapter{Il pensiero neopositivistico-analitico}
\bigskip
\bigskip
\bigskip

La determinazione dei confini tra neopositivismo
e filosofia analitica non è semplice e comunque
è un esercizio controverso: dipende molto da
quanto ampia o ristretta consideriamo la definizione
di filosofia analitica.

Secondo la versione più ampia, analitica può esser
detta una filosofia che si basa sull'analisi, cioè sull'esplicazione dei concetti e dei
presupposti inespressi, sull'argomentazione, sull'individuazione
di parallogismi e fallacie: indica più uno stile
che una scuola filosofica. Il movimento analitico
assume dei contorni definiti (nonostante i numerosi
precursori del passato, dall'\textit{Organon} di Aristotele in
avanti) con Gottlob Frege (1848-1925), George
E. Moore (1873-1958) e Bertrand Russel (1872 -1970),
ovvero con l'applicazione di quello stile filosofico
al linguaggio. Il neopositivismo sarebbe perciò una
filosofia analitica (intesa in senso ampio) caratterizzata
verso l'attenzione specifica per il linguaggio scientifico.

Secondo la versione più ristretta la filosofia analitica
nascerebbe in Inghilterra negli anni trenta con il nuovo
corso che Wittgenstein stava imprimendo al suo
pensiero. Esso si caratterizza per l'attenzione al
linguaggio nella molteplicità delle sue forme, senza
perseguirne né una formalizzazione, né un programma
scientista, né una teoria empirica della conoscenza, e
neppure il formalismo logico di Frege, Russell e Carnap.
L'attenzione è alla molteplicità dei linguaggi: religioso,
etico, metafisico, giuridico, logico-matematico; non è insomma
una filosofia del linguaggio ordinario.

Il movimento neopositivista invece nasce nel 1924
con la costituzione del "Circolo di Vienna", cui
prendono parte filosofi e scienziati, fra i quali
Moritz Schlick (1882-1936, l'ispiratore), Neurath,
Carnap, Hahn e Frank. Il suo programma è
rigorosamente empirista e caratterizzato dal
metodo dell'analisi logica, nonché dall'idea che
solo le proposizioni verificabili empiricamente
abbiano senso, ovvero le proposizioni delle scienze
empiriche. Alle riunioni del circolo la figura
di Ludwig Wittgenstein costituì un costante
oggetto di ispirazione, nonostante non fece mai
parte del movimento neopositivista.

I neopositivisti
si videro costretti, con l'avvento del nazismo,
ad emigrare nei paesi anglosassoni, dove il loro
pensiero venne a liberalizzarsi soprattutto
a contatto con il pragmatismo americano,
assumendo quei tratti, pragmatici appunto, che
la filosofia analitica inglese era andata
assumendo sulla spinta del "nuovo corso"
wittgensteiniano.
Il linguaggio è dunque il vero tema delle due scuole e ciò
che le unisce; questo studio della filosofia
neopositivista-analitica è coevo all'ermeneutica
heidegeriana, e si apre anche ad un'esigenza
di de-assolutizzare il linguaggio, di ripensarlo
in connessione con quegli elementi pragmatici di
natura extralinguistica. Il razionalismo critico
popperiano rappresenta l'avvio di quella fase
pragmatica del neopositivismo che aprirà la
strada alla condotta epistemologia post-neopositivista.


Lo stesso processo avviene all'interno della filosofia
analitica, in cui all'analisi logica si sostituiva
l'indagine del linguaggio ordinario, nel quale
svolgono un ruolo centrale le pratiche d'uso e le
abitudini linguistiche. Di tale svolta pragmatica
le "Ricerche filosofiche" di Wittgenstein
rappresentano il testo esemplare. In esso la
centralità del linguaggio si riconferma proprio
a partire dal riconoscimento delle sue
componenti non linguistiche: prosecuzione
ideale è la filosofia post-analitica, tipica del
pensiero americano del secondo Novecento,
caratterizzata dall'intreccio di linguaggio,
naturalismo, realismo e pragmatismo.

\section{Wittgenstein: Tractatus logico-philosophicus (1921)}

\subsection{"Il mondo è tutto ciò che accade"}
