\chapter{Lo storicismo tedesco}

La crisi del sistema hegeliano rese improponibile il vecchio concetto metafisico di ragione oggettiva: non nel senso che non si riconoscerà più all'ambito degli eventi oggettivi una qualche forma di ordine e razionalità, ma le nuove figure di ragione oggettiva non avranno più il carattere fondamentale di quella vecchia, cioè la totale riducibilità alla logica.

\textbf{In questo quadro a fine '800 il clima intellettuale ritorna a Kant}, come risultato della dissoluzione del suo antagonista idealistico: si afferma un concetto soggettivo di ragione, in cui la razionalità della natura e della storia ha il suo fondamento nella costituzione-costruzione soggettiva di esse, e non nella presunta razionalità della natura e della storia.
Il neokantismo tiene conto delle critiche dell'idealismo a Kant (critica al concetto di cosa in sé) e non deve pervenire nuovamente alla sintesi soggetto-oggetto (come avvenuto nell'idealismo).

\textbf{La cosa in sé viene interpretata non come una realtà esistente al di là del pensiero, ma come un limite della conoscenza, così che la trascendentalità diventa un segno non di assolutezza ma di finitezza}. La ragione non "signoreggia" la totalità delle cose, ma è un processo che organizza una parte limitata di una realtà molto più grande e complessa. \textbf{L'abbandono del polo oggettivo per quello soggettivo comporta l'abbandono dell'oggetto al non-razionale e all'inconoscibile.}

All'interno del clima neokantiano si colloca \textbf{lo storicismo tedesco: per esso la razionalità della storia non è più proprietà in sé dei fatti, ma solo ricostruzione soggettiva degli eventi, che non può svelare nessuna essenza  o senso ultimo della storia.}

Lo storicismo radicalizza ulteriormente la soggettivazione della ragione, sotto l'influenza della critica nietzscheana: \textbf{è ripensata la stessa trascendentalità del soggetto. Infatti, se la ragione non riesce a ricondurre a sé la totalità dei processi oggettivi, questi finiranno per condizionare la stessa soggettività, penetrando al suo interno per cambiarne la natura. E' quindi messa in dubbio la capacità della ragione di ordinare il mondo e il soggetto perde la sua essenza logica, esponendosi a quel mondo vitale irrazionale che il neokantismo aveva pensato di tenere a distanza.}

\section{Dilthey e L'introduzione alle scienze dello spirito (1883)}

\textbf{Il punto di partenza di Wilhelm Dilthey (1833-1911) è la consapevolezza della radicale storicità del soggetto, il suo appartenere ad un orizzonte storico che lo determina, che gli impedisce di affermare la propria superiorità e indipendenza delle cose che lo circondano.}

Ciò che caratterizza  in modo eminente l'individualità umana non è ne la coscienza ne la logica, ma quel complesso pratico-spirituale che Dilthey definisce vita, non riducibile alla ragione. In Hegel lo spirito contiene la vita al proprio interno e si pone superiore ad essa, mentre in Dilthey la vita non è riducibile a quella logicità che per Hegel è la verità dello spirito. La molteplicità delle manifestazioni vitali non è rinchiudibile dentro un sapere concettuale, mentre lo spirito hegeliano era alla fine trasparente al sapere (in ciò consisteva appunto il sapere assoluto).

La relazione del soggetto vitale con il mondo non sarà di tipo teorico-neutrale, come viene inteso dalla metodologia scientifica, ma di tipo pratico-vitale. \textbf{Dilthey indica non già nella percezione (atto solo conoscitivo) ma nell'esperienza vissuta ("Erlebnis") la modalità primaria attraverso cui ci si presenta il mondo spirituale: se la conoscenza del mondo avviene tramite esperienza, essa non può essere neutrale, bensì un'esperienza teoretica e pratica insieme, in cui l'osservatore non è separabile dai sentimenti e dai desideri}.

Nelle "scienze della natura" (conoscenza teorico-oggettivante) l'oggetto è privo di relazioni nei confronti della vita spirituale; nelle "scienze dello spirito" abbiamo a che fare con creazioni dello spirito umano (prodotti culturali, artistici, morali, politici, \dots), in cui vive lo stesso spirito che vive in noi e verso cui non sperimentiamo quella estraneità che si manifesta verso la natura. Lo scopo di queste scienze è comprendere eventi individuali, altrimenti incomprensibili all'osservazione fatta con i criteri delle scienze naturali.

\section{Dilthey e gli sviluppi successivi}

L'opera di Dilthey aprì il dibattito su:

\begin{itemize}
	\item nuova prospettiva gnoseologica;
	\item differenti basi antipositivistiche  in cui Dilthey collocava il mondo delle scienze storico-sociali.
\end{itemize}  

In seguito a critiche riguardo ad una accentuazione quasi metafisica delle differenze tra natura e storia e di muoversi all'interno dell'irrazionalismo della filosofia, Dilthey modificò la sua posizione di partenza, orientandosi all'individualità come elemento caratterizzante le scienze dello spirito ("Contributo allo studio dell'individualità" 1896).

Nella storia l'elemento decisivo è rappresentato dall'individualità, e \textbf{la pretesa idealistica e positivistica di poter dire il senso universale della storia deve essere rimessa in discussione; perciò non deve esserci nessuna pretesa di affermare il significato ultimo della storia, e deve essere rifiutata l'idea di progresso continuo in cui l'epoca successiva è vista come superamento della precedente}.

Dilthey compie una "critica della ragione storica", ovvero esplora la facoltà dell'uomo di conosce se stesso, la società e la storia che egli crea: estende la critica kantiana della ragione alle scienze dello spirito, contro la filosofia della storia di stampo idealistico e positivistico.

Negli ultimi scritti (1905-1911) emerge l'improponibilità di un senso complessivo della storia. Infatti:

\begin{itemize}
	\item la storia ha struttura vitalistica, irrazionale, cioè \textbf{l'evento storico non può essere un dato, si oppone alla pretesa razionalistica di comprenderlo fino in fondo};
	\item il senso complessivo sarebbe raggiungibile esclusivamente da un soggeto in grado di sollevarsi al di sopra della storia stessa. Ma \textbf{ogni prospettiva è determinata dalla storia, da un punto di vista intrastorico. Alla filosofia della storia Dilthey oppone la coscienza storica, del nostro essere storici, dunque il relativismo.}
\end{itemize} 

\textbf{Che la storia abbia un senso dipende dalla connessione dinamica in cui ogni elemento singolo è posto in rapporto con gli altri e con la totalità; senso che non è depositato nell'oggettività del mondo ma proviene dalla vita stessa, cioè dall'esperienza vissuta (Erlebnis) e dalla nostra comprensione, ovvero dalla nostra interpretazione con cui ci rapportiamo alla storia stessa}.

