\chapter{Lo storicismo tedesco}

La crisi del sistema hegeliano rese improponibile il vecchio concetto metafisico di ragione oggettiva: non nel senso che non si riconoscerà più all'ambito degli eventi oggettivi una qualche forma di ordine e razionalità, ma le nuove figure di ragione oggettiva non avranno più il carattere fondamentale di quella vecchia, cioè la totale riducibilità alla logica.

\textbf{In questo quadro a fine '800 il clima intellettuale ritorna a Kant}, come risultato della dissoluzione del suo antagonista idealistico: si afferma un concetto soggettivo di ragione, in cui la razionalità della natura e della storia ha il suo fondamento nella costituzione-costruzione soggettiva di esse, e non nella presunta razionalità della natura e della storia.
Il neokantismo tiene conto delle critiche dell'idealismo a Kant (critica alla concetto di cosa in sé) e non deve pervenire nuovamente alla sintesi soggetto-oggetto (come avvenuto nell'idealismo).

\textbf{La cosa in sé viene interpretata non come una realtà esistente al di là del pensiero, ma come un limite della conoscenza, così che la trascendentalità diventa un segno non di assolutezza ma di finitezza}. La ragione non "signoreggia" la totalità delle cose, ma è un processo che organizza una parte limitata di una realtà molto più grande e complessa. \textbf{L'abbandono del polo oggettivo per quello soggettivo comporta l'abbandono dell'oggetto al non-razionale e all'inconoscibile.}

All'interno del clima neokantiano si colloca \textbf{lo storicismo tedesco: per esso la razionalità della storia non è più proprietà in sé dei fatti, ma solo ricostruzione soggettiva degli eventi, che non può svelare nessuna essenza  o senso ultimo della storia.}

Lo storicismo radicalizza ulteriormente la soggettivazione della ragione, sotto l'influenza della critica nietzschiana: \textbf{è ripensata la stessa trascendentalità del soggetto. Infatti, se la ragione non riesce a ricondurre a sé la totalità dei processi oggettivi, questi finiranno per condizionare la stessa soggettività, penetrando al suo interno per cambiarne la natura. E' quindi messa in dubbio la capacità della ragione di ordinare il mondo e il soggetto perde la sua essenza logica, esponendosi a quel mondo vitale irrazionale che il neokantismo aveva pensato di tenere a distanza.}