\chapter{Lo storicismo tedesco}

La crisi del sistema hegeliano rese improponibile il vecchio concetto metafisico di ragione oggettiva: non nel senso che non si riconoscerà più all'ambito degli eventi oggettivi una qualche forma di ordine e razionalità, ma le nuove figure di ragione oggettiva non avranno più il carattere fondamentale di quella vecchia, cioè la totale riducibilità alla logica.

\textbf{In questo quadro a fine '800 il clima intellettuale ritorna a Kant}, come risultato della dissoluzione del suo antagonista idealistico: si afferma un concetto soggettivo di ragione, in cui la razionalità della natura e della storia ha il suo fondamento nella costituzione-costruzione soggettiva di esse, e non nella presunta razionalità della natura e della storia.
Il neokantismo tiene conto delle critiche dell'idealismo a Kant (critica al concetto di cosa in sé) e non deve pervenire nuovamente alla sintesi soggetto-oggetto (come avvenuto nell'idealismo).

\textbf{La cosa in sé viene interpretata non come una realtà esistente al di là del pensiero, ma come un limite della conoscenza, così che la trascendentalità diventa un segno non di assolutezza ma di finitezza}. La ragione non "signoreggia" la totalità delle cose, ma è un processo che organizza una parte limitata di una realtà molto più grande e complessa. \textbf{L'abbandono del polo oggettivo per quello soggettivo comporta l'abbandono dell'oggetto al non-razionale e all'inconoscibile.}

All'interno del clima neokantiano si colloca \textbf{lo storicismo tedesco: per esso la razionalità della storia non è più proprietà in sé dei fatti, ma solo ricostruzione soggettiva degli eventi, che non può svelare nessuna essenza  o senso ultimo della storia.}

Lo storicismo radicalizza ulteriormente la soggettivazione della ragione, sotto l'influenza della critica nietzscheana: \textbf{è ripensata la stessa trascendentalità del soggetto. Infatti, se la ragione non riesce a ricondurre a sé la totalità dei processi oggettivi, questi finiranno per condizionare la stessa soggettività, penetrando al suo interno per cambiarne la natura. E' quindi messa in dubbio la capacità della ragione di ordinare il mondo e il soggetto perde la sua essenza logica, esponendosi a quel mondo vitale irrazionale che il neokantismo aveva pensato di tenere a distanza.}

\section{Dilthey e L'introduzione alle scienze dello spirito (1883)}

\textbf{Il punto di partenza di Wilhelm Dilthey (1833-1911) è la consapevolezza della radicale storicità\footnote{la qualità  di quanto è soggetto ad un divenire storico.} del soggetto, il suo appartenere ad un orizzonte storico che lo determina, che gli impedisce di affermare la propria superiorità e indipendenza dalle cose che lo circondano.}

Ciò che caratterizza  in modo eminente l'individualità umana non è ne la coscienza ne la logica, ma quel complesso pratico-spirituale che Dilthey definisce vita, non riducibile alla ragione. In Hegel lo spirito contiene la vita al proprio interno e si pone superiore ad essa, mentre in Dilthey la vita non è riducibile a quella logicità che per Hegel è la verità dello spirito. La molteplicità delle manifestazioni vitali non è rinchiudibile dentro un sapere concettuale, mentre lo spirito hegeliano era alla fine trasparente al sapere (in ciò consisteva appunto il sapere assoluto).

La relazione del soggetto vitale con il mondo non sarà di tipo teorico-neutrale, come viene inteso dalla metodologia scientifica, ma di tipo pratico-vitale. \textbf{Dilthey indica non già nella percezione (atto solo conoscitivo) ma nell'esperienza vissuta ("Erlebnis") la modalità primaria attraverso cui ci si presenta il mondo spirituale: se la conoscenza del mondo avviene tramite esperienza, essa non può essere neutrale, bensì un'esperienza teoretica e pratica insieme, in cui l'osservatore non è separabile dai sentimenti e dai desideri}.

Nelle "scienze della natura" (conoscenza teorico-oggettivante) l'oggetto è privo di relazioni nei confronti della vita spirituale; nelle "scienze dello spirito" abbiamo a che fare con creazioni dello spirito umano (prodotti culturali, artistici, morali, politici, \dots), in cui vive lo stesso spirito che vive in noi e verso cui non sperimentiamo quella estraneità che si manifesta verso la natura. Lo scopo di queste scienze è comprendere eventi individuali, altrimenti incomprensibili all'osservazione fatta con i criteri delle scienze naturali.

\section{Dilthey e gli sviluppi successivi}

L'opera di Dilthey aprì il dibattito su:

\begin{itemize}
	\item nuova prospettiva gnoseologica;
	\item differenti basi antipositivistiche  in cui Dilthey collocava il mondo delle scienze storico-sociali.
\end{itemize}  

In seguito a critiche riguardo ad una accentuazione quasi metafisica delle differenze tra natura e storia e di muoversi all'interno dell'irrazionalismo della filosofia, Dilthey modificò la sua posizione di partenza, orientandosi all'individualità come elemento caratterizzante le scienze dello spirito ("Contributo allo studio dell'individualità" 1896).

Nella storia l'elemento decisivo è rappresentato dall'individualità che interpreta gli eventi, e \textbf{la pretesa idealistica e positivistica di poter dire il senso universale della storia deve essere rimessa in discussione; perciò non deve esserci nessuna pretesa di affermare il significato ultimo della storia, e deve essere rifiutata l'idea di progresso continuo in cui l'epoca successiva è vista come superamento della precedente}.

Dilthey compie una "critica della ragione storica", ovvero esplora la facoltà dell'uomo di conosce se stesso, la società e la storia che egli crea: estende la critica kantiana della ragione alle scienze dello spirito, contro la filosofia della storia di stampo idealistico e positivistico.

Negli ultimi scritti (1905-1911) emerge l'improponibilità di un senso complessivo della storia. Infatti:

\begin{itemize}
	\item la storia ha struttura vitalistica, irrazionale, cioè \textbf{l'evento storico non può essere un dato, si oppone alla pretesa razionalistica di comprenderlo fino in fondo};
	\item il senso complessivo sarebbe raggiungibile esclusivamente da un soggeto in grado di sollevarsi al di sopra della storia stessa. Ma \textbf{ogni prospettiva è determinata dalla storia, da un punto di vista intrastorico. Alla filosofia della storia Dilthey oppone la coscienza storica, del nostro essere storici, dunque il relativismo.}
\end{itemize} 

\textbf{Che la storia abbia un senso dipende dalla connessione dinamica in cui ogni elemento singolo è posto in rapporto con gli altri e con la totalità; senso che non è depositato nell'oggettività del mondo ma proviene dalla vita stessa, cioè dall'esperienza vissuta (Erlebnis) e dalla nostra comprensione, ovvero dalla nostra interpretazione con cui ci rapportiamo alla storia stessa}.

\section{Weber e la crisi della totalità}

In Max Weber (1864-1920) la centralità della vita pone in discussione il concetto di totalità, mentre precedentemente in Dilthey si era preso le distanze da una concezione logico-trascendentale del soggetto. Vitalità del tutto non indica un ritorno alle idee rinascimentali, piuttosto un'infinità dei suoi processi e assunzione dell'incapacità di catturarli all'interno di un sapere: le cose fanno sempre  parte di un tutto a cui tutte si riconducono, ma la crisi consiste nell'impossibilità di darne un senso e nel relazionare fra loro i vari elementi.

L'oggetto diventa incomponibile da parte del soggetto teoretico, che può solo arrivare a delle interpretazioni del reale; inoltre sotto il profilo pratico la razionalizzazione dell'oggetto risulta una provvisoria sistemazione, non già la sua effettiva riduzione al potere del soggetto. Comunque ci muoviamo, anche nel singolo, abbiamo sempre a che fare con l'infinito al suo interno, rispetto al quale siamo impotenti: da ciò \textbf{la necessità di isolare una parte di questa infinità inesauribile o una parte di un singolo fenomeno, i cui componenti risultino comprensibili. Questo isolamento è necessario per dare un significato ad una parte finita dell'infinito numero dei fenomeni che compongono il reale}; dato un soggetto finito, questo esito è inevitabile, ovvero \textbf{il significato può darsi solo nel dominio della finitezza, della condizionatezza e della limitatezza.}

\textbf{A Hegel, Weber non contesta la possibilità di raccogliere i fenomeni storici attorno ad un senso unitario, bensì rivendica l'assoluta arbitrarietà di quel senso, che invece risulta essere l'esito di una selezione di certi fatti a scapito di altri.}

\textbf{Weber fonda le scienze storico-sociali su un sapere nomologico (che ha forma di legge) finalizzandolo  però alla comprensione dei fenomeni individuali: è una "sociologia comprendente".} Questa scienza non può fondare la propria scientificità sull'esperienza vissuta di Dilthey (Erlebnis), ma sulla capacità di accettare relazioni causali tra eventi singolari, e tra "scienze della natura" e "scienze dello spirito" è da preferire interrelazione e non contrapposizione. Il comprendere storico si differenzia dai procedimenti delle scienze naturali solo in quanto lo scopo di quest'ultime è arrivare a un sapere universale.

Il processo di determinazione all'interno della totalità delle cause di un singolo evento, ha una certa dose di arbitrarietà: \textbf{in un'infinità di momenti causali, siamo costretti a scegliere basandosi su una particolare "relazione ai valori". Vengono cioè selezionati dei fatti e delle cause oggetto di indagine, che poi vengono applicati avalutativamente all'evento: è la selezione che viene svolta alla luce di certi valori, che sono particolari e contingenti. Le nostre opzioni di valore non sono più vincolate  ad un piano trascendentale, bensì vi è equivalenza nelle opzioni etiche di fondo che determinano la scelta del valore (relativismo etico).}

\textbf{Weber è attento ad evitare un'arbitrarietà assoluta nell'indagine storico-sociale}, che può avere dignità scientifica includendo la verifica empirica: \textbf{viene osservato se gli elementi da noi scelti come scatenanti un determinato evento si avvicinino ai due casi limite di "causazione adeguata"(la loro esclusione comporterebbe l'impossibilità di arrivare all'evento) o della "causazione accidentale"(l'esclusione di una certa causa non impedisce di arrivare all'evento)}. E' una verifica empirica limitata  alle cause da noi prese in considerazione, basandosi su nostri valori arbitrari, e non può estendersi alla totalità delle cause.

\textbf{Il sapere causale ha il suo fondamento nella relazione al valore, a un valore che ha perso ogni caratterizzazione universale e necessaria, e che si presenta come esito di una scelta arbitraria dipendente  da  un atteggiamento pratico.}
Quindi \textbf{il sapere nomologico} che sta alla base della conoscenza causale non ha nulla di oggettivo: esso \textbf{è un sapere costituito  di casi limite ideali ("tipi ideali"), ovvero di schemi concettuali utili alla conoscenza dell'individuale, ed essi stessi esito di  opzioni valutative del soggetto. I tipi ideali hanno funzione di archetipi la cui definizione rispecchia la cultura che li ha generati: ad esempio democrazia ideale, dispotismo ideale, \dots; questi vengono presi come metro per giudicare avalutativamente gli eventi reali.}
 
Il problema del valore sulla base del quale il soggetto interpreta gli eventi rende l'importanza della domanda tragica nietzscheana: "Chi interpreta?". Per Weber l'interpretante è lo studioso di scienze sociali, che appunto stabilisce arbitrariamente i valori che delimitano il suo campo di studio; una volta posti i limiti, l'analisi weberiana degli eventi prosegue con oggettività e seguendo la causalità, mentre Nietzsche pone la razionalità stessa come una forma di interpretazione, da cui quindi si può prescindere.

\section{Weber e la razionalità formale}

\textbf{Per Weber un agire può essere ritenuto razionale solo nella forma logico-matematica, cioè se può essere calcolato, misurato e ricondotto in termini numerici. Al contrario di questa "razionalità formale", quella "materiale" non viene soddisfatta dal calcolo}, bensì fa valere esigenze etiche, politiche, utilitaristiche, di ceto, \dots, misurando in base ad esse razionalmente rispetto al valore o rispetto ad uno scopo materiale, i risultati dell'agire (economico). Nella razionalità materiale si ripresentano le differenti versioni della vecchia filosofia pratica, che collocava l'operare della ragione nei fini o nelle intenzioni; \textbf{un agire basato sull'orientamento ai fini di un mondo o di un soggetto fortemente messi in discussione, non può essere definito razionale. L'unica razionalità può consistere esclusivamente nella coerenza interna (a dei valori) dell'agire e nella sua riducibilità in termini di calcolo.}

\textbf{Un comportamento pratico i cui fini sono assunti come incommensurabili, non potrà essere calcolato riferito ai fini stessi, bensì ai mezzi, ovvero gli strumenti da utilizzare per raggiungere quei certi scopi. Da ciò la stretta connessione istituita da Weber tra questa "razionalità rispetto allo scopo" e la razionalità formale.}

\textbf{Al contrario la "razionalità rispetto al valore" (conforme ad un certo sistema di valori}: la pura intenzione, la bellezza, il bene assoluto \dots), \textbf{quando diventa l'istanza principale dell'agire a scapito del calcolo, si rileva una forma di irrazionalismo}, perché tiene tanto minor conto delle conseguenze dell'agire, quanto più assume come incondizionato il suo valore in sé.

\textbf{Per conoscere i mezzi più adatti a raggiungere i fini prefissati, per compiere i calcoli razionalmente, Weber si serve delle scienze empiriche: in particolare queste descrivano ciò che si può fare, e mai che cosa si deve fare}.

In Weber quindi la razionalità deve limitare il suo campo di influenza e rinunciare alla totalità del reale. Ciò non significa cessare di razionalizzare i fini dell'agire: questi infatti possono rientrare nell'ambito della razionalità formale, a patto di essere considerati dei mezzi in relazione ad ulteriori fini. Ma così facendo la razionalità rispetto allo scopo avrà sempre un fine al di fuori di lei, cioè una parte del reale non razionalizzata.

\section{Weber e il processo di razionalizzazione}

\textbf{Nel mondo la ragione può assumere i caratteri di razionalità oggettivamente esistente, nonostante la prospettiva soggettivista e formalistica della razionalità. Ciò avviene se la razionalità viene introdotta nel mondo oggettivo da un agire che abbia seguito le sue regole e quindi si sia oggettivato in istituzioni sociali, economiche e politiche. Secondo Weber, in particolare dall'epoca moderna soprattutto la società occidentale è caratterizzata da un processo di razionalizzazione, ovvero dalla penetrazione di strutture razionali  nell'ambito della vita umana, soprattutto  nella sfera economica (capitalismo) e politica (stato burocratico).}

L'impresa capitalistica rappresenta la prima realizzazione della razionalità formale, perché in essa si afferma un agire economico sottratto a esigenze che non siano quelle del puro calcolo: ciò è reso passibile dalla lotta fra interessi contrapposti (concorrenza sui prezzi, calcolabile) e l'impersonalità del rapporto di lavoro.
Il calcolo capitalistico è possibile solo con un corretto funzionamento dell'amministrazione pubblica e con la certezza del diritto; perciò lo sviluppo dell'impresa capitalistica avviene contestualmente allo sviluppo dell'apparato burocratico-statale, animata dalla medesima razionalità.

\textbf{Con Weber si afferma una concezione impersonale del potere, meccanica e senza volto: la forma astratta della ragione diventa storicamente visibile e l'astrattezza formale passa dal mondo dei concetti al farsi realtà sensibile nella materialità delle relazioni economiche e politiche. Perché ciò avvenisse era necessaria la disponibilità dell'uomo moderno a regolare la propria vita sull'impersonalità, sulla calcolabilità e sulla strumentalità, e ciò è stato possibile secondo Weber dalla religiosità protestante e dall'etica del lavoro ad essa collegata ("L'etica protestante e lo spirito del capitalismo", 1904-5). Soprattutto  la dottrina calvinista considera il successo mondano  un segno della propria avvenuta elezione alla salvezza, un simbolo della grazia divina; da ciò una condotta metodica del credente per l'intera vita, che interiorizza una razionalità orientata al profitto e alla ricerca dei mezzi più efficaci per ottenerlo.}

\textbf{La razionalità (economica e politica), infondata teoreticamente, viene rifondata dalla sua vittoria pratica, dalla storia che l'ha imposta a noi come un destino. E' un processo, quello della razionalizzazione, che da un lato semplifica molte decisioni, e dall'altro lascia totalmente senza risposta le questioni ultime e ogni richiesta di senso e di libertà}: tutte le scienze vogliono dominare qualche aspetto della vita, senza tuttavia mai spiegarne il significato, che viene dato per presupposto per i loro fini. \textbf{Il mondo subisce un processo di "disincantamento": viene  dissolta ogni immagine magica o mitica-religiosa del cosmo, così come viene a sgretolarsi ogni etica tradizionale. Il mondo dei valori si ritrova privo di fondamento, e dalle ceneri del monoteismo rinasce  l'antico paganesimo sotto forma di un politeismo di valori}. Ne deriva un contemporaneo "disancoramento dei valori" della struttura motivazionale dell'individuo moderno, che si ritrova smarrito e con forti contrasti etici (causati dal politeismo dei valori); la razionalità formale non riesce con le sue strutture a comprendere la mancanza di senso da essa stessa prodotta e manifestandosi  nella sua radicale insensatezza.

Concludendo su Weber, che verrà ripreso da Popper, \textbf{il concetto di ragione non solo riduce la razionalità a un comportamento pratico (l'"atteggiamento razionalistico"), ma mostra come questo atteggiamento sia radicalmente contingente. Questa contingenza della ragione significa assenza di qualsiasi necessità teoretica che costringa  ad assumere l'atteggiamento razionale, ovvero l'assoluta equivalenza di razionalismo e irrazionalismo. Alla base della ragione, che è un atteggiamento pratico, è posto di nuovo un'altra decisione pratica, irrazionale, o più precisamente la fede che è preferibile la ragione all'irrazionalismo. La ragione, diventata contingente e artificiale, è ora posta sullo stesso piano di quel politeismo weberiano dei valori, in cui tutto è equivalente, e in cui rimane solo il potere sovrano e ingiudicabile della decisione.}

