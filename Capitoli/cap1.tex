\chapter{Critica ad Hegel}
\bigskip
\section{La realtà per Hegel e le critiche ad essa}

Hegel sostiene che la realtà storica, in quanto unione di essenza ed esistenza, di verità filosofica e di verità di fatto, è appunto la manifestazione necessaria dell'essenza; anzi, l'essenza è tale in quanto in grado di manifestarsi compiutamente, cioè in quanto realizza pienamente le sue possibilità e potenzialità nei fatti storici. La realtà per Hegel va intesa come realtà compiuta, come effettualità, come essenza che diviene reale, cui è tolto ogni dualismo con l'esistenza. In questo concetto di reale a coincidere sono anche la logica (scienza delle relazioni) e la storia: quindi la storia è razionale nel suo compimento, ed ogni evento deve essere compreso esclusivamente attraverso lo studio delle relazioni con il resto del reale: relazioni, è bene ricordarlo,  di tipo dialettico.

La critica post-hegeliana si baserà proprio sulla consapevolezza dell'insuperabile divaricazione di esistenza ed essenza.

\section{L'ultimo Schelling}

La critica di Schelling al concezione della realtà hegeliana è interna all'idealismo: viene messa in discussione la versione hegeliana dell'Assoluto come connessione di libertà e logicità. Secondo Schelling, se l'assoluto è veramente libero, il mondo non può esser dedotto, ovvero la realtà non si sviluppa attraverso regole logico-dialettiche  e alla base dell'Assoluto (Dio) c'è qualcosa di non spiegabile, di irrazionale, di non concettualizzabile.

Essendo l'Assoluto non logico, la logica non spiega la natura, che perciò è già assunta come esistente: i concetti devono derivare dalla natura e non essere ritenuti per realtà prima di ciò da cui sono astratti (mentre ricordiamo che in Hegel la natura deriva logicamente dall'idea, dal concetto). In Schelling, le manifestazioni naturali hanno "anche" carattere logico, oltre a qualcosa al di là dei limiti della ragione.

La filosofia di Schelling è positiva (quella di Hegel è negativa, cioè esprime la realtà per serie successive di negazioni, il movimento dialettico appunto), è posta, ovvero l'esistente è separato dall'essenza, non riducibile ad alcuna logica; ne consegue che la libertà non è più intesa come autotrasparenza e autodeterminazione (dovute alla logica che gli eventi seguono nella realtà hegeliana), bensì è intesa come l'accidentalità di un accadere positivo. L'essere non è più riducibile a pensiero e l'ontologia è ripensata in termini di accadimento cieco.

\section{Feuerbach e la critica materialistica}

In questo caso l'esistenza viene separata dall'essenza contrapponendo la sfera del pensare e la sfera della natura. E' utile qui ricordare che Hegel attribuisce alla realtà materiale la struttura del pensiero logico: non nel senso di identificare il mondo naturale con il mondo del soggetto (Io di Fichte), bensì la natura è vista come il momento oggettivo del pensare, seguendo perciò le leggi logico-dialettiche del pensiero stesso.

Feuerbach rivendica l'indipendenza della natura non solo dal soggetto ma anche nei confronti del pensare, allontanandosi dal mondo della logica. La logica perciò non ha più la pretesa idealistica di essere l'essenza della realtà, da cui la completa dissoluzione dell'unione hegeliana  di essenza ed esistenza.

Da questa critica, la natura non solo non è più momento del pensare (riflesso del pensiero, che segue le leggi logiche del logos), ma è il pensiero ad essere parte dello sviluppo naturale: l'Assoluto è ricondotto alla natura e all'uomo, che diventa l'essere supremo, mentre il pensiero viene deontologizzato  e ricondotto a espressione della natura, criticando l'impianto teologico nascosto dentro la filosofia hegeliana (che divinizzava il pensiero, astraendolo dal concreto uomo pensante, facendone una pura determinazione oggettiva).

Programma della "filosofia dell'avvenire" di Feuerbach  è dissolvere la teologia nell'antropologia, mostrando come l'immagine religiosa di Dio non sia altro che una rappresentazione inconsapevole che l'uomo ha di sé, proiettata al di là della sfera naturale e sensibile. Tutte le qualificazioni dell'essere divino sono quelle dell'essere umano: religione è dunque alienazione.

L'incapacità del singolo individuo di attribuirsi quei caratteri universali e supremi tipici della specie umana, finisce con il conferire quei caratteri ad un essere onnipotente e trascendente, cioè Dio. La critica della religione si risolve in un processo di emancipazione, nell'affermazione di un umanesimo radicale.

Da questo momento si prende congedo dalla nozione di pensare così come è stata assunta dall'antichità fino a Hegel, che astraeva il pensiero dal sensibile, privandolo di determinazione naturale e dandogli qualificazione teologica e sovrasensibile, impedendogli poi di venire in contatto con l'essere (naturale); l'essere sarà quindi sempre al di là del pensiero, e potrà essere esperito solo con la sensibilità, con la quale posso fare esperienza di un qualcosa che mi resiste e che è diverso da me.

In Hegel reale equivaleva a concettuale; in Feuerbach il reale è irriducibile al logico, e verrà affrontato con il metodo empiristico dei positivisti per tutto '800.

\section{Kierkegaard e la critica esistenziale all'idealismo}

Se in Feurbach l'autonomia è rivendicata dal momento naturale, Kierkegaard (1813-55) rivendica l'indipendenza dell'individuo rispetto al pensiero logico-concettuale. Egli ha un debito nei confronti della filosofia positiva di Shelling, di cui fu uditore delle lezioni tenute a Berlino nel 1841. Secondo entrambi i filosofi, Hegel ha ignorato l'esistenz quando ha deciso di racchiuderla all'interno di un sistema di categorie logiche.

Le accuse mosse da Kierkegaard alla logica hegeliana sono:

\begin{enumerate}
	
	\item Impossibilità di un inizio logico: viene attaccato il "cominciamento" della "Scienza della logica", in cui Hegel discute di un inizio senza presupposti. Nel linguaggio hegeliano, pensare qualcosa senza presupposti significa pensarlo immediato, cioè indimostrato e indipendente da qualsiasi principio. Kierkegaard obietta che tale immediatezza è solo apparentemente immediata, perché è preceduta  proprio dalla riflessione che a quella nozione immediata conduce. Poiché all'interno di un sapere logico la riflessione è inevitabile, ad un immediato sarebbe possibile solo arrestando la riflessione, altrimenti infinita di per sé. Il vero immediato da porre all'inizio non deve dipendere dalla riflessione, deve essere qualcosa di non logico. Ciò implica "l'altro dalla logica", l'impensabile, uno scacco verso le pretese logicistiche di Hegel, qualcosa di incomprensibile al pensiero puro, di irrazionale. Il vero inizio senza presupposti, che rompe con l'infinita catena delle mediazioni logiche, è quell'immediatezza in cui consiste l'esistenza.

	\item Impossibilità di un divenire logico: secondo Hegel l'Assoluto è un divenire necessario, cioè diviene con una necessità che si articola nelle categorie logiche da cui l'Assoluto è costituito. Per Kierkegaard l'identità di necessità logica e divenire è insostenibile: un divenire necessario non sarebbe vero divenire, perché ogni fase di quel movimento esisterebbe da sempre; cioè, se il passaggio tra i vari momenti era già implicito nelle premesse, tale dialettica non è un vero processo, in quanto si limita ad esplicare ciç che è da sempre. Quindi Hegel non è il filosofo del divenire, ma il suo negatore più estremo. 
	
	La tesi hegeliana che vede la necessità come "unità delle possibilità e della realtà", come cioè punto di arrivo della realtà quando questa ha sviluppato tute le sue potenzialità (quando cioè diventa "realtà svolta"), conferma la tesi di Kierkegaard: un processo il cui senso finale è la necessità non è un processo e ne ha senso collocare la possibilità tra i momenti di esso; dove c'è logicità e necessità non può esserci ne possibilità ne divenire, che quindi non è logico.
	
	\item Impossibilità di un esistenza logica: non è possibile un sistema dell'esistenza, cioè renderla un apparato logico ("C'è qualcosa che non si lascia pensare: l'esistere"). Kierkegaard intende per esistenza la singola esistenza, la singolarità esistenziale del soggetto umano, e non il puro e semplice ente. Perciò il pensiero deve prescindere dall'esistenza, perché il singolo non si lascia pensare; solo l'universale può essere pensato.
	
\end{enumerate}

La critica di Kierkegaard a Hegel non consiste, come è spesso riportato, nella rivendicazione dell'essenzialità dell'individuo rispetto all'universalità del concetto; non dimentichiamo infatti che per Hegel l'Assoluto raggiunge se stesso, cioè realizza compiutamente le proprie potenzialità solo nell'autocoscienza umana (che è anche radice di ogni individualità), e l'individuo è superiore all'astratta individualità.

 PEr Kierkegaard l'individuo è irriducibile alla logica, mentre per Hegel la massima espressione dell'autocoscienza è proprio il compimento dela logica, la realizzazione della sua natura più profonda. Per Kierkegaard la specificità dell'individuo è dal salvaguardare dal logico; l'individuo è possibilità di contro alla necessità logica, è accidentalità e scelta (nel senso di possibilità di fare). Perciò un esistente ha come unica realtà la propria realtà etica, intesa come decisioni tra alternative irriducibili. 
 
 L'esistenza non è pensabile, e l'unità di pensiero ed essere hegeliano è unita solo con l'esser pensato; quindi in Kierkegaard l'esistenza è contraddizione non risolta, "aut-aut" non conciliabile, laddove invece in Hegel è elevata ad astrazione logica e dunque sciolta nelle sue contraddizioni. In Hegel è l'universalità  a dire che cosa è l'individuo, trasformando, afferma Kierkegaard, l'uomo in animale, perché è nel regno dell'animalità che il genere è superiore all'individuo. Dunque la vera individualità sta nell'illogicità dell'esistenza individuale, che non è guidata da nessuna logicità, bensì è caratterizzata dall'angosciante possibilità di potere (possibilità che le si aprono davanti). L'infinità di possibilità che generano angoscia non è l'infinito hegeliano controllato dal movimento logico-dialettico, ma è infinità irraggiungibile, dunque angosciante, che rinchiude l'individuo nella sua finitezza di scelte finite di fronte ad infinite possibilità.
 
 \section{Critica politica all'idealismo: sinistra hegeliana e Marx}
 
 Hegel riteneva che la sua filosofia coincidesse con il sapere assoluto, e che la realtà storica fosse giunta a compimento delle sue possibilità (identità di reale e razionale, di essenza ed esistenza), con la nascita dello stato moderno successivamente alla rivoluzione francese.
 
 La sinistra hegeliana al contrario giudicava la propria epoca non ancora compiuta e dunque in conflitto con la verità filosofica: da qui l'opposizione tra quest'ultima (giunta a compimento) e la realtà storico-fattuale (non ancora realizzata in pratica come la filosofia vorrebbe), che va a costituire un altro tipo di rottura tra esistenza ed essenza.
 
 In Hegel ciò che viene tolto e superato nel processo storico-sociale in realtà non è mai veramente confutato, bensì permane come momento necessario interno alla vita dell'assoluto. Abbiamo perciò una concezione filosofica conciliante e giustificatrice verso i processi storico-sociali che descrive. Per i giovani hegeliani invece la contraddittorietà del reale è il segno della sua falsità: essi salvano il senso critico-confutativo della dialettica, rompendo però la conciliazione di idea e realtà, vietando l'innalzamento al piani dell'essenza dell'esistenza storica di fatto. Nel concreto, lo stato non deve essere spacciato come l'essenza dello stato e le sue contraddizioni come segni della sua razionalità, bensì della sua imperfezione.
 
 Ancor di più con Marx la coscienza filosofica è spinta dall'irrazionalità del reale all'opposizione nei confronti del mondo. Marx non si accontenta della sola critica verso lo stato di cose (come avviene per la sinistra hegeliana), ma afferma la necessità della prassi, intesa come modo specifico con cui si vuole colmare il divario tra filosofia e mondo ("la forza materiale deve essere abbattuta dalla forza materiale"). La filosofia compiuta deve continuare se stessa nella prassi al fine di cambiare i mondo, deve radicalizzarsi se vuole proseguire.
 
 E tale proseguimento trova il suo ambito di applicazione nella sfera economica; la realtà non è più quella logica di Hegel, ne il generico materialismo di Feuerbach o l'esistenza singolare di Kierkegaard, bensì è la realtà del lavoro, cioè l'essere umano che riproduce la sua esistenza, costruendosi i propri mezzi materiali per vivere e sopravvivere.
 
 Con il lavoro l'uomo trova la sua realizzazione, trova se stesso nei suoi prodotti, vi trasmette la sua essenza; perciò il lavoro ha natura etico-pratica, è indirizzato più al mondo interno che a quello esterno, tramite esso realizziamo la nostra esistenza.
 
 Da ciò la critica all'alienazione del lavoro, alla sottrazione dei prodotti del lavoro al lavoratore. In Hegel l'alienazione è conquista di sé (perché è un processo di oggettivazione, cioè di esteriorizzazione della propria essenza, grazie al quale il soggetto è messo in condizione di conoscerla, ponendosela davanti), in Marx è pura perdita (perché l'oggetto si manifesta come estraniazione dell'operaio).
 
 Per Marx dall'operaio vengono alienati:
 
 \begin{enumerate}
 	\item il prodotto del proprio lavoro, che è l'essenza individuale del lavoratore;
 	\item il lavoro, cioè l'alienazione in atto, cioè l'operaio non è padrone della gestione del suo lavoro;
 	\item la natura umana, ovvero l'operaio non può esprimere liberamente la propria creatività ed essenza;
 	\item i rapporti umani, giacché il prodotto del lavoro è sottratto all'operaio da un altro uomo, il capitalista, che non tratta il lavoratore come fine ma come mezzo, instaurando un rapporto non paritario.
 \end{enumerate}

Quest'ultimo punto rimanda direttamente al prodotto più radicale dell'alienazione: la lotta di classe. Alla base della teoria dell'alienazione sta dunque quella concezione allargata del soggetto, già avviata da Hegel, secondo cui la soggettività non può essere ristretta all'interiorità individuale, ma va intesa come l'intero campo delle relazioni soggetto-oggetto-soggetti. Marx ha sciolto questa concezione hegeliana dalle sue coordinate idealistiche, evidenziando come ciò che accade nel mondo (i processi di alienazione), influenzino l'umanità del soggetto. In Feuerbach natura e corporeità sono oggetti e materia, in Marx esse sono parte di quella soggettività sociale che si riproduce attraverso i processi di lavoro e l'attività pratico-relazionali.

\section{L'alternativa metafisica all'idealismo: Schopenhauer}

Schopenhauer (1788-1860) segue la linea di pensiero tracciata da Kant, così come gli idealisti, ma ponendosi in radicale opposizione con quest'ultimi.

Se assumiamo la distinzione kantiana tra fenomeno e noumeno, possiamo interpretare l'idealismo come una rigorosa applicazione del divieto kantiano di una conoscenza della cosa in sé. L'idealismo si attiene infatti alla dimensione fenomenica e alla pretesa kantiana che solo in questa dimensione vi possa essere conoscenza vera. All’interno di quella assunzione introduce poi alcune decisive correzioni: da un lato ritenendo che non solo non si possa conoscere la cosa in sé ma che questa debba essere intesa come un prodotto del pensare, e dall’altro depurando l’ambito fenomenico da ogni commistione con l’empirico e con la datità (cioè, secondo Kant, con le intuizioni sensibili). D’altra parte questa trasformazione del fenomeno rigorizza ciò che Kant aveva stabilito, ovvero che esso sia il risultato della “spontaneità dei concetti”, sia cioè un prodotto categoriale. Tolta la cosa in sé e concepito il fenomeno come appartenente all’ambito delle categorie dell’intelletto, la conseguenza inevitabile è l’assunzione della sfera dei concetti puri come il mondo della verità.

Contrariamente agli idealisti Schopenhauer depotenzia la sfera fenomenica degradandola a mera illusione. Contro l’opinione di Kant, secondo cui solo nel fenomeno si trova la verità nelle sue caratteristiche di universalità e necessità, Schopenhauer considera il fenomeno come una rappresentazione illusoria, come un arbitrario prodotto della nostra soggettività. Certo, anche per lui lo spazio e il tempo sono forme a priori che determinano la nostra rappresentazione sensibile degli oggetti, ma essi (cui egli aggiunge anche la forma della causalità, considerata anch’essa un’intuizione), invece che fornirci un’immagine vera del mondo ce ne trasmettono una deformata e illusoria ("una parvenza illusoria").

Contemporaneamente a questo depotenziamento della sfera fenomenica, Schopenhauer rivaluta la sfera noumenica, che egli, contravvenendo al divieto kantiano, ritiene del tutto accessibile alla nostra conoscenza. Abbiamo così il paradosso che quanto per Kant era mera parvenza (la pretesa metafisica di conoscere la cosa in sé e l’essenza del mondo) viene ora elevato a unica verità, e quanto per Kant era verità (l’universalità e necessità dei giudizi scientifici sul mondo fenomenico) viene degradato a mera illusione. Ovviamente la pretesa di conoscere la cosa in sé del mondo comporta la riabilitazione della metafisica dopo la demolizione kantiana. Però questa nuova metafisica si discosta assai da quella antica e moderna, caratterizzata da razionalità e da un'implicita unità di essere e pensiero; ora Schopenhauer ritiene che verità della cosa in sé stia al di là dei nostri concetti e al di là del principio di ragione. La razionalità è per Schopenhauer solo produttrice di illusioni, e perciò il superamento del mondo in cui noi viviamo, fatto di ombre e di sogni, rende necessario l’abbandono di tutto ciò che è concetto, ragione, pensiero. Ciò spiega la caratterizzazione schopenhaueriana della cosa in sé: essa non è né un essere, né una sostanza, né tantomeno un soggetto razionale. Essa è volontà; è forza anonima che vuole affermare se stessa, attraversando a tutti i livelli la natura tutta, ciecamente e liberamente, energia irrazionale e oscura ce non ha altro scopo al di fuori della propria autoaffermazione. Essa agisce.

Nessuna ragione la sostiene, nessuna giustificazione, nessuna spiegazione. E se le sue manifestazioni fenomeniche sono plurali e differenti, essa è in sé unica e indivisibile. Lo spazio, il tempo, la causalità riguardano solo il mondo come rappresentazione ma non il mondo noumenico della volontà. L’infinità degli spazi materiali, le catene causali, la molteplicità degli esseri non le appartengono, perché sono solo rappresentazioni.

La volontà vuole solo se stessa e gli esseri umani non possono che obbedirle. Illusoria è la nostra pretesa di libertà, perché ogni nostra decisione è in realtà imposta da questa volontà oscura. Noi non siamo che pedine inconsapevoli, dominati da un destino ferreo e immutabile.

Quella tesi filosofica che abbiamo visto attraversare l’intero pensiero post-hegeliano, secondo la quale la verità delle cose è radicalmente irriducibile alla ragione, al concetto e al pensiero, trova nella teorizzazione schopenhaueriana della volontà la sua espressione più radicale. Se l’idealismo tedesco, muovendo dalla dottrina kantiana delle categorie, aveva delineato una totalità razionale, ora, partendo dalle stesse premesse, si arriva a conseguenze diametralmente opposte. La catena degli eventi naturali, umani, sociali non ha alcun senso se non la ripetizione incessante di una forza cieca ed enigmatica.
Molto più radicalmente dell’ultimo Schelling, il quale aveva collocato il fondo oscuro delle cose in un contesto ancora teologico, Schopenhauer rompe decisamente i ponti con quella tradizione, mostrando come dietro alle relazioni di causa-effetto, dietro allo spazio infinito non ci sia alcun Dio benevolo, alcuna provvidenza, alcun amore che salvi, ma una forza imperscrutabile. 

Tuttavia il nichilismo schopenhaueriano non rinuncia alla pretesa di manifestare la natura ultima delle cose e pone proprio nell’insensatezza della volontà il senso del tutto. La cosa in sé può così uscire dalle nebbie dell’inconoscibilità, rendendoci accessibile l’essenza del mondo. Una tale pretesa viene legittimata da Schopenhauer ricorrendo all’intuizione, e più precisamente a quella intuizione intellettuale che Kant aveva escluso dalle facoltà a disposizione degli esseri umani (com’è noto, per lui l’unica intuizione possibile rimaneva quella empirica, in quanto l’intelletto era ritenuto capace solo di “ragionare”, cioè di giudicare, unificare, sintetizzare il dato proveniente dalle intuizioni empiriche, e mai di intuire direttamente un qualsiasi oggetto). Ancora una volta dunque l’intuizione intellettuale diventa la risorsa primaria cui attingere per fondare pretese metafisiche. Perché di questo si tratta: l’alternativa schopenhaueriana al supposto razionalismo metafisico hegeliano si presenta a sua volta come una vera e propria metafisica, anche se di segno “negativo”, una metafisica della volontà in cui l’essere, la sostanza, il soggetto, la logica, Dio vengono sostituiti dal potere di un dominio oscuro. Ora quel dominio trapassa dalla volontà universale in ogni singola manifestazione, in ogni singolo fenomeno. La volontà, benché una e indivisibile, per esercitare la sua forza deve pluralizzarsi, moltiplicarsi, articolarsi. La vita fenomenica le è necessaria, perché solo in essa può sviluppare ed esercitare a fondo il suo potere. 

Schopenhauer esprime questo passaggio dalla sfera noumenica della volontà a quella fenomenica degli enti determinati teorizzando la necessaria oggettivazione della volontà: sue oggettivazioni sono, in prima battuta, i modelli universali dai quali è costituito il mondo (una ripresa delle idee di Platone), poi le forze della natura, i corpi e infine gli individui (il luogo in cui il processo di oggettivazione trova il suo compimento e la sua massima espressione). Proprio perché la volontà è indivisibile, essa estrinseca la medesima forza in ogni sua singola manifestazione. Perciò negli individui essa si mostra in tutto il suo potere.

La volontà deve sempre manifestarsi in una pluralità di individui. Ma tale molteplicità non concerne la volontà in sé, bensì, e soltanto, i suoi fenomeni: la volontà esiste intera e indivisibile in ciascuna delle sue manifestazioni, e vede intorno a sé l’infinitamente ripetuta immagine della propria essenza. Nel "principium individuationis" ritroviamo perciò la forza della volontà e la sua capacità di condizionamento allo stato massimo. Ciò spiega l’egoismo, l’ansia di autoaffermazione, il perenne confliggere che caratterizza il genere umano. L’antagonismo, la lotta per la vita e per la morte, la brama irrefrenabile "di strappare all’altro ciò che desidera per sé", dipendono dal fatto che ogni individuo "sente di essere la volontà di vivere tutta intera". Ma questo perenne volere, bramare, desiderare genera tormento: quella soddisfazione da sempre agognata non è mai interamente raggiunta. Tutti i viventi, e l’uomo fra questi ("il più bisognoso degli esseri"), sono limitati, manchevoli, impotenti, sicché quel desiderio infinito si scontra inevitabilmente con la loro strutturale finitezza. Ogni tendere nasce da una privazione, da una scontentezza del proprio stato; è dunque, finché non soddisfatto, un soffrire; ma nessuna soddisfazione è durevole; anzi non è che il punto di partenza di un nuovo tendere. Il tendere si vede sempre impedito, sempre in lotta, è dunque un soffrire; non c’è nessun fine ultimo al tendere: dunque, nessuna misura e nessun fine al soffrire. Il desiderio si capovolge in sofferenza. La volontà è perciò essenzialmente dolore. Qui affonda la radice del pessimismo schopenhaueriano: noi non possiamo sottrarci al potere della volontà e perciò non possiamo sottrarci al dolore. La vita è sofferenza, una continua battaglia contro la morte, la tensione verso un piacere mai raggiunto, la ricerca impossibile di un ristoro, di una pace, di una tregua, in un affanno mai sopito, mai domo. La vita quanto alla forma è un perpetuo morire. 

Consideriamola ora sotto il punto di vista fisico: nello stesso modo che il nostro camminare si risolve in una successione di cadute evitate, anche la vita del nostro corpo non è che un’agonia continuamente impedita, una morte differita d’istante in istante. E infine, anche l’attività del nostro spirito non è che uno sforzo costante per cacciare la noia.  Non c’è tentativo, non c’è sforzo, non c’è alcuna buona volontà che ci possa liberare da questa condizione. Né può farlo la ragione: essa stessa infatti non è che uno strumento nelle mani della volontà, una sua oggettivazione fenomenica.

Tuttavia, pur disperando nello strumento della ragione, Schopenhauer indica una soluzione alla condizione drammatica della nostra esistenza, una via della “redenzione”, come egli ama chiamarla. E singolarmente il fondamento di questa via d’uscita assume i tratti caratteristici dell’etica, ovvero di un’opzione che tradizionalmente aveva a che fare proprio con la volontà. A dire il vero l’etica della redenzione schopenhaueriana consiste innanzitutto in una presa di coscienza. Essa nasce dalla consapevolezza dell’inutilità e vanità del nostro perenne lottare: in fondo siamo tutti il prodotto della medesima volontà, tutti attraversati dalla medesima sofferenza. Questo infatti ciascuno deve riconoscere: che le proprie pene sono "le pene infinite di tutti gli altri esseri, e farà suo tutto il dolore dell’universo". Dal superamento dell’egoismo nasce la costruzione di una comunità di disperati e di doloranti, la cui solidarietà origina dalla compassione reciproca (nel senso letterale di “patire/soffrire insieme” ma anche provare il sentimento della “pietà” per l’altro, oltre che per sé). 

L’istanza etica sembra qui contraddire le premesse dell’ontologia negativa schopenhaueriana, perché dalla compassione reciproca dovrebbe scaturire una solidarietà attiva, una pratica altruistica al posto dell’inutile egoismo, in breve un’etica intersoggettiva. In realtà quest’apertura viene subito ritrattata e richiusa da Schopenhauer. L’etica della compassione si risolve non in un agire, ma nella negazione dell’agire. Essa comporta infatti la rinuncia all’azione morale, la rinuncia a qualsiasi impegno nei confronti degli altri e al proprio sé. Le sue coordinate sono quelle del sacrificio, della castità, dell’ascesi. 

In conclusione, la via della “redenzione” può consistere unicamente nel rifiuto della volontà, nella "noluntas", nell’opposto speculare alla volontà di vita. Solo quando la nostra volontà cesserà di volere saremo finalmente liberi. Tuttavia, come abbiamo visto, la volontà è la fonte di ogni esistenza, dello spazio e del tempo, di ogni individualità e della stessa vita. Perciò il rifiuto della volontà non potrà produrre che la negazione della vita, del mondo, dell’essere stesso. Con la libera negazione, con il sacrificio della volontà, vengono soppressi anche i suoi fenomeni. Con il fenomeno si estinguono le sue forme universali, tempo e spazio; e con queste, infine, si distrugge anche la forma ultima fondamentale, il soggetto e l’oggetto. Se non c’è più volontà, non c’è più rappresentazione, non più universo. Non resta, dunque, che il nulla. Il mondo è scomparso e la tumultuosa volontà di vita si è capovolta in un "oceano di quiete".

 L’etica comunitaria, che sembrava a un certo punto la soluzione finale al problema del dolore e della volontà, viene alla fine superata a sua volta da questa ascesi verso il nirvana, o verso quello che il cristianesimo chiama stato di grazia, ma che qui alla pienezza promessa dal messaggio cristiano sostituisce il vuoto del nulla. Un tale estremo esito nichilistico è l’inevitabile conseguenza della metafisica irrazionalistica che lo precede. A dire il vero, nelle ultime righe del "Mondo come volontà e rappresentazione", Schopenhauer sembra voler reagire a quest’obiezione, accusando di nichilismo proprio "coloro che sono ancora animati dal volere". Sono loro infatti a non accorgersi che quel mondo reale, cui sono così tenacemente attaccati, non è altro che il nulla: i fenomeni, lo spazio, il tempo, il "principium individuationis" sono apparenze, la vera realtà delle cose è un’altra. Tuttavia l’alternativa a questo supposto nichilismo resta un nichilismo ancora più radicale: l’esplicita scelta del nulla e del suo "oceano di quiete". 
 
 Come vedremo, la teoria schopenhaueriana della volontà troverà prosecuzione nella filosofia di Friedrich Nietzsche, anche se in un contesto di profonda presa di distanza sia dal suo carattere metafisico sia dai suoi tratti nichilistici. Il pensiero di Nietzsche, assieme alla filosofia dello storicismo tedesco, costituisce un momento indispensabile per collegare la crisi del sistema idealistico con la filosofia a noi più contemporanea. Grazie a questi due fondamentali passaggi filosofici vengono poste le basi per la trasformazione del concetto di ragione (da oggettiva a soggettiva), per la critica della teoria tradizionale della conoscenza (attraverso la tematizzazione del carattere interpretativo di ogni esperienza) e per la definitiva messa in discussione del principio fondamentale della metafisica moderna, il principio del soggetto.