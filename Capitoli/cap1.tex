\chapter{Critica ad Hegel.}
\bigskip
\section{La realtà per Hegel e le critiche ad essa.}

Hegel sostiene che la realtà storica, in quanto unione di essenza ed esistenza, di verità filosofica e di verità di fatto, è appunto la manifestazione necessaria dell'essenza; anzi, l'essenza è tale in quanto in grado di manifestarsi compiutamente, cioè in quanto realizza pienamente le sue possibilità e potenzialità nei fatti storici. La realtà per Hegel va intesa come realtà compiuta, come effettualità, ed è l'unità di esistenza ed essenza, cui sono tolti ogni dualismo. In questo concetto di reale a coincidere sono anche la logica (scienza delle relazioni) e la storia: quindi la storia è razionale nel suo compimento, ed ogni evento deve essere compreso esclusivamente attraverso lo studio delle relazioni con il resto del reale.

La critica post-hegeliana si baserà proprio sulla consapevolezza dell'insuperabile divaricazione di esistenza ed essenza.