\chapter{Critica ad Hegel}
\bigskip
\section{La realtà per Hegel e le critiche ad essa}

Hegel sostiene che la realtà storica, in quanto unione di essenza ed esistenza, di verità filosofica e di verità di fatto, è appunto la manifestazione necessaria dell'essenza; anzi, l'essenza è tale in quanto in grado di manifestarsi compiutamente, cioè in quanto realizza pienamente le sue possibilità e potenzialità nei fatti storici. La realtà per Hegel va intesa come realtà compiuta, come effettualità, come essenza che diviene reale, cui è tolto ogni dualismo con l'esistenza. In questo concetto di reale a coincidere sono anche la logica (scienza delle relazioni) e la storia: quindi la storia è razionale nel suo compimento, ed ogni evento deve essere compreso esclusivamente attraverso lo studio delle relazioni con il resto del reale: relazioni, è bene ricordarlo,  di tipo dialettico.

La critica post-hegeliana si baserà proprio sulla consapevolezza dell'insuperabile divaricazione di esistenza ed essenza.

\section{L'ultimo Schelling}

La critica di Schelling al concezione della realtà hegeliana è interna all'idealismo: viene messa in discussione la versione hegeliana dell'Assoluto come connessione di libertà e logicità. Secondo Schelling, se l'assoluto è veramente libero, il mondo non può esser dedotto, ovvero la realtà non si sviluppa attraverso regole logico-dialettiche  e alla base dell'Assoluto (Dio) c'è qualcosa di non spiegabile, di irrazionale, di non concettualizzabile.

Essendo l'Assoluto non logico, la logica non spiega la natura, che perciò è già assunta come esistente: i concetti devono derivare dalla natura e non essere ritenuti per realtà prima di ciò da cui sono astratti (mentre ricordiamo che in Hegel la natura deriva logicamente dall'idea, dal concetto). In Schelling, le manifestazioni naturali hanno "anche" carattere logico, oltre a qualcosa al di là dei limiti della ragione.

La filosofia di Schelling è positiva (quella di Hegel è negativa, cioè esprime la realtà per serie successive di negazioni, il movimento dialettico appunto), è posta, ovvero l'esistente è separato dall'essenza, non riducibile ad alcuna logica; ne consegue che la libertà non è più intesa come autotrasparenza e autodeterminazione (dovute alla logica che gli eventi seguono nella realtà hegeliana), bensì è intesa come l'accidentalità di un accadere positivo. L'essere non è più riducibile a pensiero e l'ontologia è ripensata in termini di accadimento cieco.

\section{Feuerbach e la critica materialistica}

In questo caso l'esistenza viene separata dall'essenza contrapponendo la sfera del pensare e la sfera della natura. E' utile qui ricordare che Hegel attribuisce alla realtà materiale la struttura del pensiero logico: non nel senso di identificare il mondo naturale con il mondo del soggetto (Io di Fichte), bensì la natura è vista come il momento oggettivo del pensare, seguendo perciò le leggi logico-dialettiche del pensiero stesso.

Feuerbach rivendica l'indipendenza della natura non solo dal soggetto ma anche nei confronti del pensare, allontanandosi dal mondo della logica. La logica perciò non ha più la pretesa idealistica di essere l'essenza della realtà, da cui la completa dissoluzione dell'unione hegeliana  di essenza ed esistenza.

Da questa critica, la natura non solo non è più momento del pensare (riflesso del pensiero, che segue le leggi logiche del logos), ma è il pensiero ad essere parte dello sviluppo naturale: l'Assoluto è ricondotto alla natura e all'uomo, che diventa l'essere supremo, mentre il pensiero viene deontologizzato  e ricondotto a espressione della natura, criticando l'impianto teologico nascosto dentro la filosofia hegeliana (che divinizzava il pensiero, astraendolo dal concreto uomo pensante, facendone una pura determinazione oggettiva).

Programma della "filosofia dell'avvenire" di Feuerbach  è dissolvere la teologia nell'antropologia, mostrando come l'immagine religiosa di Dio non sia altro che una rappresentazione inconsapevole che l'uomo ha di sé, proiettata al di là della sfera naturale e sensibile. Tutte le qualificazioni dell'essere divino sono quelle dell'essere umano: religione è dunque alienazione.

L'incapacità del singolo individuo di attribuirsi quei caratteri universali e supremi tipici della specie umana, finisce con il conferire quei caratteri ad un essere onnipotente e trascendente, cioè Dio. La critica della religione si risolve in un processo di emancipazione, nell'affermazione di un umanesimo radicale.

Da questo momento si prende congedo dalla nozione di pensare così come è stata assunta dall'antichità fino a Hegel, che astraeva il pensiero dal sensibile, privandolo di determinazione naturale e dandogli qualificazione teologica e sovrasensibile, impedendogli poi di venire in contatto con l'essere (naturale); l'essere sarà quindi sempre al di là del pensiero, e potrà essere esperito solo con la sensibilità, con la quale posso fare esperienza di un qualcosa che mi resiste e che è diverso da me.

In Hegel reale equivaleva a concettuale; in Feuerbach il reale è irriducibile al logico, e verrà affrontato con il metodo empiristico dei positivisti per tutto '800.

\section{Kierkegaard e la critica esistenziale all'idealismo}

Se in Feurbach l'autonomia è rivendicata dal momento naturale, Kierkegaard (1813-55) rivendica l'indipendenza dell'individuo rispetto al pensiero logico-concettuale. Egli ha un debito nei confronti della filosofia positiva di Shelling, di cui fu uditore delle lezioni tenute a Berlino nel 1841. Secondo entrambi i filosofi, Hegel ha ignorato l'esistenz quando ha deciso di racchiuderla all'interno di un sistema di categorie logiche.

Le accuse mosse da Kierkegaard alla logica hegeliana sono:

\begin{enumerate}
	\item Impossibilità di un inizio logico:
	\item Impossibilità di un divenire logico:
	\item Impossibilità di un esistenza logica:
\end{enumerate}

