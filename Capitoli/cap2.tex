\chapter{Nietzsche e le conseguenze radicali della dissoluzione dell'idealismo}
\bigskip
\bigskip
\bigskip
\bigskip
\bigskip
\bigskip


La filosofia di Friederich Nietzsche (1844-1900), con la dottrina dell'eterno ritorno e la teoria della volontà di potenza, attesta l'esito inevitabile cui doveva condurre la dissoluzione del sapere assoluto hegeliano. \textbf{L'eterno ritorno indica che la realtà non ha più un significato, in quanto gli viene tolto qualsiasi fondamento logico}: se tutto ritorna, qualsiasi sviluppo storico è messo in discussione, ed inoltre vi è una sostanziale equivalenza del tutto, l'impossibilità di scale di valori, di gerarchie; l'idea di realtà è privata di razionalità e logicità.

\textbf{La volontà di potenza mostra che la volontà è resa libera dalla verità, o meglio fa dipendere la verità da se stessa}, una volta tolti i vincoli di presunte verità logiche o oggettive. L'idea moderna di soggettività verrà ridimensionata, in quanto anche essa fondata sulla superiorità del pensare e della ragione.

\section{Nichilismo ed eterno ritorno}

In nucleo centrale del "pensiero più abissale" di Nietzsche sta nella messa in discussione della struttura lineare del tempo e perciò la sua subordinazione a logiche che pretendano di mostrarne lo sviluppo e il progresso: nessun momento è privilegiato all'interno del tempo; se tutto ritorna, tutto è uguale, tutto è equivalente. \textbf{Ogni attimo esaurisce la totalità dell'essenza, perché tutto è attimo}.
E' un annuncio con connotati nichilistici, venendo meno ogni fondamento, ogni valore e ogni verità. L'esistenza è senza senso e senza scopo, ma ritornante: \textbf{il nichilismo è  mancanza di significato del mondo e della vita, svalorizzazione dei valori supremi. Senza "cosa in sé" (costituzione assolla morte di Dio porta con sé la morte del platonismo, cioè dell'intera configurazione morale, logica e razionale della nostra civiltà, venendo meno qualsiasi capacità di orientamento dell'uomo.uta delle cose) si è giunti al nichilismo più estremo, in cui la totalità è priva di senso.}

Questo pensiero è "terribile" solo per chi è rimasto legato alla vecchia idea metafisica di una verità in sé delle cose. \textbf{Per chi si è liberato di questa visione, l'annuncio dell'eterno ritorno è il guadagno di una nuova dimensione dell'esistenza, quella dell'innocenza del divenire: il divenire delle cose non porta da nessuna parte, non ci sono mete, ma tutto l'essere è pieno di significato, quello che gli attribuiamo.}

Il vero nichilismo è la convinzione che il mondo non sia nulla, e questo è l'esito di un processo che affonda le radici nelle origini platonico-cristiane della nostra civiltà, dove la volontà di trascendenza ha mascherato la volontà del nulla. Il mondo è stato ridotto a nulla, per fondare su questo nulla l'esistenza di Dio; successivamente la stessa civiltà creatrice di Dio ha riconosciuto la nullità di questo essere supremo, portando a compimento il nichilismo.

Il nichilismo estremo, la consapevolezza che non c'è più alcuna verità, ne al di là ne al di qua, necessaria conseguenza degli ideali finora coltivati, è comunque \textbf{il passaggio necessario verso il superuomo}, colui che porrà nuovi valori, una volta capito che non esistano valori assoluti.

La fede in Dio è legata alla fede nella ragione, alla fiducia nella struttura razionale del mondo; \textbf{la morte di Dio porta con sé la morte del platonismo, cioè dell'intera configurazione morale, logica e razionale della nostra civiltà, venendo meno qualsiasi capacità di orientamento dell'uomo.}

